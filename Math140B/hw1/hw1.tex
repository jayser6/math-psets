\documentclass[12pt]{article}

%% Basic document formatting
\usepackage{amsmath, amsthm, amssymb, amsfonts}
\usepackage{mathtools}
\usepackage{xspace}
\usepackage{thmtools}
\usepackage{graphicx}
\usepackage{setspace}
\usepackage{fancyhdr}
\usepackage{titling}
\usepackage[left=0.4in,right=0.4in,top=1in,bottom=1in]{geometry}
\usepackage{float}
\usepackage{tabularx}
\usepackage[utf8]{inputenc}
\usepackage[english]{babel}
\usepackage{framed}
\usepackage[dvipsnames]{xcolor}
\usepackage{environ}
\usepackage{tcolorbox}
\tcbuselibrary{theorems,skins,breakable}

\usepackage{enumitem}
\setlist[enumerate]{leftmargin=*}
\setlist[enumerate,1]{labelindent=\parindent}
\setlist[enumerate,2]{labelindent=0pt}

% Blackboard Bold
\newcommand{\N}{\mathbb{N}}         % natural numbers
\newcommand{\Z}{\mathbb{Z}}         % integers
\newcommand{\Zpl}{\mathbb{Z}_{+}}   % positive integers
\newcommand{\Q}{\mathbb{Q}}         % rationals
\newcommand{\Qpl}{\mathbb{Q}_{+}}   % positive Rationals
\newcommand{\R}{\mathbb{R}}         % reals
\newcommand{\Rpl}{\mathbb{R}_{+}}   % positive Reals
\newcommand{\C}{\mathbb{C}}         % complex numbers
\newcommand{\F}{\mathbb{F}}         % field

% Words
\newcommand{\st}{\text{ such that }}
\newcommand{\wrt}{\text{ with respect to }}
\newcommand{\with}{\text{ with }}
\newcommand{\ie}{\text{, i.e. }}

% Operators
\newcommand{\abs}[1]{\left|#1\right|}                   % absolute value
\newcommand{\floor}[1]{\left\lfloor #1 \right\rfloor}   % floor
\newcommand{\ceil}[1]{\left\lceil #1 \right\rceil}      % ceiling

% Algebra 
\newcommand{\Syl}{\text{Syl}}           % set of Sylow-p subgroups 
\newcommand{\<}{\langle}                % \<x\>, subgroup generated by x 
\renewcommand{\>}{\rangle}
\newcommand{\id}{\text{id}}             % identity element
\newcommand{\order}[1]{\text{o}(#1)}    % order of an element
\let\oldcong\cong
\let\oldequiv\equiv
\renewcommand{\cong}{\oldequiv}
\renewcommand{\equiv}{\oldcong}

\makeatletter % cycle
\newcommand{\cyc}[1]{(\mathbf{\cyc@process#1\relax})}
\def\cyc@process#1#2\relax{%
	#1%
	\ifx\relax#2\relax
	\else
		\,\cyc@process#2\relax
	\fi
}
\makeatother

% Linear Algebra
\newcommand{\GL}{\text{GL}}                             % general linear group
\newcommand{\SL}{\text{SL}}                             % special linear group
\newcommand{\bmat}[1]{\begin{bmatrix}#1\end{bmatrix}}   % bracketed matrix
\newcommand{\rank}{\operatorname{rank}}                 % rank
\newcommand{\nullity}{\operatorname{nullity}}           % nullity

% Topology/Analysis
\newcommand{\ball}[2]{\text{B}_{#1}(#2)}  % B_r(x): open r-balls around x
\newcommand{\diam}{\text{diam}}           % diamter of a set in metric space 

% Misc Notation
\newcommand{\defeq}{\vcentcolon=}   % :=
\newcommand{\eqdef}{=\vcentcolon}   % =: 
\renewcommand{\bf}[1]{\textbf{#1}}


\renewcommand{\thesection}{\arabic{section}.}
\renewcommand{\thesubsection}{(\alph{subsection})}

\newtheorem{claim}{Claim}
\newtheorem*{lemma}{Lemma}

%% Headers & title setup
\newcommand{\course}{Math140B}
\newcommand{\myname}{Jay Ser}
\setlength{\headheight}{14.5pt}
\pagestyle{fancy}
\fancyhf{}
\renewcommand{\headrulewidth}{0.4pt}
\lhead{\course}
\rhead{\myname}
\cfoot{\thepage}
\setlength{\droptitle}{-4em} 
\title{\course\ - HW \#1}
\author{\myname}
\date{2026.01.11}

\begin{document}
\maketitle
\thispagestyle{fancy}

%------------------------------------------------------------------------------%

\section{Q2}
Suppose $f$ is not strictly increasing on $(a, b)$, that is, $\exists c, d \in \R \st a < c < d < b$ and $f(c) \geq f(d)$. 
$f$ is differentiable on $(a, b)$ with $(c, d) \subset (a, b)$, so there is an open neighborhood containing $[c, d] \st f$ is continuous on the neighborhood. 
Of course, $f$ is differentiable on $(c, d)$, thus satisfying the hypothesis of the Mean Value Theorem. 
Applying the Mean Value Theorem to $f$ on points $c$ and $d$, $\exists x \in (c, d) \st f(d) - f(c) = (d - c)f'(x)$. 
Namely, $c < d$ and $f(c) \geq f(d)$, so $f'(x) \leq 0$, which contradicts the fact that $f$ has a positive derivative on $(a, b)$. 

Since $f$ is strictly increasing on $(a, b)$, i.e. $f(x_1) \neq f(x_2) \, \forall x_1 \neq x_2 \in (a, b)$, $f$ is injective on $(a, b)$; 
$f$ is bijection from $(a, b)$ to the image $f((a, b))$. 
Denote $f^{-1}$ as $g: f((a, b)) \rightarrow (a, b)$ and let $h(t) \defeq g(f(t)) = t$.  

First, assume $g$ is differentiable. 
By the chain rule, $h'(x) = g'(f(x)) f'(x)$.
Simultaneously, $h'(x) = 1$. 
Combining these equalities, 
$$g'(f(x)) = \frac{1}{f'(x)}$$

Symmetric reasoning shows that if $g$ is differentiable, then necessarily $g'(y) = \frac{1}{f'g(y)}$ for any $y \in f((a, b)$ 
Indeed, this calculation can be used to confirm that $g$ is indeed differentiable.
Notice that because $f$ is differentiable on any $x \in (a, b)$ and $f'(x) \neq 0$, 
$$\lim_{t \rightarrow x} \frac{1}{\frac{f(x) - f(t)}{x - t}} = \lim_{t \rightarrow x} \frac{x - t}{f(x) - f(t)}$$ 
exists, and the value of the limit is $1 / f'(x)$. 
Now, fix $\epsilon > 0$ and pick any $y \in f((a, b))$. 
Since $f$ is injective, $\exists! x \in (a, b) \st f(x) = y$. 
$\lim_{t \rightarrow x} \frac{x - t}{f(x) - f(t)} = 1/f'(x)$, so $\exists \eta > 0 \st \forall t \in \ball{\eta}{x}$, $\abs{\frac{x - t}{f(x) - f(t)} - 1/f'(x)} < \epsilon$. 

Now, $g$ is continuous, so $\exists \delta > 0 \st \forall s \in \ball{\delta}{y}$, $\abs{g(y) - g(s)} < \eta$. 
Denote $t = g(s)$. 
Since $g(y) = x$, $\abs{x - t} < \eta$ and thus the following holds: 
\begin{align*}
  \abs{\frac{g(y) - g(s)}{y - s} - \frac{1}{f'(g(y))}} = \abs{\frac{x - t}{f(x) - f(t)} - \frac{1}{f'(x)}} < \epsilon
\end{align*}
This shows that $g$ is differentiable on its entire domain with $g'(y) = 1/f'(g(y))$.  

\section{Q4}
Define 
$$g(x) \defeq c_0 x + \frac{1}{2} c_1 x^2 + \ldots + \frac{1}{n}c_{n - 1}x^n + \frac{1}{n + 1}c_n x^{n + 1}$$
Notice
\begin{itemize}
 \item $g(1) = 0$ by the hypothesis, and also $g(0) = 0$. 
 \item $g'(x) = c_0 + c_1 x + \ldots + c_{n - 1} x^{n - 1} + c_n x^n = 0$, which is the polynomial of interest in this question.  
\end{itemize}

$g$ is a polynomial, so it is infinitely differentiable everywhere. 
By the Mean Value Theorem, $\exists x_0 \in (0, 1) \st g'(x_0) (1 - 0) = g(1) - g(0)$, i.e., $g'(x_0) = 0$, as was to be shown.  

\section{Q8} 
Fix $\epsilon > 0$. 
By the definition of the derivative and the continuity of $f'$, $\forall x \in [a, b], \exists \delta_x > 0 \st \forall t \in [a, b]$ satisfying $\abs{x - t} < \delta_x$, 
$$\abs{f'(x) - \frac{f(x) - f(t)}{x - t}} < \epsilon / 2, \quad \abs{f'(x) - f'(t)} < \epsilon / 2$$

$\bigcup_{x \in [a, b]} \ball{\delta_x / 2}{x}$ form an open cover for $[a, b]$ and $[a, b]$ is compact, so $\exists x_1, \ldots, x_n \in [a, b]$ with associated neighborhoods $\ball{\delta_1 / 2}{x_1}, \ldots, \ball{\delta_n / 2}{x_n}$ whose union is an open subcover of $[a, b]$. 
Let $$\delta \defeq \min\{\delta_1 / 2, \ldots, \delta_n / 2\}$$

Then $\forall x, t \in [a, b]$ with $\abs{x - t} < \delta$,
$$\abs{x_k - t} \leq \abs{x_k - x} + \abs{x - t} \leq \delta_k$$
for some $k \in \{1, \ldots, n\}$ satisfying $\abs{x - x_k} < \delta_k / 2$, which is guaranteed to exist since the neighborhoods of the $x_i$'s form an open cover for $[a, b]$. 
Finally, because both $x, t \in \ball{\gamma_k}{x_k}$,  
$$\abs{f'(x) - \frac{f(x) - f(t)}{x - t}} \leq \abs{f'(x) - f'(x_k)} + \abs{f'(x_k) - \frac{f(x_k) - f(t)}{x_k - t}} < \epsilon / 2 + \epsilon / 2 = \epsilon$$ 
as was to be shown. 

Under the same hypothesis but with $f:[a, b] \rightarrow \R^n$, the result holds as previous. 
The easiest way to convince oneself of this fact is to use the box metric on $\R^n$. 
Then the previous calculations apply more or less immediately to the current case, with nothing more to be checked. 

\section{Q9}
By the L'Hospital's Rule, 
$$\lim_{x \rightarrow 0} \frac{f(x) - f(0)}{x - 0}$$
exists: the continuity of $f$ ensures $f(x) - f(0) \rightarrow 0$ as $x \rightarrow 0$. 
Of course, the denominator vanishes as $x \rightarrow 0$ as well. 
The derivative of the numerator with respect to $x$ is just $f'(x)$, while the derivative of the denominator is 1, so 
the L'Hospital's rule applies, and 
$$f'(0) \defeq \lim_{x \rightarrow 0} \frac{f(x) - f(0)}{x - 0} = \lim_{x \rightarrow 0} \frac{f'(x)}{1} = 3$$

\end{document}

\documentclass[12pt]{article}

%% Basic document formatting
\usepackage{amsmath, amsthm, amssymb, amsfonts}
\usepackage{mathtools}
\usepackage{xspace}
\usepackage{thmtools}
\usepackage{graphicx}
\usepackage{setspace}
\usepackage{fancyhdr}
\usepackage{titling}
\usepackage[left=0.4in,right=0.4in,top=1in,bottom=1in]{geometry}
\usepackage{float}
\usepackage{tabularx}
\usepackage[utf8]{inputenc}
\usepackage[english]{babel}
\usepackage{framed}
\usepackage[dvipsnames]{xcolor}
\usepackage{environ}
\usepackage{tcolorbox}
\tcbuselibrary{theorems,skins,breakable}

\usepackage{enumitem}
\setlist[enumerate]{leftmargin=*}
\setlist[enumerate,1]{labelindent=\parindent}
\setlist[enumerate,2]{labelindent=0pt}

% Blackboard Bold
\newcommand{\N}{\mathbb{N}}         % natural numbers
\newcommand{\Z}{\mathbb{Z}}         % integers
\newcommand{\Zpl}{\mathbb{Z}_{+}}   % positive integers
\newcommand{\Q}{\mathbb{Q}}         % rationals
\newcommand{\Qpl}{\mathbb{Q}_{+}}   % positive Rationals
\newcommand{\R}{\mathbb{R}}         % reals
\newcommand{\Rpl}{\mathbb{R}_{+}}   % positive Reals
\newcommand{\C}{\mathbb{C}}         % complex numbers
\newcommand{\F}{\mathbb{F}}         % field

% Words
\newcommand{\st}{\text{ such that }}
\newcommand{\wrt}{\text{ with respect to }}
\newcommand{\with}{\text{ with }}
\newcommand{\ie}{\text{, i.e. }}

% Operators
\newcommand{\abs}[1]{\left|#1\right|}                   % absolute value
\newcommand{\floor}[1]{\left\lfloor #1 \right\rfloor}   % floor
\newcommand{\ceil}[1]{\left\lceil #1 \right\rceil}      % ceiling

% Algebra 
\newcommand{\Syl}{\text{Syl}}           % set of Sylow-p subgroups 
\newcommand{\<}{\langle}                % \<x\>, subgroup generated by x 
\renewcommand{\>}{\rangle}
\newcommand{\id}{\text{id}}             % identity element
\newcommand{\order}[1]{\text{o}(#1)}    % order of an element
\let\oldcong\cong
\let\oldequiv\equiv
\renewcommand{\cong}{\oldequiv}
\renewcommand{\equiv}{\oldcong}

\makeatletter % cycle
\newcommand{\cyc}[1]{(\mathbf{\cyc@process#1\relax})}
\def\cyc@process#1#2\relax{%
	#1%
	\ifx\relax#2\relax
	\else
		\,\cyc@process#2\relax
	\fi
}
\makeatother

% Linear Algebra
\newcommand{\GL}{\text{GL}}                             % general linear group
\newcommand{\SL}{\text{SL}}                             % special linear group
\newcommand{\bmat}[1]{\begin{bmatrix}#1\end{bmatrix}}   % bracketed matrix
\newcommand{\rank}{\operatorname{rank}}                 % rank
\newcommand{\nullity}{\operatorname{nullity}}           % nullity

% Topology/Analysis
\newcommand{\ball}[2]{\text{B}_{#1}(#2)}  % B_r(x): open r-balls around x
\newcommand{\diam}{\text{diam}}           % diamter of a set in metric space 

% Misc Notation
\newcommand{\defeq}{\vcentcolon=}   % :=
\newcommand{\eqdef}{=\vcentcolon}   % =: 
\renewcommand{\bf}[1]{\textbf{#1}}


\renewcommand{\thesection}{\arabic{section}.}
\renewcommand{\thesubsection}{(\alph{subsection})}

\newtheorem{claim}{Claim}
\newtheorem*{lemma}{Lemma}

%% Headers & title setup
\newcommand{\course}{Math140B}
\newcommand{\myname}{Jay Ser}
\setlength{\headheight}{14.5pt}
\pagestyle{fancy}
\fancyhf{}
\renewcommand{\headrulewidth}{0.4pt}
\lhead{\course}
\rhead{\myname}
\cfoot{\thepage}
\setlength{\droptitle}{-4em} 
\title{\course\ - HW \#4}
\author{\myname}
\date{2026.02.01}

\begin{document}
\maketitle
\thispagestyle{fancy}

%------------------------------------------------------------------------------%
\section*{Question 7} 
\subsection{} 
Let $S \defeq \int_{0}^{1} f(x) dx$ as in the canonical definition of the integral. 
Showing $S = \lim_{t \rightarrow 0} \int_{t}^{1} f(x) dx$ is equivalent to showing 
$$S - \int_{t}^{1} f(x) dx \rightarrow 0$$ 
as $t \rightarrow 0$.

Fix some $t \in [0, 1]$.  
Since $f \in \mathcal{R}_{0}^{1}$, $\int_{0}^{t} f(x) dx$ exists, and   
$$S - \int_{t}^{1} f(x) dx = \int_{0}^{t} f(x) dx$$ 
Let $M \defeq \sup_{x \in [0, 1]} \abs{f(x)}$. 
Then $M \geq \sup_{x \in [0, t]} \abs{f(x)}$ and thus 
$$\abs{\int_{0}^{t} f(x) dx} \leq M(t - 0) = Mt$$
$Mt \rightarrow 0$ as $t \rightarrow 0$ (from the positive, since the space of consideration is $[0, 1]$), and $\abs{\int_{0}^{t} f(x) dx} \geq 0$, so 
$$\abs{\int_{0}^{t} f(x) dx} \rightarrow 0$$ 
as $t \rightarrow 0$. 
This implies $\int_{0}^{t} f(x) dx = 0$, hence $\lim_{t \rightarrow 0} \int_{t}^{1} f(x) dx = S$. 

\subsection{} 
Consider the two limits 
$$\int_{1}^{\infty} \frac{\sin{x}}{x} dx \hspace{5mm} \text{ and } \hspace{5mm} \int_{1}^{\infty} \abs{\frac{\sin{x}}{x}} dx$$
The former converges while the latter diverges. 
Integrating the first limit by parts with $F(x) = \frac{1}{x}$ and $G(x) = -\cos{x}$, 
$$\int_{1}^{b} \frac{\sin{x}}{x} dx = \left.-\frac{\cos{x}}{x}\right|_{1}^{b} - \int_{1}^{b} \frac{\cos{x}}{x^2} dx$$
Both terms in the sum converge as $b \rightarrow \infty$. 
The first term converges because $-\frac{1}{b} \leq \frac{\cos{b}}{b} \leq \frac{1}{b}$ and the lower and upper bounds both tend to 0 as $b \rightarrow \infty$.  
The second term converges by the Integral Test described in Q8 of Rudin: $\int_{1}^{\infty} f(x) dx$ converges if and only if $\sum_{n = 1}^{\infty} f(n)$ converges.
Here, $\abs{\frac{\cos{n}}{n^2}} \leq \frac{1}{n^2}$ and $\sum \frac{1}{n^2}$ converges, so $\sum_{n = 1}^{\infty} \frac{\cos{n}}{n^2}$ converges by the Comparison Test. 
This demonstrates that $\int_{1}^{\infty} \frac{\sin{x}}{x} dx$ converges. 

To see that $\int_{1}^{\infty} \abs{\frac{\sin{x}}{x}} dx$ does not converge, observe the following inequalities arising from basic trigonometry: 
\begin{align*}
  \abs{\frac{\sin{x}}{x}} = \frac{\abs{\sin{x}}}{x} & \geq \frac{\sin^{2}{x}}{x} \\
                                                    & = \frac{\cos{2x} - 1}{2x}  
\end{align*}
assuming $x \geq 1$.
Now, the Integral Test shows that $\sum_{n = 1}^{\infty} \frac{\cos{2n}}{2n}$ converges (e.g., $\int_{1}^{\infty} \frac{\cos{x}}{x} dx$ converges for the same reason that $\int_{1}^{\infty} \frac{\sin{x}}{x}$ converges), and it is already known that $\sum_{n = 1}^{\infty} \frac{1}{2n}$ does not converge.  
Hence $\sum_{n = 1}^{\infty} \frac{\cos{2n} - 1}{2n}$ diverges.  
By the Comparison Test, $\sum_{n = 1}^{\infty} \abs{\frac{\sin{x}}{x}}$ diverges as well, and it follows that that $\int_{1}^{\infty} \abs{\frac{\sin{x}}{x}} dx$ diverges as well. 

By the previous analysis, it is immediate that for $c \in (0, 1]$, $\int_{1}^{1 / c} \frac{\sin{x}}{x} dx$ converges while $\int_{1}^{1 / c} \abs{\frac{\sin{x}}{x}} dx$ diverges as $c \rightarrow 0$
(for any $c \in (0, 1]$, both integrals exist because $\frac{\sin{x}}{x}$ and $\abs{\frac{\sin{x}}{x}}$ are continuous on $[c, 1]$).
Since $\phi(x) \defeq \frac{1}{x}$ is a strictly decreasing function that maps $[c, 1]$ to $[1, 1/c]$, it is possible to perform change of variables, yielding 
$$\int_{c}^{1} x \sin{\frac{1}{x}} dx = \int_{1}^{1/c} \frac{\sin{x}}{x} dx \hspace{5mm} \text{ and } \hspace{5mm} \int_{c}^{1} x \abs{\sin{\frac{1}{x}}} dx = \int_{1}^{1/c} \abs{\frac{\sin{x}}{x}} dx $$
Letting $f(x) = x \sin{\frac{1}{x}}$, one sees $\int_{c}^{1} f(x) dx$ converges while $\int_{c}^{1} \abs{f(x)} dx$ diverges as $c \rightarrow 0$, as was to be demonstrated.  

\section*{Question 10}
\setcounter{subsection}{0}
\subsection{} 
Here, I use some facts from previous calculus classes without proof;
I hope they will be excused...

First, notice that since $1/p + 1/q = 1$ and $p, q \geq 0$, 
$$p = \frac{q}{q - 1}, \: q = \frac{p}{p - 1}, \text{ which implies } p, q > 1$$
Now, take any $v \geq 0$ and define 
$$f(u) = \frac{u^p}{p} - \frac{v^q}{q} - uv$$
Regardless of the specific value of $p$, $u^p$ is continuous on $[0, \infty)$, hence $f(u)$ is continuous on $[0, \infty)$. 
Also, because $p > 1$, $u^p$ grows faster than $-uv$, hence $f(u) \rightarrow \infty$ as $u \rightarrow \infty$. 
Next, 
$$f'(u) = u^{p - 1} - v$$ 
Note $f'(0) = -v < 0$. 
Because $f(u) \rightarrow \infty$ and $f'$ is continuous on $[0, \infty)$, there must be a point $u_0 \in [0, \infty)$ such that $f'(u_0) = 0$. 
But because $p > 1$, $f'$ is monotonic increasing, hence $u_0$ is a relative minimum of $f'$. 
Furthermore, $u_0$ is the unique value satisfying $f'(u_0) = 0$: 
\begin{equation} \label{eq:eq1}
  f'(u) = 0 \iff u^{p - 1} - v = 0 \iff u^{p - 1} = v
\end{equation}
Since $u > 0$, only a single value for $u$ satisfies $u^{p - 1} - v = 0$, namely $u = u_0$.
The final equation of (\ref{eq:eq1}), gives two identities satisfied by $u_0$:  
$$u_{0}^{(p - 1)q} = v^q \iff u_{0}^{p} = v^q$$
$$u_{0}^{p - 1} u_0 = u_0 v \iff u_{0}^p = u_0 v$$
Namely, the if and only if signifies that $u_0$ is the unique value that satisfies these identities.  

Recapitulating, the above shows that $u_0$ is the only relative extreme of $f$. 
Particularly, $f$ obtains a relative minimum on $u_0$. 
Using the relations satisfied by $u_0$, 
\begin{align*}
  f(u_0) & = \frac{u_{0}^{p}}{p} + \frac{v^q}{q} - u_{0} v \\  
         & = u_{0}^{p} (\frac{1}{p} + \frac{1}{q} - 1) \\ 
         & = 0
\end{align*}
which means $f(u) \geq 0$. 
This yields the following conclusions, as desired:
\begin{enumerate}
  \item $uv \leq \frac{u^{p}}{p} + \frac{v^q}{q}$ for any $u, v \geq 0$.
  \item Equality holds if and only if $u^p = v^q$.  
\end{enumerate}

\subsection{} 
By the inequality derived in (a), so $fg \leq \frac{f^p}{p} + \frac{g^q}{q}$. 
By Theorem 6.12(b), 
\begin{align*}
  \int_{a}^{b} fg d \alpha & = \int_{a}^{b} (\frac{f^p}{p} + \frac{g^p}{p}) d \alpha \\ 
                           & = \frac{1}{p} \int_{a}^{b} f^p d \alpha + \frac{1}{q} \int_{a}^{b} g^q d \alpha \\ 
                           & = 1/p + 1/q = 1
\end{align*}
where all the integrals are guaranteed to exist due to Theorems 6.12 and 6.13. 

\subsection{} 
Let 
$$F \defeq \left[\int_{a}^{b} \abs{f}^p d \alpha\right]^{1/p}, \: G \defeq \left[\int_{a}^{b} \abs{g}^q d \alpha\right]^{1 / q}$$ 
The desired inequality is equivalent to 
$$\frac{\abs{\int_{a}^{b} fg d \alpha}}{FG} \leq 1$$  
Let $h(x) \defeq \frac{\abs{f(x)}}{F}$ and $q(x) \defeq \frac{\abs{g(x)}}{G}$.
Notice 
\begin{align*} 
  \int_{a}^{b} h^p d \alpha & = \int_{a}^{b} \frac{\abs{f}^p}{F^p} d \alpha \\ 
                            & = \frac{1}{F^p} \int_{a}^{b} \abs{f}^p d \alpha \\ 
                            & = \frac{1}{F^p} \cdot F^p \\ 
                            & = 1
\end{align*}
Similarly, $\int_{a}^{b} q^p d \alpha = 1$. 
By definition, $h, q \geq 0$; 
$\abs{h} = h$ and $\abs{q} = q$. 
Then, by Theorem 6.25 and the result from part (b), 
\begin{align*} 
  \frac{\abs{\int_{a}^{b} fg d\alpha}}{FG} & \leq \int_{a}^{b} \frac{\abs{f}}{F} \frac{\abs{g}}{G} d\alpha \\  
                              &  = \int_{a}^{b} hq d\alpha \\ 
                              & \leq 1
\end{align*} 
as was to be shown.

\section*{Question 15}
\setcounter{subsection}{0}
\subsection{}
First, I show the equality 
$$\int_{a}^{b} x f(x) f'(x) dx = -1/2$$
Define $H(x) \defeq xf(x)$ and $G(x) = f(x)$. 
Then $h(x) = f(x) + xf'(x)$ and $g(x) = f'(x)$. 
Integrating by parts, 
\begin{align*} 
  \int_{a}^{b} x f(x) f'(x) dx & = \int_{a}^{b} H(x)g(x) dx \\ 
                               & = \left.H(x)G(x)\right|_{a}^{b} - \int_{a}^{b} h(x)G(x) dx \\ 
                               & = \left.xf^{2}(x)\right|_{a}^{b} - \int_{a}^{b} f^{2}(x) dx - \int_{a}^{b} x f(x) f'(x) dx
\end{align*}
Algebraically manipulating the first expression and last expression of the equalities above and substituting known values,  
$$2\int_{a}^{b} x f(x) f'(x) dx = -1$$
which gives the desired equality. 

\subsection{}
\setcounter{equation}{0}
Next, I show the inequality 
$$\int_{a}^{b} \left[f'(x)\right]^2 dx \cdot \int_{a}^{b} x^2 f^{2}(x) dx \geq 1/4$$
Let $p = q = 2$. 
Then $1/p + 1/q = 1$. 
Define the functions $r(x) \defeq xf(x)$ and $t(x) = f'(x)$. 
By Q10(c), 
\begin{equation} \label{eq:eq2}
  \abs{\int_{a}^{b} r(x)t(x) dx} \leq \left[\int_{a}^{b} \abs{r(x)}^{2} dx\right]^{1/2} \left[\int_{a}^{b} \abs{t(x)}^{2} dx\right]^{1/2}
\end{equation}
Because $r(x)$ and $t(x)$ are real-valued functions, (\ref{eq:eq2}) substitutes to  
$$\abs{\int_{a}^{b} xf(x) f'(x)dx} = 1/2 \leq \left[ \int_{a}^{b} \left[f'(x)\right]^2 dx \right]^{1/2} \cdot \left[\int_{a}^{b} x^2 f^{2}(x) dx \right]^{1/2}$$
and squaring both sides gives the desired inequality.

*** one should be able to show the strict equality by checking the criterion for equality outlined in Q10, but that is very tedious. 

\end{document}

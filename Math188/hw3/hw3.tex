\documentclass[12pt]{article}

%% Basic document formatting
\usepackage{amsmath, amsthm, amssymb, amsfonts}
\usepackage{mathtools}
\usepackage{xspace}
\usepackage{thmtools}
\usepackage{graphicx}
\usepackage{tikz}
\usepackage{setspace}
\usepackage{array}
\usepackage{fancyhdr}
\usepackage{titling}
\usepackage[left=0.4in,right=0.4in,top=1in,bottom=1in]{geometry}
\usepackage{float}
\usepackage{tabularx}
\usepackage[utf8]{inputenc}
\usepackage[english]{babel}
\usepackage{framed}
\usepackage[dvipsnames]{xcolor}
\usepackage{environ}
\usepackage{tcolorbox}
\tcbuselibrary{theorems,skins,breakable}

\usepackage{enumitem}
\setlist[enumerate]{leftmargin=*}
\setlist[enumerate,1]{labelindent=\parindent}
\setlist[enumerate,2]{labelindent=0pt}

% Blackboard Bold
\newcommand{\N}{\mathbb{N}}         % natural numbers
\newcommand{\Z}{\mathbb{Z}}         % integers
\newcommand{\Zpl}{\mathbb{Z}_{+}}   % positive integers
\newcommand{\Q}{\mathbb{Q}}         % rationals
\newcommand{\Qpl}{\mathbb{Q}_{+}}   % positive Rationals
\newcommand{\R}{\mathbb{R}}         % reals
\newcommand{\Rpl}{\mathbb{R}_{+}}   % positive Reals
\newcommand{\C}{\mathbb{C}}         % complex numbers
\newcommand{\F}{\mathbb{F}}         % field

% Words
\newcommand{\st}{\text{ such that }}
\newcommand{\wrt}{\text{ with respect to }}
\newcommand{\with}{\text{ with }}
\newcommand{\ie}{\text{, i.e. }}

% Operators
\newcommand{\abs}[1]{\left|#1\right|}                   % absolute value
\newcommand{\floor}[1]{\left\lfloor #1 \right\rfloor}   % floor
\newcommand{\ceil}[1]{\left\lceil #1 \right\rceil}      % ceiling

% Algebra 
\newcommand{\Syl}{\text{Syl}}           % set of Sylow-p subgroups 
\newcommand{\<}{\langle}                % \<x\>, subgroup generated by x 
\renewcommand{\>}{\rangle}
\newcommand{\id}{\text{id}}             % identity element
\newcommand{\order}[1]{\text{o}(#1)}    % order of an element
\let\oldcong\cong
\let\oldequiv\equiv
\renewcommand{\cong}{\oldequiv}
\renewcommand{\equiv}{\oldcong}

\makeatletter % cycle
\newcommand{\cyc}[1]{(\mathbf{\cyc@process#1\relax})}
\def\cyc@process#1#2\relax{%
	#1%
	\ifx\relax#2\relax
	\else
		\,\cyc@process#2\relax
	\fi
}
\makeatother

% Linear Algebra
\newcommand{\GL}{\text{GL}}                             % general linear group
\newcommand{\SL}{\text{SL}}                             % special linear group
\newcommand{\bmat}[1]{\begin{bmatrix}#1\end{bmatrix}}   % bracketed matrix
\newcommand{\rank}{\operatorname{rank}}                 % rank
\newcommand{\nullity}{\operatorname{nullity}}           % nullity

% Topology/Analysis
\newcommand{\ball}[2]{\text{B}_{#1}(#2)}  % B_r(x): open r-balls around x
\newcommand{\diam}{\text{diam}}           % diamter of a set in metric space 

% Misc Notation
\newcommand{\defeq}{\vcentcolon=}   % :=
\newcommand{\eqdef}{=\vcentcolon}   % =: 
\renewcommand{\bf}[1]{\textbf{#1}}


\renewcommand{\thesection}{\arabic{section}.}
\renewcommand{\thesubsection}{(\alph{subsection})}

\newtheorem{claim}{Claim}
\newtheorem*{lemma}{Lemma}

%% Headers & title setup
\newcommand{\course}{Math188}
\newcommand{\myname}{Jay Ser}
\setlength{\headheight}{14.5pt}
\pagestyle{fancy}
\fancyhf{}
\renewcommand{\headrulewidth}{0.4pt}
\lhead{\course}
\rhead{\myname}
\cfoot{\thepage}
\setlength{\droptitle}{-4em} 
\title{\course\ - HW \#3}
\author{\myname}
\date{2026.01.26}

\begin{document}
\maketitle
\thispagestyle{fancy}

%------------------------------------------------------------------------------%
\section{} 
The equality to demonstrate is 
$$A_n = \sum_{k = 0}^{n - 1} (-1)^k \binom{n - 1}{k} B(n - k)$$
Let $C_{n} (k) \defeq \{\text{set partition of } [n] \mid k, k + 1 \text{ are in the same block}\}$ for $k \in [n - 1]$.
Then given any $I \subset [n - 1]$, 
\begin{align*} %TODO: explain combinatorially why the second equality holds
  \#\bigcap_{k \in I} C_{n} (k) & = \# \{\text{set partition of } [n] \mid k, k + 1 \text{ in the same block for all } k \in I\} \\  
                                & = B(n - \#I)
\end{align*}
where the final equality is interpreted as follows: 
each $k \in I$ ``removes" a single distinct equivalence class of $[n]$; 
if $k$ is already blocked with $k - 1$ because $k - 1 \in I$, then $k + 1$ blocks with $k$ and $k - 1$; 
if $k \in I$ is not blocked with $k - 1$, then the distinct equivalence class of $k + 1$ is absorbed into equivalence class of $k$.
So after blocking each $k \in I$, then there are $n - \#I$ equivalence classes that can be freely partitioned;
there are $B(n - \#I)$ ways to partition these equivalence classes.  

Then, using the principle of inclusion-exclusion, 
\begin{align*}
  \# A_n & = B(n) - \# \bigcup_{k = 1}^{n - 1} C_{n} (k) \\ 
         & = B(n) - \sum_{I \neq \emptyset \subset [n - 1]} (-1)^{\#I + 1} \# \bigcap_{k \in I}C_{n} (k) \\ 
         & = B(n) + \sum_{k = 1}^{n - 1} (-1)^k \binom{n - 1}{k} B(n - k) \\ 
         & = \sum_{k = 0}^{n - 1} (-1)^k \binom{n - 1}{k} B(n - k)
\end{align*} 
where the final equality follows by considering the lone $B(n)$ term as the $k = 0$ summand. 

\section{} 

\section{} 

\section{} 
Consider $n \times m$ matrices. 
Without loss of generality, assume $n \geq m$, i.e., more rows than cols 
(if $m > n$, then consider  the transpose of the matrix). 
$$A_k \defeq \{\text{matrices with no column of 0's and row } k \text{ is 0}\}$$
Then by the principle of inclusion and exclusion,  
\begin{align*}
  M(n, m) & = (2^n - 1)^m - \# \bigcup_{k = 1}^{n} A_k \\ 
          & = (2^n - 1)^m - \sum_{I \neq \emptyset \subset [n]} (-1)^{\#I + 1} \# (\bigcap_{k \in I} A_k)
\end{align*}
where the first term is the number of matrices without any column of 0's:
in each column, there are $2^n$ different configurations and only a single configuration, the row of column of 0's, must be avoided. 
By a similar counting argument, 
$$\#(\bigcap_{k \in I} A_k) = \#\{\text{matrices with no column of 0's, } k\text{th row is 0 for each } k \in I\} = (2^{n - \#I} - 1)^m$$
where the $2^{n - \#I}$ accounts for  the fact that $\#I$ rows are fixed. 
Thus 
\begin{align*}
  M(n, m) & = (2^n - 1)^m + \sum_{k = 1}^{n} (-1)^k \binom{n}{k} (2^{n - k} - 1)^m \\ 
          & = \sum_{k = 0}^{n} (-1)^k \binom{n}{k} (2^{n - k} - 1)^m  
\end{align*}

\section{} 

\end{document}

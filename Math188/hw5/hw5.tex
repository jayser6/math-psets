\documentclass[12pt]{article}

%% Basic document formatting
\usepackage{amsmath, amsthm, amssymb, amsfonts}
\usepackage{mathtools}
\usepackage{xspace}
\usepackage{thmtools}
\usepackage{graphicx}
\usepackage{tikz}
\usepackage{ytableau}
\usepackage{blkarray}
\usepackage{hyperref}
\usepackage{setspace}
\usepackage{array}
\usepackage{fancyhdr}
\usepackage{titling}
\usepackage[left=0.4in,right=0.4in,top=1in,bottom=1in]{geometry}
\usepackage{float}
\usepackage{cancel}
\usepackage{tabularx}
\usepackage[utf8]{inputenc}
\usepackage[english]{babel}
\usepackage{framed}
\usepackage[dvipsnames]{xcolor}
\usepackage{environ}
\usepackage{tcolorbox}
\tcbuselibrary{theorems,skins,breakable}

\usepackage{enumitem}
\setlist[enumerate]{leftmargin=*}
\setlist[enumerate,1]{labelindent=\parindent}
\setlist[enumerate,2]{labelindent=0pt}

% Blackboard Bold
\newcommand{\N}{\mathbb{N}}         % natural numbers
\newcommand{\Z}{\mathbb{Z}}         % integers
\newcommand{\Zpl}{\mathbb{Z}_{+}}   % positive integers
\newcommand{\Q}{\mathbb{Q}}         % rationals
\newcommand{\Qpl}{\mathbb{Q}_{+}}   % positive Rationals
\newcommand{\R}{\mathbb{R}}         % reals
\newcommand{\Rpl}{\mathbb{R}_{+}}   % positive Reals
\newcommand{\C}{\mathbb{C}}         % complex numbers
\newcommand{\F}{\mathbb{F}}         % field

% Words
\newcommand{\st}{\text{ such that }}
\newcommand{\wrt}{\text{ with respect to }}
\newcommand{\with}{\text{ with }}
\newcommand{\ie}{\text{, i.e. }}

% Operators
\newcommand{\abs}[1]{\left|#1\right|}                   % absolute value
\newcommand{\floor}[1]{\left\lfloor #1 \right\rfloor}   % floor
\newcommand{\ceil}[1]{\left\lceil #1 \right\rceil}      % ceiling

% Algebra 
\newcommand{\Syl}{\text{Syl}}           % set of Sylow-p subgroups 
\newcommand{\<}{\langle}                % \<x\>, subgroup generated by x 
\renewcommand{\>}{\rangle}
\newcommand{\id}{\text{id}}             % identity element
\newcommand{\order}[1]{\text{o}(#1)}    % order of an element
\let\oldcong\cong
\let\oldequiv\equiv
\renewcommand{\cong}{\oldequiv}
\renewcommand{\equiv}{\oldcong}

\makeatletter % cycle
\newcommand{\cyc}[1]{(\mathbf{\cyc@process#1\relax})}
\def\cyc@process#1#2\relax{%
	#1%
	\ifx\relax#2\relax
	\else
		\,\cyc@process#2\relax
	\fi
}
\makeatother

% Linear Algebra
\newcommand{\GL}{\text{GL}}                             % general linear group
\newcommand{\SL}{\text{SL}}                             % special linear group
\newcommand{\bmat}[1]{\begin{bmatrix}#1\end{bmatrix}}   % bracketed matrix
\newcommand{\rank}{\operatorname{rank}}                 % rank
\newcommand{\nullity}{\operatorname{nullity}}           % nullity

% Topology/Analysis
\newcommand{\ball}[2]{\text{B}_{#1}(#2)}  % B_r(x): open r-balls around x
\newcommand{\diam}{\text{diam}}           % diamter of a set in metric space 

% Misc Notation
\newcommand{\defeq}{\vcentcolon=}   % :=
\newcommand{\eqdef}{=\vcentcolon}   % =: 
\renewcommand{\bf}[1]{\textbf{#1}}


\renewcommand{\thesection}{\arabic{section}.}
\renewcommand{\thesubsection}{(\alph{subsection})}

\newtheorem{claim}{Claim}
\newtheorem*{lemma}{Lemma}

%% Headers & title setup
\newcommand{\course}{Math188}
\newcommand{\myname}{Jay Ser}
\setlength{\headheight}{14.5pt}
\pagestyle{fancy}
\fancyhf{}
\renewcommand{\headrulewidth}{0.4pt}
\lhead{\course}
\rhead{\myname}
\cfoot{\thepage}
\setlength{\droptitle}{-4em} 
\title{\course\ - HW \#5}
\author{\myname}
\date{2026.02.17}

\begin{document}
\maketitle
\thispagestyle{fancy}

%------------------------------------------------------------------------------%

\section{} 
\newcommand{\mdeg}[1]{\text{mdeg}(#1)}
Suppose $f(x) = \sum a_n x^n$, $f(x) \neq 0$, has a square root, say $g(x) = \sum b_n x^n$. 
By multiplication in $\C[[x]]$, 
\begin{equation} \label{eq:a_coef}
  a_n = \sum_{i = 0}^{n}  b_i b_{n - i} 
\end{equation} 
Assume $a_0 = a_1 = \ldots = a_{2n} = 0$.
To prove that $\mdeg{f}$ is even, it suffices to show that $a_{2n + 1} = 0$.  
By \eqref{eq:a_coef} and the fact that $2n + 1$ is odd, 
\begin{equation} \label{eq:odd_coef}
  a_{2n + 1} = 2\sum_{i = 0}^{n} b_i b_{2n + 1 - i}
\end{equation} 
One can induct on $i$ to show that each $b_i = 0$ for $0 \leq i \leq n$. 
When $i = 0$, $a_0 = b_0 b_0 = 0 \implies b_0 = 0$. 
Suppose for $0 \leq i < n$ that $b_0 = b_1 = \ldots = b_i = 0$.
Since $2i + 2 \leq 2(n - 1) + 2 = 2n$, 
$$a_{2i + 2} = 0 = 2\sum_{j = 0}^{i} b_j b_{2i + 2 - j} + b_{i + 1}^2$$
Because $b_j = 0$ for each summand, $a_{2i + 2} = 0 = b_{i + 1}^2 = 0 \implies b_{i + 1} = 0$, as was to be shown.
Since $b_0 = b_1 = \ldots = b_n = 0$, it is clear that \eqref{eq:odd_coef} is equal to 0. 

Conversely, let $f(x) = \sum a_n x^n$, $f(x) \neq 0$ with $\mdeg{f} = 2k$.
Let $g(x) = \sum b_n x^n$ with $b_n$ defined as such: 
\begin{itemize}
  \item $b_0 = b_1 = b_{k - 1} \defeq 0$. 
  \item$b_k \defeq \sqrt{a_{2k}}$, where $b_k \neq 0$ since $a_{2k} \neq 0$ by definition of $f$. 
  \item Suppose $b_0, b_1, \ldots, b_n$ is defined for $n \geq k$. 
    Then let $b_{n + 1} \defeq \sqrt{a_{2(n + 1)} - 2\sum_{i = 0}^{n} b_i b_{2(n + 1) - i}}$
\end{itemize}
Then one can check that $g(x)^2 = f(x)$ (e.g., via calculation demonstrated in the previous paragraph). 

\section{} 
\subsection{} 
\newcommand{\kder}[1]{\frac{\text{d}^#1}{\text{d}x^#1}}
\begin{align*}
  \sum_{n \geq 0} a_n x^n & = \sum_{n \geq 0} n^2 x^n \\ 
                          & = \sum_{n \geq 0} \left[x^2 \kder{2} (x^n)  + nx^n\right] \\ 
                          & = x^2 \kder{2} (\sum_{n \geq 0} x^n) + \sum_{n \geq 0} nx^n \\ 
                          & = \frac{2x^2}{(1 - x)^3} + \frac{x}{(1 - x)^2} \\ 
                          & = \frac{x^2 + x}{(1 - x)^3}
\end{align*} 

\subsection{} 
\begin{align*}
  \sum_{n \geq 0} a_n x^n & = \sum_{n \geq 0} \binom{n}{k} x^n \\ 
                  & = \sum_{n \geq k} \frac{n!}{k!(n - k)!} x^n \\ 
                  & = \frac{1}{k!} \sum_{n \geq k} n(n - 1) \cdots (n - k + 1) x^n \\ 
                  & = \frac{1}{k!} \sum_{n \geq k} x^k \kder{k} (x^n) \\ 
                  & = \frac{x^k}{k!} \kder{k} (\sum_{n \geq k} x^n) \\ 
                  & = \frac{x^k}{k!} \frac{k!}{(1 - x)^k} \\ 
                  & = \frac{x^k}{(1 - x)^k}
\end{align*}

\subsection{} 
Let $A(x) \defeq \sum_{n \geq 0} a_n x^n$. 
\begin{align*}
  A(x) - a_0 & = \sum_{n \geq 0} a_{n + 1} x^{n + 1} \\ 
             & = \sum_{n \geq 0} (\frac{a_n}{3} + 1)x^{n + 1} \\ 
             & = \frac{x}{3} A(x) + \frac{x}{1 - x} \\ 
\end{align*}
which gives 
\begin{align*} 
\MoveEqLeft  A(x) (1 - \frac{x}{3}) = \frac{x}{1 - x} + \frac{1 - x}{1 - x} \\ 
  \implies A(x) & = \frac{1}{1 - x} \cdot \frac{3}{3 - x} \\ 
                & = \frac{3}{2(1 - x)} - \frac{3}{2(3 - x)}
\end{align*} 

\section{} %TODO: is the RHS supposed to start at k = 0 or k = 1? 
\setcounter{equation}{0}
The equality to demonstrate is 
\begin{equation} \label{eq:durfee}
  \prod_{i \geq 1} \frac{1}{1 - q^i} = \sum_{k \geq 0} \frac{q^{k^2}}{\prod_{i = 1}^{k} (1 - q^{i})^2}
\end{equation}
The left hand side of \eqref{eq:durfee} is the generating function 
$$\sum_{\text{partitions } \lambda} q^{\abs{\lambda}}$$
The only partition that doesn't have a Durfee square (i.e., a Durfee square of size 0) is the empty partition. 
Any nonempty partition $\lambda$ has a Durfee square of size at least 1;
suppose the Durfee square of $\lambda$ is of size $k$. 
For example, if $k = 4$, then $\lambda$ looks something like this:
\begin{center}
  \ydiagram[*(gray!40)]{4,4,4,4}
  *[*(red!30)]{4+3, 4+2, 4+1, 4+1}
  *[*(blue!30)]{0,0,0,0,4,4,3} 
\end{center} 
Notice that for any partition with a Durfee square of size $k$, the pink blocks form a Young diagram with at most $k$ parts, while the blue blocks form a Young diamgram with the first part of size at most $k$.
Conjugating the pink partition, the pink young diagram is a partition whose first part has size at most $k$.
Finally, observe that the Durfee square contributes $k^2$ blocks to $\lambda$. 
This gives 
\begin{align*}
  \sum_{\text{partitions } \lambda} q^{\abs{\lambda}} & = \sum_{k \geq 0} \sum_{\substack{\text{partitions } \lambda, \\ \text{ds}(\lambda) = k}} q^{\abs{\lambda}}\\ 
                                                      & = \sum_{k \geq 0} q^{k^2} (1 + q + q^2 + \ldots)^2 (1 + q^2 + q^4 + \ldots)^2 \cdots (1 + q^k + q^{2k} + \ldots)^2 \\ 
                                                      & = \sum_{k \geq 0} \frac{q^{k^2}}{\prod_{i = 1}^{k} (1 - q^{i})^2} 
\end{align*} 
where $\text{ds}(\lambda)$ denotes the size of the Durfee squaure of $\lambda$.
This proves the desired identity. 

\section{}  
\setcounter{equation}{0}
The equality to demonstrate is 
\begin{equation} \label{eq:durfee2}
  \prod_{i \geq 0} (1 + q^{2i + 1}) = \sum_{k \geq 0} \frac{q^{k^2}}{\prod_{i = 1}^{k} (1 - q^{2i})}
\end{equation}
Notice that the left hand side of \eqref{eq:durfee2} is the generating function
$$\sum_{\substack{\text{partitions } \lambda \\ \text{with distinct odd parts}}} q^{\abs{\lambda}} = (1 + q)(1 + q^3)(1 + q^5) \cdots $$
Recall the bijection $\{\lambda \vdash n \mid \lambda = \lambda'\} \longleftrightarrow \{\lambda \vdash n \mid \lambda \text{ has distinct odd parts} \}$.
So in general, there is a bijection 
$$\{\text{partition } \lambda \mid \lambda = \lambda'\} \longleftrightarrow \{\text{partition } \lambda \mid \lambda \text{ has distinct odd parts} \}$$
where both the forward map and the reverse map preserve the size of the partition. 
Hence the left hand size of \eqref{eq:durfee2} is 
$$\sum_{\substack{\text{partitions } \lambda, \\ \lambda = \lambda'}} q^{\abs{\lambda}}$$
Notice that any partition $\lambda \with \lambda = \lambda'$ has a Young diagram that looks like this: 
\begin{center}
  \ydiagram[*(gray!40)]{4,4,4,4}
  *[*(red!30)]{4+3, 4+2, 4+1, 4+1}
  *[*(blue!30)]{0,0,0,0,4,2,1} 
\end{center} 
To verbally describe it, if $\lambda$ has a Durfee square of size $k$, the pink blocks form a Young diagram with at most $k$ parts, the blue blocks form a Young diagram whose first part has size at most $k$, and the pink Young diagram and blue Young diagram are conjugates of each other.  
Thus $\abs{\lambda} = k^2 + 2\abs{\lambda_{\text{pink}}}$.
This gives 
\begin{align*}
  \sum_{\substack{\text{partitions } \lambda, \\ \lambda = \lambda'}} q^{\abs{\lambda}} 
  & = \sum_{k \geq 0} q^{k^2} (1 + q^2 + q^4 + \ldots) (1 + q^4 + q^8 + \ldots) \cdots (1 + q^{2k} + q^{4k} + \ldots)\\ 
  & = \sum_{k \geq 0} \frac{q^{k^2}}{\prod_{i = 1}^{k}(1 - q^{2i})}
\end{align*}
as was to be shown. 

\section{}
\setcounter{equation}{0}
The whatever derivative rule gives  
$$\kder{k} (1 + x)^a = a(a - 1) \cdots (a - k + 1) (1 + x)^{a - k}$$ 
Thus the Taylor expansion of $f(x) = (1 + x)^a$ is 
\begin{align*}
  (1 + x)^a & = \sum_{k \geq 0} \frac{f^{(k)}(0)}{k!} x^k \\ 
            & = \sum_{k \geq 0} \frac{a(a - 1) \cdots (a - k + 1)}{k!} x^k \\ 
            & = \sum_{k \geq 0} \binom{a}{k} x^k 
\end{align*} 

Next, I show that $(1 + x)^a (1 + x)^b = (1 + x)^{a + b}$
as power series.
Rewriting the left-hand side as power series,
\begin{equation} \label{eq:big}
(\sum_{k \geq 0}\binom{a}{k} x^k) \cdot (\sum_{k \geq 0} \binom{b}{k} x^k) = \sum_{k \geq 0} c_k x^k
\end{equation}
where 
$$c_k = \sum_{i + j = k} \binom{a}{i} \binom{b}{j}$$
View $c_k$ as a polynomial in $a$.
If $a$ is any integer, then 
$$c_k \defeq \sum_{i + j = k} \binom{a}{i} \binom{b}{j} = \binom{a + b}{k}$$
by the following combinatorial proof:  
The right hand side counts the number of ways to choose $k$ elements from $[a + b]$. 
The total number of ways to do so is the sum of the number of ways to choose $0 \leq i \leq k$ elements from $[a]$, then choose the remaining $k - i$ elements from $[b]$, which is the left hand side. 

Uh technically the above proof only holds if $b$ is also an integer, and for two polynomials in $\C[a, b]$, showing that they agree on infinitely many pairs $(a_0, b_0)$ doens't prove that the two polynomials are identical, but oh well, I tried my best. 
Assuming $c_k = \binom{a + b}{k}$ identically, \eqref{eq:big} is equal to 
$$(x + 1)^{a + b}$$
as was to be shown. 


\section{} 
Let 
$$b_n \defeq \begin{cases}
  (n - 1)!, & \text{if } n \in \{1, 2, 3, 6\} \\ 
  0, & \text{else}
\end{cases} $$
In other words, if $n \in \{1, 2, 3, 6\}$, $b_n$ is the number of ways to form an $n$-cycle in $[n]$; 
if $n$ is not a divisor of 6, then forcibly set the number of ways to structure $[n]$ to be zero. 

Now, since $a_n$ counts the number of $w \in \mathfrak{S}_n \st w^6 = 1$, which is equivalently the number of $w \in \mathfrak{S}_n$ whose cycle decomposition only consists of 1, 2, 3, and 6-cycles, $a_n$ can be interpreted as the number of ways to: 
\begin{enumerate}
  \item Partition $[n]$ into $B_1, B_2, \ldots, B_k$ such that the size of each $B_i \in \{1, 2, 3, 6\}$. 
  \item For each $B_i$, put the elements of $B_i$ into a single cycle. 
\end{enumerate}
As such, the following exponentiation makes sense:
\begin{align*}
  A(x) \defeq \sum_{n \geq 0} \frac{a_n}{n!} x^n & = \exp(\sum_{m \geq 0} \frac{b_m}{m!} x^m) \\ 
                                                 & = \exp(x + \frac{x^2}{2} + \frac{x^3}{3} + \frac{x^6}{6})
\end{align*}

\section{}
$c_n$ can be interpreted as the number of ways to: 
\begin{enumerate}
  \item Partition $[n]$ into $B_1, \ldots, B_k$.  
  \item \label{item:*} For each $B_i$, single out a single element, then form a single cycle out of the rest of the elements.  
\end{enumerate}
Namely, if $\#B_i = m > 1$, then the number of ways to achieve \eqref{item:*} is $m (m - 2)!$.
If $\#B_i = 1$, then of course the number of ways to achieve \eqref{item:*} is just 1. 
Thus 
\begin{align*}
  C(x) \defeq \sum_{n \geq 0} \frac{c_n}{n!} x^n & = \exp(x + \sum_{m > 1} \frac{m(m - 2)!}{m!} x^m) \\ 
                                                 & = \exp(x + \sum_{m \geq 1} \frac{x^{m + 1}}{m}) \\ 
                                                 & = \exp(x - x\log(1 - x)) \\ 
                                                 & = e^x e^{-x\log(1 - x)} \\ 
                                                 & = e^x \left(\frac{1}{1 - x}\right)^{-x}
\end{align*}

\section{}
\textbf{*** I accidently made $c_n = \#\text{ of decorated permutations in } \mathfrak{S}_n$ instead of $b_n$!!!}

Define
\begin{align*}
  a_n & \defeq \#\text{ of ways to form a single cycle from } [n] \\ 
      & = (n - 1)!, \\ 
  b_n  & \defeq \# \text{ of ways to pick a single element from } [n] \\
      & = n
\end{align*}
and let 
\begin{align*}
  A(x) & \defeq \sum_{n \geq 1} a_n \frac{x^n}{n!} \\ 
  & = \sum_{n \geq 1} \frac{x^n}{n} \\ 
  & = -\log(1 - x)
\end{align*}
and 
\begin{align*}
  B(x) & \defeq \sum_{n \geq 0} b_n \frac{x^n}{n!} \\
  & = \sum_{n \geq 0} n \frac{x^n}{n!} \\ 
  & = x e^x 
\end{align*}
Now, note that $c_n$ counts the number of ways to:
\begin{enumerate}
  \item partition $[n]$ into $B_1, \ldots, B_k$. 
  \item \label{item:a_n} For each $B_i$, form a single cycle from the elements of $B_i$. 
  \item \label{item:b_n} Pick one block from $B_1, \ldots, B_k$ (to decorate it)
\end{enumerate}
Particularly, for each $B_i$, if $\#B_i = m$, \eqref{item:a_n} is given by $a_m$. 
If $[n]$ is partitioned into $k$ blocks $B_1, \ldots, B_k$, then \eqref{item:b_n} is given by $b_k$. 
This means 
\begin{align*}
  C(x) \defeq \sum_{n \geq 0} c_n \frac{x^n}{n!} & = B(A(x)) \\ 
                                                 & = (-\log(1 - x))e^{-\log(1 - x)} \\ 
                                                 & = \frac{-\log(1 - x)}{1 - x}
\end{align*} 

\end{document}

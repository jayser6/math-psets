\documentclass[12pt]{article}

%% Basic document formatting
\usepackage{amsmath, amsthm, amssymb, amsfonts}
\usepackage{mathtools}
\usepackage{xspace}
\usepackage{thmtools}
\usepackage{graphicx}
\usepackage{tikz}
\usepackage{setspace}
\usepackage{array}
\usepackage{fancyhdr}
\usepackage{titling}
\usepackage[left=0.4in,right=0.4in,top=1in,bottom=1in]{geometry}
\usepackage{float}
\usepackage{tabularx}
\usepackage[utf8]{inputenc}
\usepackage[english]{babel}
\usepackage{framed}
\usepackage[dvipsnames]{xcolor}
\usepackage{environ}
\usepackage{tcolorbox}
\tcbuselibrary{theorems,skins,breakable}

\usepackage{enumitem}
\setlist[enumerate]{leftmargin=*}
\setlist[enumerate,1]{labelindent=\parindent}
\setlist[enumerate,2]{labelindent=0pt}

% Blackboard Bold
\newcommand{\N}{\mathbb{N}}         % natural numbers
\newcommand{\Z}{\mathbb{Z}}         % integers
\newcommand{\Zpl}{\mathbb{Z}_{+}}   % positive integers
\newcommand{\Q}{\mathbb{Q}}         % rationals
\newcommand{\Qpl}{\mathbb{Q}_{+}}   % positive Rationals
\newcommand{\R}{\mathbb{R}}         % reals
\newcommand{\Rpl}{\mathbb{R}_{+}}   % positive Reals
\newcommand{\C}{\mathbb{C}}         % complex numbers
\newcommand{\F}{\mathbb{F}}         % field

% Words
\newcommand{\st}{\text{ such that }}
\newcommand{\wrt}{\text{ with respect to }}
\newcommand{\with}{\text{ with }}
\newcommand{\ie}{\text{, i.e. }}

% Operators
\newcommand{\abs}[1]{\left|#1\right|}                   % absolute value
\newcommand{\floor}[1]{\left\lfloor #1 \right\rfloor}   % floor
\newcommand{\ceil}[1]{\left\lceil #1 \right\rceil}      % ceiling

% Algebra 
\newcommand{\Syl}{\text{Syl}}           % set of Sylow-p subgroups 
\newcommand{\<}{\langle}                % \<x\>, subgroup generated by x 
\renewcommand{\>}{\rangle}
\newcommand{\id}{\text{id}}             % identity element
\newcommand{\order}[1]{\text{o}(#1)}    % order of an element
\let\oldcong\cong
\let\oldequiv\equiv
\renewcommand{\cong}{\oldequiv}
\renewcommand{\equiv}{\oldcong}

\makeatletter % cycle
\newcommand{\cyc}[1]{(\mathbf{\cyc@process#1\relax})}
\def\cyc@process#1#2\relax{%
	#1%
	\ifx\relax#2\relax
	\else
		\,\cyc@process#2\relax
	\fi
}
\makeatother

% Linear Algebra
\newcommand{\GL}{\text{GL}}                             % general linear group
\newcommand{\SL}{\text{SL}}                             % special linear group
\newcommand{\bmat}[1]{\begin{bmatrix}#1\end{bmatrix}}   % bracketed matrix
\newcommand{\rank}{\operatorname{rank}}                 % rank
\newcommand{\nullity}{\operatorname{nullity}}           % nullity

% Topology/Analysis
\newcommand{\ball}[2]{\text{B}_{#1}(#2)}  % B_r(x): open r-balls around x
\newcommand{\diam}{\text{diam}}           % diamter of a set in metric space 

% Misc Notation
\newcommand{\defeq}{\vcentcolon=}   % :=
\newcommand{\eqdef}{=\vcentcolon}   % =: 
\renewcommand{\bf}[1]{\textbf{#1}}


\renewcommand{\thesection}{\arabic{section}.}
\renewcommand{\thesubsection}{(\alph{subsection})}

\newtheorem{claim}{Claim}
\newtheorem*{lemma}{Lemma}

%% Headers & title setup
\newcommand{\course}{Math188}
\newcommand{\myname}{Jay Ser}
\setlength{\headheight}{14.5pt}
\pagestyle{fancy}
\fancyhf{}
\renewcommand{\headrulewidth}{0.4pt}
\lhead{\course}
\rhead{\myname}
\cfoot{\thepage}
\setlength{\droptitle}{-4em} 
\title{\course\ - HW \#1}
\author{\myname}
\date{2026.01.12}

\begin{document}
\maketitle
\thispagestyle{fancy}

%------------------------------------------------------------------------------%

\section{} 
\begin{lemma}
  If a set of dominos tiles a $2 \times n$ grid, then the size of this set of dominos is exactly $n$. 
\end{lemma}
\begin{proof}[Proof of Lemma]
  The proof follows by strong induction on $n$. 
  The base cases are trivial since $f_0 = f_1 = 1$; 
  The only way to fill a $2 \times 0$ grid and a $2 \times 1$ grid is by using no blocks and by using a single vertical domino, respectively. 

  Suppose only a set of $n - 1$ dominos can tile a $2 \times (n - 1)$ grid and only a set of $n - 2$ dominos can tile a $2 \times (n - 2)$ grid.
  The combinatorial interpretation of the recursion rule $f_n = f_{n - 1} + f_{n - 2}$ implies that the only ways to complete a tiling of a $2 \times n$ grid is to 
  \begin{enumerate}
    \item append a single vertical domino to a tiling of the first $n - 1$ columns, which is composed of $n - 1$ dominos.
    \item append a stack of two horizontal dominos to a tiling of the first $n - 2$ columns, which is composed n - 2 dominos.   
  \end{enumerate}

  In both cases, the total number of dominos used to tile the complete $2 \times n$ grid is $n$, as was to be shown. 
\end{proof}

The lemma gives the easy formula $a_n = 3^n f_n$; 
counting $a_n$ is equivalent to first counting the number of ways to tile the $2 \times n$ grid without distinguishing color, then counting the number of ways to assign each $n$ blocks one of three possible colors. 
This yields a recursive definition for $a_n$ when $n \geq 2$:
$$a_n = 3^n f_n = 3^n (f_{n - 1} + f_{n -2}) = 3(a_{n - 1} + 3a_{n - 2})$$
Manuel computation shows $a_0 = 1$, $a_1 = 3$. 

Finally, one can derive the closed form for the generating function $A(x) \defeq \sum a_n x^n$: 
\begin{align*}
  A(x) & = \sum_{n = 0}^{\infty} a_n x^n \\ 
       & = 1 + 3x + 3\sum_{n = 2}^{\infty}(a_{n - 1} + 3a_{n - 2})x^n \\
       & = 1 + 3x + 3(x\sum_{n = 1}^{\infty} a_n x^n + 3x^2 \sum_{n = 0}^{\infty} a_n x^n) \\ 
       & = 1 + 3x + 3(x(A(x) - a_0) + 3x^2 A(x)) \\ 
       & = 1 + 3xA(x) + 9x^2A(x)
\end{align*}

Factoring the final equality leads to 
$$A(x) = \frac{1}{1 - 3x - 9x^2}$$ 

\section{} 
\subsection{Solution 1 (original, messier solution)}
\begin{center}
  \begin{tikzpicture}[scale=1.5]
    % 1x2 Vertical Domino
    \draw[thick] (0,0) rectangle (1,2);
    \draw[dotted] (0,1) -- (1,1);
    \fill (0.5,0.5) circle (0.05);
    \fill (0.5,1.5) circle (0.05);
    \node[below] at (0.5,-0.3) {Configuration A};

    % Two 2x1 Horizontal Dominos Stacked
    \draw[thick] (3,0) rectangle (5,1);
    \draw[dotted] (4,0) -- (4,1);
    \fill (3.5,0.5) circle (0.05);
    \fill (4.5,0.5) circle (0.05);

    \draw[thick] (3,1) rectangle (5,2);
    \draw[dotted] (4,1) -- (4,2);
    \fill (3.5,1.5) circle (0.05);
    \fill (4.5,1.5) circle (0.05);

    \node[below] at (4,-0.3) {Configuration B};

  \end{tikzpicture}
\end{center}

I prove an equivalent identity for when $n \geq 2$: 
$$f_n = \sum_{i = 0}^{n - 2} f_i + 1$$ 

Using the language of domino tiles, the definition $f_n = f_{n - 1} + f_{n -2 }$ can be understood as the sum of two cases from which the $n$th column can be filled: 
either through the use of a single vertical tile (configuration A) -- counted by $f_{n - 1}$ -- or the use of a two horizontal tiles stacked on top of each other (configuration B) -- counted by $f_{n - 2}$.  

Similarly, $f_{n - 1}$ can be decomposed into the sum of two cases: $f_{n - 2}$, which counts the number of ways to fill the $(n - 1)$th column using configuration A, and $f_{n - 3}$, which counts the number of ways to fill the $(n - 1)$th column using configuration B.

Continuously decomposing the number of ways to fill the $k$th column using configuration A, where $k \geq 2$, into
\begin{center}
number of ways to fill the $(k - 1)$th column using configuration A 
\\+ number of ways to fill the $(k - 1)$th column using configuration B
\end{center}
yields the equality 
$$f_n = f_0 + f_1 + f_1 + f_2 + f_3 + \ldots + f_{n - 2}$$ 
$f_1 = 1$, so the equality can be rewritten as
$$f_n = \sum_{i = 0}^{n - 2} f_n + 1$$
as desired. 

\subsection{Solution 2} 
The equality to demonstrate is 
$$\sum_{i = 0}^{n - 1} = f_{n + 1} - 1$$

The right hand side can be interpreted as the number of ways to tile a $2 \times (n + 1)$ grid, excluding the tiling that only uses the vertical domino, i.e., 
$$f_{n + 1} - 1 = \# \text{ of ways to tile a } 2 \times (n + 1) \text{ grid by using at least one instance of configuration B}$$
which can be decomposed into 
\begin{align*}
  \sum_{i = 0}^{n - 1} (\# \text{ of ways to tile a } 2 \times (n + 1) \text{ grid such that the rightmost confiugration B covers columns } i + 1, i + 2)
\end{align*}

For each $i \in \{0, \ldots, n - 1\}$, notice that by the definition of the decomposition, each of the $i + 3, i + 4, \ldots, n$th columns are filled via configuration A.
Thus, the number of ways to fill the grid such that the rightmost configuration B occurs in columns $i + 1, i + 2$ is the number of ways to fill columns $1, \ldots, i$, which is given by $f_i$.
Thus, the decomposition is equivalent to 
$$\sum_{i = 0}^{n - 1} f_i$$
as was to be shown. 

\section{} 
\subsection{Solution 1 (original, messier solution)}
The equality to demonstrate is 
$$f_{2n + 1} = \sum_{i = 0}^{n} f_{2i}$$

By definition, the left hand side, $f_{2n + 1}$, counts the number of ways to tile a $2 \times (2n + 1)$ grid. 

On the right hand side, consider each summand $f_{2i}$ as the number of ways to tile the first $2i$ columns in a $2 \times (2n + 1)$ grid, where the $(2i + 1)$st column is filled via configuration A and the rest of the columns to the right are all filled via configuration B.  
Notice the union of configurations associated with each $\left.f_{2i}\right|_{i = 0}^{n}$ lists all possible confiugrations to tile a $2 \times (2n + 1)$ grid:
by the definition of how each $f_{2i}$ is interpreted, the collection certainly doesn't double count;
meanwhile, the collection accounts for every possible tiling of a $2 \times (2n + 1)$ grid because it exhausts every way to tile the first $(2n + 1) - k$ columns given every possible way to tile the last $k$ columns for each $k \in \{1, \ldots, 2n + 1\}$. 

\subsection{Solution 2} 
$f_{2n + 1}$ counts the number of ways to tile a $2 \times (2n + 1)$ grid. 
Notice that because configuration B always fills an even number of columns and $2n + 1$ is odd, every tiling of the grid must utilize at least one instance of configuration A. 

Suppose the rightmost instance of configuration A occurs at column $i \in \{1, \ldots, 2n + 1\}$. 
Because all columns to the right of column $i$ are filled by configuration B, it is necessary that $i$ is odd to ensure there is an even number of columns to the right of column $i$. 
Write $i = 2j + 1$ for some $j \in \{0, \ldots, n\}$. 
Then the number of tilings of the $2 \times (2n + 1)$ grid such that the rightmost instance of configuration A is at column $i$ is just the number of ways to fill the first $i - 1 = 2j$ columns, which is given by $f_{2j}$.

Finally, notice that $f_{2n + 1}$ can be decomposed into the sum of the number of ways to tile a $2 \times (2n + 1)$ grid such that the rightmost instance of confiugration A is in column $i$, where $i = 2j + 1$ for $j$ varying over $0, \ldots, n$. 
In all, this combinatorial interpretation yields the desired identity 
$$f_{2n + 1} = \sum_{j = 0}^{n} f_{2j}$$

\section{} 
\subsection{Recursion Formula for the Multinomial Coefficient} 
The analogue of the Pascal recursion for the multinomial coefficient is the following: 
$$\binom{n}{a_1, a_2, \ldots, a_m} = \binom{n - 1}{(a_1 - 1), a_2, a_3, \ldots, a_m}  \binom{n - 1}{a_1, (a_2 - 1), a_3, \ldots, a_m} \cdots  \binom{n - 1}{a_1, a_2, a_3, \ldots, (a_m - 1)}$$  

I give a combinatorial proof of this identity that relies on part (b). 
Interpret $\binom{n}{a_1, a_2, \ldots, a_m}$ as the number of length $n$ words of the alphabet $\{1, \ldots, m\}$ with $a_1$ copies of 1's', \ldots, $a_m$ copies of $m$'s.  
Thus $\binom{n}{a_1, a_2, \ldots, a_m}$ can be decomposed into the sum over each $k \in \{1, \ldots, m\}$ of the number of length $n$ words such that the $n$th letter is fixed as $k$ and the number of copies of the letter $i$ in the word is given by $a_i$.
Particularly, when the $n$th letter is fixed as $k$ for any $k \in \{1, \ldots, m\}$, the number of words satisfying the conditions above is given by 
$$\binom{n - 1}{a_1, a_2, \ldots, a_{k - 1}, (a_k - 1), a_{k + 1}, \ldots, a_m}$$ 
since amongst the unfixed letters $1, \ldots, n - 1$, the number of copies of letter $i$ must be 
$$\begin{cases} 
  a_i, & \text{if } i \neq k\\ 
  a_i - 1, & \text{if } i = k
\end{cases}$$

This gives the desired identity. 

\subsection{Combinatorial Interpretation of the Multinomial Coefficient}
Suppose $a_1 + \ldots + a_m = n$ and each $a_i \geq 0$. 
By the definition of the multinomial coefficient, 
\begin{align*}
  \binom{n}{a_1, a_2, \ldots, a_m} & \defeq \frac{n!}{a_1! a_2! \cdots a_m!} \\
                                   & = \frac{n!}{a_1!(n - a_1)!} \frac{(n - a_1)!}{a_2! (n - a_1 - a_2)!} \frac{(n - a_1 - a_2)!}{a_3! (n - a_1 - a_2 - a_3)!} \cdots \frac{(n - a_1 - a_2 - \ldots - a_{m - 1})!}{a_m! (n - a_1 - a_2 - \ldots - a_m)!} \\ 
                                   & = \binom{n}{a_1} \binom{n - a_1}{a_2} \binom{n - a_1 - a_2}{a_3} \cdots \binom{n - a_1 - a_2 - \ldots - a_{m - 1}}{a_m} 
\end{align*}

Combinatorially interpreted, the final equality's product of binomial coefficients counts the number of ways to label exactly $a_1$ of the $n$ items as ``1", label exactly $a_2$ of the remaining unlabeled $n - a_1$ items as ``2", label exactly $a_3$ of the remaining unlabeled $n - a_1 - a_2$ items as ``3", \ldots, and label exactly $a_m$ of the remaining unlabeled $n - a_1 - a_2 - \ldots - a_{m - 1}$ items as $``m"$.
This is the desired combinatorial interpretation of the multinomial coefficient $\binom{n}{a_1, a_2, \ldots, a_m}$. 

\section{} 
\subsection{$S(n, 1) = S(n, n) = 1$} 
A partition of $[n]$ into one part is simply the collection $\{[n]\}$, so $S(n, 1) = 1$. 
Simimarly, the only way to partition $[n]$ into $n$ parts is to place each $i \in [n]$ into the singleton $\{i\}$, which gives $S(n, n) = 1$. 

\subsection{$S(n, 2) = 2^{n - 1} - 1$}
There are $2^n$ ways to subdivide $[n]$ into sets $X$ and $Y$ because for each element $i \in [n]$, there are two ways to place $i$ (either $X$ or $Y$, but not both), and the placement of each $i$ is independent of the placement of the other elements. 
Of this $2^n$ number of ways to allocate $[n]$ into $X$ and $Y$ are the cases 
\begin{enumerate}
  \item $X = \emptyset$, $Y = [n]$
  \item $X = [n]$, $Y = \emptyset$
\end{enumerate}

These two cases partition $[n]$ into a single part, so the two cases can be discarded. 
Let $S$ be the set of all remaining $2^n - 2$ reallocations of $[n]$ into $X$ and $Y$. 
For each $s \in S$, $\exists! s' \in S$ satisfying the following: 
\begin{enumerate}
  \item $s \neq s'$ 
  \item Suppose $s$ subdivides $[n]$ into $X_s = \{i_1, \ldots, i_k\}$ and $Y_s = \{i_{k + 1}, \ldots, i_{n}\}$, where $0 < k < n$ and $i_1, \ldots, i_n$ is a permutation of $1, \ldots, n$.
    Then $s'$ subdivides $[n]$ into $X_{s'} = \{i_{k + 1}, \ldots, i_n\}$ and $Y_{s'} = \{i_1, \ldots, i_k\}$. 
\end{enumerate}

Then $s$ and $s'$ are the equivalent two-part partitions of $[n]$.
This means exactly half of the elements of $S$ are unique partitions of $[n]$ into two parts.
Conversely, any partition of $[n]$ into two parts, by definition, can be represented as an element of $S$. 
$\abs{S} = 2^n - 2$, so there are $2^{n - 1} - 1$ unique ways to partition $[n]$ into two parts, i.e., 
$$S(n, 2) = 2^{n - 1} - 1$$

\subsection{$S(n, n - 1) = \binom{n}{2}$} 
It is clear that any partition of $[n]$ into $n - 1$ parts consists of $n - 2$ singletons and a subset of size 2. 
Thus, the number of unique two-part partitions of $[n]$ is equal to the number of ways to form pairs in $[n]$, which is $\binom{n}{2}$. 

\section{} 
Denote the square occupying the $i$th column and $j$th row $ij$. 
For example, the rooks in the following figure occupy positions 15, 23, 37, and 56 on the $6 \times 6$ diagonal chessboard:
\begin{center}
  \includegraphics[width=0.25\linewidth]{tableux.png}
\end{center}
Note that for any position $ij$ on the triangular chessboard, $i < j$. 

The following bijection shows $r_{n, k} = S(n, n - k)$. 
First, take any position on the diagonal $(n - 1) \times (n - 1)$ chess board with $k$ non-attacking rooks. 
Let $i_1 j_1, i_2 j_2, \ldots i_k j_k$ be the positions of the rooks.
Because no two rooks occupy the same row or column, $i_r \neq i_s$ and $j_r \neq j_s$ for any $r \neq s \in \{1, \ldots, k\}$. 
Thus, without loss of generality, let $i_r < i_s$ given $r < s$. 
Given two rooks in positions $i_r j_r$ and $i_s j_s$ with $r < s$, it is possible that $j_r = i_s$.
Say rooks $r < s$ are \textbf{connected} if there exist indices $l_1 < \ldots < l_t \in \{1, \ldots, k\}$ for some $t$ such that 
$$j_r = i_{l_1}, j_{l_1} = i_{l_2}, \ldots, j_{l_{t - 1}} = i_{l_t}, j_{l_t} = i_s$$ 
It is easy to see that the \textbf{connectiveness} of rooks is an equivalence relation for the set of rooks. 

For each equivalence class of connected rooks, let $S \subset [n]$ where $a \in S \iff $ there is a rook $r$ in the equivalence class such that $i_r = a$ or $j_r = a$.
Let $P$ be the collection of $S$'s for each equivalence class of rooks and singletons $\{a\}$ for each $a \in [n]$ such that $a$ is not in any $S$.
By the definition of connectiveness, $P$ is clearly a partition of $[n]$. 
Particularly, it is a partition of $[n]$ into $n - k$ parts:
Given an equivalence class of rooks $K$ with $\abs{K} = m$, one derives from the definition of connectiveness that a total of $m + 1$ unique indices in $\{1, \ldots, n\}$ are represented amongst the column or row coordinates of the rooks in the equivalence class. 
Let $K_1, \ldots, K_t$ be all the distinct equivalences classes of rooks with each $K_\alpha$ of size $m_\alpha$ and representing $m_\alpha + 1$ unique indices in $\{1, \ldots, n\}$. 
Because $K_\alpha$ and $K_\beta$ are disjoint for each $\alpha \neq \beta \in \{1, \ldots, t\}$, no common index is represented by both $K_\alpha$ and $K_\beta$. 
Thus the total number of unique indices represented by the equivalence classes of rooks is given by 
\begin{equation}
  (m_1 + 1) + (m_2 + 2) + \ldots + (m_t + 1)
\end{equation}
Of course, because $K_1, \ldots, K_t$ partition the set of all rooks, $m_1 + \ldots + m_t = k$. 
Thus equation (1) is equal 
$$m_1 + \ldots + m_t + t = k + t$$

Let $u$ be the number of indices in $\{1, \ldots, n\}$ that are not represented by any of the equivalence classes. 
That is, $u \defeq n - (k + t)$. 
By definition of $P$, $P$ is a partition of $[n]$ into $t + u$ parts.
But $t + u = n - k$, so $P$ is a partition of $[n]$ into $n - k$ parts, as was to be shown. 

Conversely, let $P$ be a partition of $[n]$ into $n - k$ parts. 

Then there are $n - k - u$ non-singleton elements in $P$;
$n - u$ indices of $[n]$ are not cintained in singletons in the partition $P$. 
Let $t \defeq n - k - u$ and $K_1, \ldots, K_t$ be the non-singleton elements of $P$.
Denoting $\abs{K_\alpha} = m_\alpha$, $\sum_{\alpha=1}^{t} m_\alpha = n - u$.


Now, given each $K_\alpha$, place $m_\alpha - 1$ rooks onto the diagonal $(n - 1) \times (n - 1)$ chessboard in the following way: 
if $K_\alpha = \{a_1, a_2, \ldots, a_{m_\alpha}\}$, place the $s$th rook on position $a_s a_{s + 1}$ for each $s \in \{1, \ldots, m_\alpha - 1\}$.
This construction ensures that no two rooks share the same column coordinate or the row coordinate, so none of the $m_\alpha - 1$ rooks attack the other. 

Because $K_1, \ldots, K_t$ are pairwise disjoint, it is clear that the rooks placed over all $\alpha \in \{1, \ldots, t\}$ are placed on distinct positions and none of the rooks attack another rook. 

Using the identites derived two paragraphs prior, one sees that the total number of rooks placed is given by 
$$  \sum_{\alpha = 1}^{t} (m_\alpha - 1) = \sum_{\alpha = 1}^{t} m_\alpha - t = n - u - t = k$$
Thus the partition of $[n]$ into $n - k$ parts yields a position with $k$ non-attacking rooks on the diagonal $(n - 1) \times (n - 1)$ board. 

\section{} 
The $k$ mutually non-attacking rooks occupy $k$ distinct rows and $k$ distinct columns, of which there are $\binom{n}{k}^2$ ways to do so. 
There are exactly $k!$ ways to place $k$ non-attacking rooks in the $k \times k$ board formed by the selected $k$ columns and $k$ rows. 
This yields 
$$r_{n, k} = k! \binom{n}{k}^2$$

\newpage
\section*{Collaboration Acknowledgement}
I used claude.ai for help with typesetting, particularly for the figures. 
Several solutions throughout the homework were inspired by the contents discucsed in Brendon's office hours. 

\end{document}

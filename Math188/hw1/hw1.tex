\documentclass[12pt]{article}

%% Basic document formatting
\usepackage{amsmath, amsthm, amssymb, amsfonts}
\usepackage{mathtools}
\usepackage{xspace}
\usepackage{thmtools}
\usepackage{graphicx}
\usepackage{tikz}
\usepackage{setspace}
\usepackage{fancyhdr}
\usepackage{titling}
\usepackage[left=0.4in,right=0.4in,top=1in,bottom=1in]{geometry}
\usepackage{float}
\usepackage{tabularx}
\usepackage[utf8]{inputenc}
\usepackage[english]{babel}
\usepackage{framed}
\usepackage[dvipsnames]{xcolor}
\usepackage{environ}
\usepackage{tcolorbox}
\tcbuselibrary{theorems,skins,breakable}

\usepackage{enumitem}
\setlist[enumerate]{leftmargin=*}
\setlist[enumerate,1]{labelindent=\parindent}
\setlist[enumerate,2]{labelindent=0pt}

% Blackboard Bold
\newcommand{\N}{\mathbb{N}}         % natural numbers
\newcommand{\Z}{\mathbb{Z}}         % integers
\newcommand{\Zpl}{\mathbb{Z}_{+}}   % positive integers
\newcommand{\Q}{\mathbb{Q}}         % rationals
\newcommand{\Qpl}{\mathbb{Q}_{+}}   % positive Rationals
\newcommand{\R}{\mathbb{R}}         % reals
\newcommand{\Rpl}{\mathbb{R}_{+}}   % positive Reals
\newcommand{\C}{\mathbb{C}}         % complex numbers
\newcommand{\F}{\mathbb{F}}         % field

% Words
\newcommand{\st}{\text{ such that }}
\newcommand{\wrt}{\text{ with respect to }}
\newcommand{\with}{\text{ with }}
\newcommand{\ie}{\text{, i.e. }}

% Operators
\newcommand{\abs}[1]{\left|#1\right|}                   % absolute value
\newcommand{\floor}[1]{\left\lfloor #1 \right\rfloor}   % floor
\newcommand{\ceil}[1]{\left\lceil #1 \right\rceil}      % ceiling

% Algebra 
\newcommand{\Syl}{\text{Syl}}           % set of Sylow-p subgroups 
\newcommand{\<}{\langle}                % \<x\>, subgroup generated by x 
\renewcommand{\>}{\rangle}
\newcommand{\id}{\text{id}}             % identity element
\newcommand{\order}[1]{\text{o}(#1)}    % order of an element
\let\oldcong\cong
\let\oldequiv\equiv
\renewcommand{\cong}{\oldequiv}
\renewcommand{\equiv}{\oldcong}

\makeatletter % cycle
\newcommand{\cyc}[1]{(\mathbf{\cyc@process#1\relax})}
\def\cyc@process#1#2\relax{%
	#1%
	\ifx\relax#2\relax
	\else
		\,\cyc@process#2\relax
	\fi
}
\makeatother

% Linear Algebra
\newcommand{\GL}{\text{GL}}                             % general linear group
\newcommand{\SL}{\text{SL}}                             % special linear group
\newcommand{\bmat}[1]{\begin{bmatrix}#1\end{bmatrix}}   % bracketed matrix
\newcommand{\rank}{\operatorname{rank}}                 % rank
\newcommand{\nullity}{\operatorname{nullity}}           % nullity

% Topology/Analysis
\newcommand{\ball}[2]{\text{B}_{#1}(#2)}  % B_r(x): open r-balls around x
\newcommand{\diam}{\text{diam}}           % diamter of a set in metric space 

% Misc Notation
\newcommand{\defeq}{\vcentcolon=}   % :=
\newcommand{\eqdef}{=\vcentcolon}   % =: 
\renewcommand{\bf}[1]{\textbf{#1}}


\renewcommand{\thesection}{\arabic{section}.}
\renewcommand{\thesubsection}{(\alph{subsection})}

\newtheorem{claim}{Claim}
\newtheorem*{lemma}{Lemma}

%% Headers & title setup
\newcommand{\course}{Math188}
\newcommand{\myname}{Jay Ser}
\setlength{\headheight}{14.5pt}
\pagestyle{fancy}
\fancyhf{}
\renewcommand{\headrulewidth}{0.4pt}
\lhead{\course}
\rhead{\myname}
\cfoot{\thepage}
\setlength{\droptitle}{-4em} 
\title{\course\ - HW \#1}
\author{\myname}
\date{2026.01.12}

\begin{document}
\maketitle
\thispagestyle{fancy}

%------------------------------------------------------------------------------%

\section{} 

\section{} 
\begin{center}
  \begin{tikzpicture}[scale=1.5]
    % 1x2 Vertical Domino
    \draw[thick] (0,0) rectangle (1,2);
    \draw[dotted] (0,1) -- (1,1);
    \fill (0.5,0.5) circle (0.05);
    \fill (0.5,1.5) circle (0.05);
    \node[below] at (0.5,-0.3) {Configuration A};

    % Two 2x1 Horizontal Dominos Stacked
    \draw[thick] (3,0) rectangle (5,1);
    \draw[dotted] (4,0) -- (4,1);
    \fill (3.5,0.5) circle (0.05);
    \fill (4.5,0.5) circle (0.05);

    \draw[thick] (3,1) rectangle (5,2);
    \draw[dotted] (4,1) -- (4,2);
    \fill (3.5,1.5) circle (0.05);
    \fill (4.5,1.5) circle (0.05);

    \node[below] at (4,-0.3) {Configuration B};

  \end{tikzpicture}
\end{center}

I prove an equivalent identity for when $n \geq 2$: 
$$f_n = \sum_{i = 0}^{n - 2} f_i + 1$$ 

Using the language of domino tiles, the definition $f_n = f_{n - 1} + f_{n -2 }$ can be understood as the sum of two cases from which the $n$th column can be filled: 
either through the use of a single vertical tile (configuration A) -- counted by $f_{n - 1}$ -- or the use of a two horizontal tiles stacked on top of each other (configuration B) -- counted by $f_{n - 2}$.  

Similarly, $f_{n - 1}$ can be decomposed into the sum of two cases: $f_{n - 2}$, which counts the number of ways to fill the $(n - 1)$th column using configuration A, and $f_{n - 3}$, which counts the number of ways to fill the $(n - 1)$th column using configuration B.

Continuously decomposing the number of ways to fill the $k$th column using configuration A, where $k \geq 2$, into
\begin{center}
number of ways to fill the $(k - 1)$th column using configuration A 
\\+ number of ways to fill the $(k - 1)$th column using configuration B
\end{center}
yields the equality 
$$f_n = f_0 + f_1 + f_1 + f_2 + f_3 + \ldots + f_{n - 2}$$ 
$f_1 = 1$, so the equality can be rewritten as
$$f_n = \sum_{i = 0}^{n - 2} f_n + 1$$
as desired. 
\section{} 

\section{} 

\section{} 

\section{} 

\section{} 

\newpage
\section*{Collaboration Acknowledgement}
I used claude.ai for help with typesetting, particularly for the figures. 

\end{document}

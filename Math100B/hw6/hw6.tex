\documentclass[12pt]{article}

%% Basic document formatting
\usepackage{amsmath, amsthm, amssymb, amsfonts}
\usepackage{mathtools}
\usepackage{xspace}
\usepackage{thmtools}
\usepackage{graphicx}
\usepackage{setspace}
\usepackage{fancyhdr}
\usepackage{titling}
\usepackage{hyperref}
\usepackage[left=0.4in,right=0.4in,top=1in,bottom=1in]{geometry}
\usepackage{float}
\usepackage{tabularx}
\usepackage[utf8]{inputenc}
\usepackage[english]{babel}
\usepackage{framed}
\usepackage[dvipsnames]{xcolor}
\usepackage{environ}
\usepackage{tcolorbox}
\tcbuselibrary{theorems,skins,breakable}

\usepackage{enumitem}
\setlist[enumerate]{leftmargin=*}
\setlist[enumerate,1]{labelindent=\parindent}
\setlist[enumerate,2]{labelindent=0pt}

% Blackboard Bold
\newcommand{\N}{\mathbb{N}}         % natural numbers
\newcommand{\Z}{\mathbb{Z}}         % integers
\newcommand{\Zpl}{\mathbb{Z}_{+}}   % positive integers
\newcommand{\Q}{\mathbb{Q}}         % rationals
\newcommand{\Qpl}{\mathbb{Q}_{+}}   % positive Rationals
\newcommand{\R}{\mathbb{R}}         % reals
\newcommand{\Rpl}{\mathbb{R}_{+}}   % positive Reals
\newcommand{\C}{\mathbb{C}}         % complex numbers
\newcommand{\F}{\mathbb{F}}         % field

% Words
\newcommand{\st}{\text{ such that }}
\newcommand{\wrt}{\text{ with respect to }}
\newcommand{\with}{\text{ with }}
\newcommand{\ie}{\text{, i.e. }}

% Operators
\newcommand{\abs}[1]{\left|#1\right|}                   % absolute value
\newcommand{\floor}[1]{\left\lfloor #1 \right\rfloor}   % floor
\newcommand{\ceil}[1]{\left\lceil #1 \right\rceil}      % ceiling

% Algebra 
\newcommand{\Syl}{\text{Syl}}           % set of Sylow-p subgroups 
\newcommand{\<}{\langle}                % \<x\>, subgroup generated by x 
\renewcommand{\>}{\rangle}
\newcommand{\id}{\text{id}}             % identity element
\newcommand{\order}[1]{\text{o}(#1)}    % order of an element
\let\oldcong\cong
\let\oldequiv\equiv
\renewcommand{\cong}{\oldequiv}
\renewcommand{\equiv}{\oldcong}

\makeatletter % cycle
\newcommand{\cyc}[1]{(\mathbf{\cyc@process#1\relax})}
\def\cyc@process#1#2\relax{%
	#1%
	\ifx\relax#2\relax
	\else
		\,\cyc@process#2\relax
	\fi
}
\makeatother

% Linear Algebra
\newcommand{\GL}{\text{GL}}                             % general linear group
\newcommand{\SL}{\text{SL}}                             % special linear group
\newcommand{\bmat}[1]{\begin{bmatrix}#1\end{bmatrix}}   % bracketed matrix
\newcommand{\rank}{\operatorname{rank}}                 % rank
\newcommand{\nullity}{\operatorname{nullity}}           % nullity

% Topology/Analysis
\newcommand{\ball}[2]{\text{B}_{#1}(#2)}  % B_r(x): open r-balls around x
\newcommand{\diam}{\text{diam}}           % diamter of a set in metric space 

% Misc Notation
\newcommand{\defeq}{\vcentcolon=}   % :=
\newcommand{\eqdef}{=\vcentcolon}   % =: 
\renewcommand{\bf}[1]{\textbf{#1}}


\renewcommand{\thesection}{\arabic{section}.}
\renewcommand{\thesubsection}{(\alph{subsection})}

\newtheorem{claim}{Claim}
\newtheorem*{lemma}{Lemma}

%% Headers & title setup
\newcommand{\course}{Math100B}
\newcommand{\myname}{Jay Ser}
\setlength{\headheight}{14.5pt}
\pagestyle{fancy}
\fancyhf{}
\renewcommand{\headrulewidth}{0.4pt}
\lhead{\course}
\rhead{\myname}
\cfoot{\thepage}
\setlength{\droptitle}{-4em} 
\title{\course\ - HW \#6}
\author{\myname}
\date{2026.02.19}

\begin{document}
\maketitle
\thispagestyle{fancy}

%------------------------------------------------------------------------------%
Here's something useful: 
\begin{claim} \label{claim:monic_irred}
  Let $D$ be a unique factorization domain and $F$ its field of fractions.  
  Suppose $f(x) \in D[x]$ is primitive. 
  Then $f$ is irreducible in $D[x] \iff f$ is irreducible in $F[x]$. 
\end{claim} 
\begin{proof}[Proof of Claim \ref{claim:monic_irred}] 
  If $f$ factors in $F[x]$, then it factors in $D[x]$ by Gauss's Lemma. 
  Conversely, suppose $f$ factors in $D[x]$. 
  Since $f$ is primitive, its factor is not a constant in $D$, so $f(x) = g(x) h(x)$ for nonconstant $g, h \in D[x]$. 
  Thus $f$ factors in $F[x]$ as well. 
\end{proof}

\section{}
\subsection{} 
$x^2 + 27x + 213 \cong x^2 + x + 1 \pmod{2}$, the latter being irreducible in $\Z / 2\Z$ since it is degree 2 and clearly has no roots. 
So $x^2 + 27x + 213$ is irreducible in $\Q[x]$. 

\subsection{}
$x^3 + 2x + 1$ is a degree 3 polynomial, so if it factors in $\Q[x]$, then it has a root in $\Q$. 
By the rational root theorem, if $r / s \in \Q$ with $\gcd(r, s) = 1$ is a root, then $r \mid 1$ and $s \mid 1$, i.e., $r = \pm 1$ and $s = \pm 1$, i.e., 
$$\text{The only possible roots of } x^3 + 2x + 1 \text{ are } \pm 1$$
But clearly, neither are. 
Hence $x^3 + 2x + 1$ is irreducible in $\Q[x]$. 

\subsection{} 
$x^5 - 3x^4 + 3$ is irreducible by the Eisenstein criterion with $p = 3$. 

\section{} 
$x^5 + 5x + 5$ is irreducible in $\Q[x]$ by the Eisenstein criterion with $p = 5$. 
Next, reducing modulo 2, $x^5 + 5x + 5 \cong x^5 + x + 1 \pmod{2}$.
$x^5 + x + 1$ clearly has no roots in $\Z / 2\Z$, so if it factors, then 
$$x^5 + x + 1 = (x^3 + ax^2 + bx + c)(x^2 + \alpha x + \gamma)$$ 
with $x^3 + ax^2 + bc + c$ and $x^2 + \alpha x + \gamma$ both irreducible. 
\begin{itemize}
  \item By Sieve of Eratosthenes or pure brute force, one can check that the only irreducible degree 2 polynomial in $\Z / 2\Z$ is $x^2 + x + 1$ (all others have roots). 
  \item Similarly, one can check that the only irreducible degree 3 polynomials in $\Z / 2\Z$ are $x^3 + x^2 + 1$ and $x^3 + x + 1$.  
\end{itemize}
From the two possible options for the degree 3 polynomial, one can check that 
$$x^5 + x + 1 = (x^2 + x + 1)(x^3 + x^2 + 1)$$ 
in $\Z / 2\Z$. 

\section{} 
In all these cases, because $f(x) \defeq x^3 + x + 1$ is a degree 3 polynomial, it suffices to check that $x^3 + x + 1$ has roots (or lack thereof). 
In the following, I drop the bar notation for convenience. 
\subsection{p = 2}
As noted in Q2, $f(x)$ has no roots in $\Z / 2 \Z$: 
$f(0) = f(1) = 1$. 
$f$ is irreducible in $\Z / 2\Z$. 

\subsection{p = 3}
$f(0) = f(2) = 1$ and $f(1) = 0$, so $x - 1 = x + 2$ is the only degree one factor of $f$.
Use the division algorithm to check that 
$$x^3 + x + 1 = (x + 2)(x^2 + x + 2)$$

\subsection{p = 5}
$f(0) = f(2) = f(3) = 1$, $f(1) = 3$, and $f(4) = 4$, so $f$ is irreducible in $\Z / 5\Z$. 

\section{}
It is an immediate result of the Eisenstein criterion that $x^n - p$ is irreducible for every every $n \geq 0$ and every prime integer $p$. 

\section{}
There are $5^2 = 25$ monic polynomials of degree 2 in $\Z / 5\Z[x]$. 
The ones that factor have the form $(x - a)(x - b)$ for $a, b \in \Z / 5\Z$.
To avoid double-counting the same polynomials, assume $a \geq b$. 
Then there are a total of 5, 4, 3, 2, 1 choices for $a$ when $b = 0, 1, 2, 3, 4$, respectively.
This sums to 15, so there are a total of 15 reducible degree 2 polynomials, and hence 10 irreducible degree 2 polynomials in $\Z / 5\Z [x]$. 

\end{document}

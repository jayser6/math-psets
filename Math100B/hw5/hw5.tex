\documentclass[12pt]{article}

%% Basic document formatting
\usepackage{amsmath, amsthm, amssymb, amsfonts}
\usepackage{mathtools}
\usepackage{xspace}
\usepackage{thmtools}
\usepackage{graphicx}
\usepackage{setspace}
\usepackage{fancyhdr}
\usepackage{titling}
\usepackage{hyperref}
\usepackage[left=0.4in,right=0.4in,top=1in,bottom=1in]{geometry}
\usepackage{float}
\usepackage{tabularx}
\usepackage[utf8]{inputenc}
\usepackage[english]{babel}
\usepackage{framed}
\usepackage[dvipsnames]{xcolor}
\usepackage{environ}
\usepackage{tcolorbox}
\tcbuselibrary{theorems,skins,breakable}

\usepackage{enumitem}
\setlist[enumerate]{leftmargin=*}
\setlist[enumerate,1]{labelindent=\parindent}
\setlist[enumerate,2]{labelindent=0pt}

% Blackboard Bold
\newcommand{\N}{\mathbb{N}}         % natural numbers
\newcommand{\Z}{\mathbb{Z}}         % integers
\newcommand{\Zpl}{\mathbb{Z}_{+}}   % positive integers
\newcommand{\Q}{\mathbb{Q}}         % rationals
\newcommand{\Qpl}{\mathbb{Q}_{+}}   % positive Rationals
\newcommand{\R}{\mathbb{R}}         % reals
\newcommand{\Rpl}{\mathbb{R}_{+}}   % positive Reals
\newcommand{\C}{\mathbb{C}}         % complex numbers
\newcommand{\F}{\mathbb{F}}         % field

% Words
\newcommand{\st}{\text{ such that }}
\newcommand{\wrt}{\text{ with respect to }}
\newcommand{\with}{\text{ with }}
\newcommand{\ie}{\text{, i.e. }}

% Operators
\newcommand{\abs}[1]{\left|#1\right|}                   % absolute value
\newcommand{\floor}[1]{\left\lfloor #1 \right\rfloor}   % floor
\newcommand{\ceil}[1]{\left\lceil #1 \right\rceil}      % ceiling

% Algebra 
\newcommand{\Syl}{\text{Syl}}           % set of Sylow-p subgroups 
\newcommand{\<}{\langle}                % \<x\>, subgroup generated by x 
\renewcommand{\>}{\rangle}
\newcommand{\id}{\text{id}}             % identity element
\newcommand{\order}[1]{\text{o}(#1)}    % order of an element
\let\oldcong\cong
\let\oldequiv\equiv
\renewcommand{\cong}{\oldequiv}
\renewcommand{\equiv}{\oldcong}

\makeatletter % cycle
\newcommand{\cyc}[1]{(\mathbf{\cyc@process#1\relax})}
\def\cyc@process#1#2\relax{%
	#1%
	\ifx\relax#2\relax
	\else
		\,\cyc@process#2\relax
	\fi
}
\makeatother

% Linear Algebra
\newcommand{\GL}{\text{GL}}                             % general linear group
\newcommand{\SL}{\text{SL}}                             % special linear group
\newcommand{\bmat}[1]{\begin{bmatrix}#1\end{bmatrix}}   % bracketed matrix
\newcommand{\rank}{\operatorname{rank}}                 % rank
\newcommand{\nullity}{\operatorname{nullity}}           % nullity

% Topology/Analysis
\newcommand{\ball}[2]{\text{B}_{#1}(#2)}  % B_r(x): open r-balls around x
\newcommand{\diam}{\text{diam}}           % diamter of a set in metric space 

% Misc Notation
\newcommand{\defeq}{\vcentcolon=}   % :=
\newcommand{\eqdef}{=\vcentcolon}   % =: 
\renewcommand{\bf}[1]{\textbf{#1}}


\renewcommand{\thesection}{\arabic{section}.}
\renewcommand{\thesubsection}{(\alph{subsection})}

\newtheorem{claim}{Claim}
\newtheorem*{lemma}{Lemma}

%% Headers & title setup
\newcommand{\course}{Math100B}
\newcommand{\myname}{Jay Ser}
\setlength{\headheight}{14.5pt}
\pagestyle{fancy}
\fancyhf{}
\renewcommand{\headrulewidth}{0.4pt}
\lhead{\course}
\rhead{\myname}
\cfoot{\thepage}
\setlength{\droptitle}{-4em} 
\title{\course\ - HW \#5}
\author{\myname}
\date{2026.02.15}

\begin{document}
\maketitle
\thispagestyle{fancy}

%------------------------------------------------------------------------------%

\section{} 
\subsection{} 
$x^2 + 1$ factors; $x = 2$ is a root. 
$$2^2 + 1 = 5 \cong 0 \pmod{5}$$

\subsection{} 
$x^2 - 3x - 3$ factors; $x = 1$ is a root since 
$$1^2 - 3(1) - 3 = -5 \cong 0 \pmod{5}$$

\subsection{} 
One can brute force check that $x^4 + 2$ has no roots, hence has no degree 1, and thereby no degree 3, factor. 
Thus if $x^4 + 2$ did factor, it would be of the form 
$$x^4 + 2 = (ax^2 + bc + c)(\alpha x^2 + \beta x + \gamma)$$ 
where the factors can be taken to monic since $\Z / 5 \Z$ is a field. 
Muliplying out the factors yields the following system of equations (among more) 
\begin{align}
  bd & = 2 \label{eq:bd} \\ 
  a + c & = 0 \label{eq:ac}\\ 
  ac + b + d & = 0 \label{eq:acbd}
\end{align}
Modding out symmetric solutions (notice $(b, d)$ have symmetric solutions and $(a, c)$ have symmetric solutions), \eqref{eq:bd} gives 
$$(b, d) \in \{(1, 2), (3, 4)\}$$
If $(b, d) = (1, 2)$, then \eqref{eq:acbd} is equivalent to $ac = 3$, which gives $(a, c) \in \{(1, 3), (2, 4)\}$. 
But both these solutions contradict \eqref{eq:ac}. 
If $(b, d) = (3, 4)$, then \eqref{eq:acbd} gives $(a, c) \in \{(1, 2), (3, 4)\}$, both of which contradict \eqref{eq:ac} once again. 

This shows that $x^4 + 2$ has no factors; it is irreducible. 

\section{} 
Induct on $d$, the degree of $f(x) \in F[x]$. 
If $d = 1$, write $f(x) = ax - b$ with $a \neq 0$.  
Then $x = ba^{-1}$ is the unique root for $f$, which proves the base case.

Suppose every polynomial of degree $d - 1$ has at most $d - 1$ roots, and take any $f(x) \in F[x]$ with degree $d$. 
If $f(x)$ has no roots, then there is nothing to be shown. 
Suppose $c$ is a root of $f(x)$. 
Then 
$$f(x) = (x - c)q(x)$$ 
where $q(x)$ must be a degree $d - 1$ polynomial since $F$ is a field. 
$q(x)$ has at most $d - 1$ roots, so $f(x)$ has at most $d$ roots, as was to be shown. 

If $F$ is not a field, then this is not true. 
For example, in $\Z / 4\Z$, the polynomial $g(x) = 2x$ has two roots: 0 and 2. 

\section{}  
Suppose there are only finitely many monic irreducible polynomials in $F[x]$. 
Let this set of irreducibles be 
$$\mathcal{F} = \{f_1, f_2, \ldots, f_n\}$$ 
Take $g(x) = \prod_{i = 1}^{n} f_i(x) + 1$. 
$g$ is clearly monic and not in $\mathcal{F}$, which means it is not irreducible.  
Since none of the $f_i \in \mathcal{F}$ divide 1 (due to the $f_i$ having a higher degree than the constant 1, for example), none of the $f_i \in \mathcal{F}$ divide $g(x)$. 
So a monic irreducible factor of $g$ exists, but it is not in the set of all monic irreducible polynomials, which is a contradiction.
Thus, $F[x]$ has infinitely many monic irreducible polynomials. 

Namely, if $F$ is a finite field, then $F[x]$ has irreducible polynomials of arbitrarily high degree because for any degree $d$, there are finitely many polynomials of degree $d$.  

\section{} 
\subsection{} 
Suppose $\<2, x\> = \<f\>$ for some $f(x) \in \Z[x]$. 
Then $f \mid 2$. 
Since $\Z$ is an integral domain $\implies \Z[x]$ integral domain $\implies \deg(f) \leq \deg(2) = 0$, $f(x)$ must be some constant, say $f(x) = c \in \Z$.  
But $\<f\> = \<c\> = \<2, x\>$ says that $c \mid x$, and the only constant that divides $x$ is 1 and -1. 
So $c$ is a unit, i.e., 
$$\<2, x\> = \<c\> = \Z[x]$$
But $x + 1$ is clearly not in $\<2, x\>$, which is a contradiction. 
Thus $\<2, x\>$ cannot be principal. 

\subsection{}  
\setcounter{equation}{0}
$f$ is an irreducible element of $\Z[x]$, which is a UFD. 
An element in a UFD is prime $\iff$ it is irreducible, so $f$ is a prime element of $\Z[x]$. 
Hence the principal ideal generated by $f$ is prime. 

While $\<f\>$ is maximal amongst principal ideals, it is not a maximal ideal. 
Take any prime number $p$ that doesn't divide the leading coefficint of $f$. 
Then by the correspondence theorem,  
\begin{equation} \label{eq:quotisom}
  \Z[x] / \<f(x), p\> \equiv \frac{\Z / p\Z [x]}{\<\bar{f(x)}\>}
\end{equation}
Since $p$ doesn't divide the leading coefficient of $f$, $\bar{f}$ has the same degree as $f$, say $n$. 
The units of $\Z / p\Z [x]$ are just the units of $\Z / p\Z$, i.e., the constant 1.  
$\bar{f}$ has the same degree as $f$ since $p$ doesn't dividie the leading coefficient of $f$. 
Namely, $\bar{f}$ is not a constant, so $\<\bar{f}\>$ is not the entire ring $\Z/p\Z [x]$, 
So \eqref{eq:quotisom} is not the zero ring, which means $\<f(x), p\>$ is a proper ideal properly containing $\<f(x)\>$.
$\<f(x)\>$ is not a maximal ideal. 

\section{} 
\subsection{} 
$a + bi$ is a root of 
$$q(x) \defeq (x - (a + bi))(x - (a - bi)) = x^2 - 2bx + a^2 + b^2$$
which, by the right hand side, is clearly a degree two polynomial in $\R[x]$. 

\subsection{} 
Consider the substitution homomorphism 
$$\varphi: \R[x] \rightarrow \C; \: x \mapsto a + bi, \, \left.\varphi\right|_{\R} = \id$$
It was shown multiple times throughout the course (e.g, the midterm) that $\ker \varphi = \<q\>$.
Since $f(a + bi) = 0$ by definition, $f \in \ker \varphi$, which means $q \mid f$. 

\subsection{} 
For any field $F$, all degree one polynomials in $F[x]$ are irreducible. 
If $ax - b = \alpha \beta$, $a \neq 0$, then $\alpha$ or $\beta$ must be a constant since $\deg(ax - b) = 1 = \deg(\alpha) + \deg(\beta)$.  

If $f(x) \in \R[x]$ has degree two, it is irreducible if and only if it has a complex root. 

Suppose $f(x)$ has degree greater than two.  
Since $\R[x] \hookrightarrow \C[x]$, $f(x)$ can be viewed as a complex polynomial. 
$\C$ is algebraically closed, so $f(x)$ has some root $\alpha$. 
If $\alpha$ has no complex part, then $f(x)$ has a linear real factor. 
If $\alpha$ has a complex part, then by (a) and (b), $f(x)$ has a quadratic factor. 
Either way, $f(x)$ factors and hence it is not irreducible.  


\section{} 
\newcommand{\rttwo}{\sqrt{-2}} 
Define the norm function 
$$\lambda:\Z[\rttwo] \rightarrow \Znn; \: a + b\rttwo \mapsto a^2 + 2b^2$$ 
See Q7 for a parallel analysis showing why $\lambda$ is a multiplicative function: 
$\forall \alpha, \beta \in \Z[\rttwo]$, $\lambda(\alpha \beta) = \lambda(\alpha) \lambda(\beta)$. 

Take any $\alpha, \gamma \in \Z[\rttwo]$, where $\alpha = a + b\rttwo$ and $\gamma \neq 0$. 
Since $\gamma$ is a complex number, $\gamma^{-1} = \frac{\bar{\gamma}}{\gamma \bar \gamma} \in \Q[\rttwo]$, hence $\alpha / \gamma = x + y\rttwo$ where $x, y \in \Q$.
So 
$$\exists \delta = c + d \rttwo \in \Z[\rttwo] \text{ where } \abs{x - c} \leq 1/2, \, \abs{y - d} \leq 1/2$$ 
Let $r \defeq \alpha - \gamma \delta \in \Z[\rttwo]$. 
Then, because the norm function is multiplicative and $\gamma \neq 0 \implies \lambda(\gamma) \neq 0$, the following analysis holds:   
\begin{align*}
  \lambda(r) / \lambda{\gamma}  & = \lambda(r / \gamma) \\ 
                                & = \lambda(x + y\rttwo - (c + d\rttwo)) \\ 
                                & = (x - c)^2 + 2(y - d)^2 \\ 
                                & \leq 1/4 + 1/2 \\
                                & < 1
\end{align*}
This shows that $\lambda(r) < \lambda(\gamma)$, so a division algorithm exists for $\Z[\rttwo]$; it is a Euclidean domain. 

\section{}
\newcommand{\adj}{\sqrt{-5}}
Define the following norm function on $\Z[\adj]$: 
$$\lambda: \Z[\adj] \rightarrow \Znn; \quad a + b\adj \rightarrow a^2 + 5b^2$$
The following calculation shows that $\lambda$ is a multiplicative function: 
\begin{align*}
  \lambda(a + b\adj) \lambda(c + d\adj) & = (a^2 + 5b^2)(c^2 + 5d^2) \\
                            & = a^2 c^2 + 25 b^2 d^2 + 5(b^2 c^2 + a^2 d^2) \\
  \lambda((a + b\adj)(c + d\adj)) & = \lambda(ac - 5bd + (bc + ad)\adj) \\
                            & = a^2 - 10abcd  + 25 b^2 d^2 + 5(b^2 c^2 + 2abcd + a^2 d^2) \\
                            & = a^2 c^2 + 25 b^2 d^2 + 5 b^2 c^2 + 5 a^2 d^2
\end{align*}
i.e., $\lambda(\alpha \beta) = \lambda(\alpha) \lambda(\beta) \, \forall \alpha, \beta \in \Z[\adj]$.

Suppose $\alpha \in \Z[\adj]$ is a unit. 
Then $\exists \beta \in \Z[\adj] \st \alpha \beta = 1$.
Applying the norm on both sides of the equality yields 
$$\lambda(\alpha) \lambda(\beta) = 1$$
Since $\lambda$ maps to the nonnegative integers, $\lambda(\alpha) = 1$.

Conversely, suppose $\lambda(\alpha) = 1$. 
Write $\alpha = a + b\adj$. 
Notice that $\lambda(a + b\adj) = (a + b\adj)(a - b\adj)$; 
$a - b\adj$ is $\alpha$'s inverse;
This shows that $\alpha \in \Z[\adj]$ is a unit $\iff \lambda(\alpha) = 1$. 

The norm function gives an easy method way of checking that 2, 3, and $1 \pm \adj$ are all irreducible. 
Suppose $2 = \alpha \beta$. 
Applying norm to both sides yields 
$$4 = \lambda(\alpha) \lambda(\beta)$$ 
Since $\lambda$ maps to nonnegative integers,
$$(\lambda(\alpha), \lambda(\beta)) \in \{(2, 2), (1, 4), (4, 1)\}$$ 
One can easily check from the definition of the norm function that 2 is not in the image of $\lambda$, so either $\lambda(\alpha) = 1$ or $\lambda(\beta) = 1$. 
Thus one of the factors must be a unit, hence 2 is irreducible. 
One can use this same method to verify that the 3 and $1 \pm \adj$ are irreducible. 
\begin{itemize}
  \item $3 = \alpha \beta \implies (\lambda(\alpha), \lambda(\beta)) \in \{(3, 3), (1, 9), (9, 1)\}$, but 3 is not in the image of $\lambda$. 
  \item $1 \pm \adj = \alpha \beta \implies (\lambda(\alpha), \lambda(\beta)) \in \{(2, 3), (3, 2), (1, 6), (6, 1)\}$, but 2 and 3 are not in the image of $\lambda$. 
\end{itemize}
Although 2 is an irreducible element, it is not a prime element of $\Z[\adj]$. 
$$2 \mid 6 = (1 + \sqrt{5})(1 - \sqrt{5}), \text{ but } 2 \nmid 1 + \sqrt{5}, 1 - \sqrt{5}$$ 

$\Z[\adj]$ is Noetherian, so irreducible decomposition exists. 
However, because irreducible elements are not neccessarily prime, irreducible decompositions are not unique.
This shows $\Z[\adj]$ is not a UFD.  

\section{}
\setcounter{equation}{0}
\subsection{} 
$\Z$ is a famous Euclidean domain that isn't a field. 

\subsection{} 
By Q4, $\Z[x]$ is a UFD that is not a PID.  

\subsection{} 
By Q7, $\Z[\sqrt{-5}]$ is not a UFD. 
It is an integral domain, however. 
Observe
\begin{equation} \label{eq:intdomain}
  (a + b\adj)(c + d\adj) = 0 \iff ac = 0, \, bd = 0, \, ad + bc = 0  
\end{equation}
  For any $a, b, c, d \in \Z$. 
To show that $\Z[\adj]$ is a UFD, it must be shown $(a, b) = (0, 0)$ or $(c, d) = (0, 0)$. 
Well, since $\Z$ is an integral domain, the right-hande side of \eqref{eq:intdomain} shows $a = 0$ or $c = 0$ and $b = 0$ or $d = 0$. 
Suppose $a = 0$ and $d = 0$.
Then $bc = 0$ and thus $b = 0$ or $c = 0$, thus either $(a, b) = (0, 0)$ or $(c, d) = (0, 0)$. 
Similarly, if $b = 0$ and $c = 0$, $ac = 0$, so once again, one of the pairs has to be $(0, 0)$. 
This shows that $\Z[\adj]$ is an integral domain.

\end{document}

\documentclass[12pt]{article}

%% Basic document formatting
\usepackage{amsmath, amsthm, amssymb, amsfonts}
\usepackage{mathtools}
\usepackage{xspace}
\usepackage{thmtools}
\usepackage{graphicx}
\usepackage{setspace}
\usepackage{fancyhdr}
\usepackage{titling}
\usepackage[left=0.4in,right=0.4in,top=1in,bottom=1in]{geometry}
\usepackage{float}
\usepackage{tabularx}
\usepackage[utf8]{inputenc}
\usepackage[english]{babel}
\usepackage{framed}
\usepackage[dvipsnames]{xcolor}
\usepackage{environ}
\usepackage{tcolorbox}
\tcbuselibrary{theorems,skins,breakable}

\usepackage{enumitem}
\setlist[enumerate]{leftmargin=*}
\setlist[enumerate,1]{labelindent=\parindent}
\setlist[enumerate,2]{labelindent=0pt}

% Blackboard Bold
\newcommand{\N}{\mathbb{N}}         % natural numbers
\newcommand{\Z}{\mathbb{Z}}         % integers
\newcommand{\Zpl}{\mathbb{Z}_{+}}   % positive integers
\newcommand{\Q}{\mathbb{Q}}         % rationals
\newcommand{\Qpl}{\mathbb{Q}_{+}}   % positive Rationals
\newcommand{\R}{\mathbb{R}}         % reals
\newcommand{\Rpl}{\mathbb{R}_{+}}   % positive Reals
\newcommand{\C}{\mathbb{C}}         % complex numbers
\newcommand{\F}{\mathbb{F}}         % field

% Words
\newcommand{\st}{\text{ such that }}
\newcommand{\wrt}{\text{ with respect to }}
\newcommand{\with}{\text{ with }}
\newcommand{\ie}{\text{, i.e. }}

% Operators
\newcommand{\abs}[1]{\left|#1\right|}                   % absolute value
\newcommand{\floor}[1]{\left\lfloor #1 \right\rfloor}   % floor
\newcommand{\ceil}[1]{\left\lceil #1 \right\rceil}      % ceiling

% Algebra 
\newcommand{\Syl}{\text{Syl}}           % set of Sylow-p subgroups 
\newcommand{\<}{\langle}                % \<x\>, subgroup generated by x 
\renewcommand{\>}{\rangle}
\newcommand{\id}{\text{id}}             % identity element
\newcommand{\order}[1]{\text{o}(#1)}    % order of an element
\let\oldcong\cong
\let\oldequiv\equiv
\renewcommand{\cong}{\oldequiv}
\renewcommand{\equiv}{\oldcong}

\makeatletter % cycle
\newcommand{\cyc}[1]{(\mathbf{\cyc@process#1\relax})}
\def\cyc@process#1#2\relax{%
	#1%
	\ifx\relax#2\relax
	\else
		\,\cyc@process#2\relax
	\fi
}
\makeatother

% Linear Algebra
\newcommand{\GL}{\text{GL}}                             % general linear group
\newcommand{\SL}{\text{SL}}                             % special linear group
\newcommand{\bmat}[1]{\begin{bmatrix}#1\end{bmatrix}}   % bracketed matrix
\newcommand{\rank}{\operatorname{rank}}                 % rank
\newcommand{\nullity}{\operatorname{nullity}}           % nullity

% Topology/Analysis
\newcommand{\ball}[2]{\text{B}_{#1}(#2)}  % B_r(x): open r-balls around x
\newcommand{\diam}{\text{diam}}           % diamter of a set in metric space 

% Misc Notation
\newcommand{\defeq}{\vcentcolon=}   % :=
\newcommand{\eqdef}{=\vcentcolon}   % =: 
\renewcommand{\bf}[1]{\textbf{#1}}


\renewcommand{\thesection}{\arabic{section}.}
\renewcommand{\thesubsection}{(\alph{subsection})}

\newtheorem{claim}{Claim}
\newtheorem*{lemma}{Lemma}

%% Headers & title setup
\newcommand{\course}{Math100B}
\newcommand{\myname}{Jay Ser}
\setlength{\headheight}{14.5pt}
\pagestyle{fancy}
\fancyhf{}
\renewcommand{\headrulewidth}{0.4pt}
\lhead{\course}
\rhead{\myname}
\cfoot{\thepage}
\setlength{\droptitle}{-4em} 
\title{\course\ - HW \#2}
\author{\myname}
\date{2026.01.18}

\begin{document}
\maketitle
\thispagestyle{fancy}

%------------------------------------------------------------------------------%
\section{} 
\subsection{}
$r(1 - 1) = 0 \iff 1r + (-1)r = 0 \iff r + (-1)r = 0$.
So $(-1)r$ is the additive inverse of $r$, i.e., $-r = -1(r)$.

\subsection{} 
Take any $r \in I$. 
$1 \in R \implies -1 \in R \implies -1 \cdot r = -r \in I$. 
So ideals are closed under additive inverses. 
Furthermore, $r - r = 0 \in I$.

\subsection{} 
Let $v \in R$ be the multiplicative inverse of $u$. 
$u \in I \implies uv = 1 \in I \implies 1r \in I$ for any $r \in R$, i.e., $I = R$.

\section{}
\subsection{} 
Suppose $\phi$ is injective and take any $x, y \in \ker \phi$. 
$\phi(x) = \phi(y) = 0 \implies x = y$ by definition of an injection, so $\ker \phi = 0$. 

Conversely, suppose $\ker \phi = 0$ and $\phi(x) = \phi(y)$.
Then $\phi(x) - \phi(y) = \phi(x - y) = 0 \implies x - y = 0 \implies x = y$. 
So $\phi$ is injective. 

\subsection{}
Forward direction follows immediately by the definition of a surjection. 
Conversely, suppose $R' \subset \phi(R)$ and $\exists a_1, a_2, \ldots, a_n \in R \st \phi(a_1) = x_1, \phi(a_2) = x_2, \ldots, \phi(a_n) = x_n$. 
Every element in $R'[x_1, \ldots, x_n]$, i.e., every multivariate polynomial, is the sum of products of elements of $R'$ and $x_1, \ldots, x_n$. 
Since $\phi$ is a homomorphism and each $r \in R$ and $x_1, \ldots, x_n$ have preimages, so do their products and sums. 
This shows that $\phi$ is surjective. 

\subsection{}
\begin{description}
  \item[Injection] $\left.\Phi\right|_{R} = \phi$, so $\Phi$ injective $\implies \phi$ by definition.
      Vice versa, suppose $\phi$ is injective and $\Phi(f(x)) = \Phi(g(x))$ in $S[x]$ for some $f(x) = \sum_{k = 0}^{n} a_k x^k, g(x) = \sum_{k = 0}^{m} b_k x^k \in R[x]$.
      Namely, suppose $f$ is of degree $n$ and $g$ is of degree $m$. 
      Using $'$ to denote images under $\Phi$, 
      $$\Phi(f(x)) = \sum_{k = 0}^{n} a_{k}' x^k, \quad \Phi(g(x)) = \sum_{k = 0}^{m} b_{k}' x^k$$
      Because $\Phi$ is a homomorphism, $\forall k \leq \min\{n, m\}$, the monomials 
      $a_{k}' x^k = b_{k}' x^k$, which implies $a' = b'$ in $S$. 
      Since $\phi$ is injective, $a = b$ in $R$. 
      Finally, suppose $n \neq m$; without loss of generality, suppose $n < m$. 
      Because $\Phi(f(x)) = \Phi(g(x))$ in $S[x]$, $b_{k}' = 0$ for all $n < k \leq m$. 
    But $b_{k}' = \Phi(b_k) = \phi(b_k)$, where $\phi$ is injective; 
    so $b_k = 0$. 
    Namely, $b_m = 0$. 
    This contradicts the fact that $g$ is a polynomial of degree $m$ in $R[x]$. 
    So $m = n$. 
    This shows that $f$ and $g$ have the same degree and the same coefficients, so $f = g$ in $R[x]$; 
    $\Phi$ is injective. 

  \item[Surjection] Once again, $\left.\Phi\right|_{R} = \phi$. 
      Suppose $\Phi$ is surjective. 
      Then every constant in $S[x]$, i.e., $\forall s \in S$, $\exists f(x) \in R[x] \st \Phi(f(x)) = s$. 
      If $f(x) = \sum_{k = 0}^{n} a_k x^k$, then $a_{k}' = 0$ for all $0 < k \leq n$ and $a_{0}' = s$. 
    So $\Phi(a_{0}') = s$ as well. 
    Namely, $a_0$ is a constant in $R[x]$, so $a_0$ can be identified as an element of $R$. 
    So $\forall s \in S$, $\Phi(r) = \phi(r) = s$ for some $r \in R$.
    This shows $\phi$ is surjective. 

    Conversely, suppose $\phi$ is surjective and take any $g(x) = \sum_{k = 0}^{n} b_k x^k \in S[x]$. 
    Then $\forall 0 \leq k \leq n$, $\exists a_k \in R \st \phi(a_k) = b_k$. 
    If $f(x) \defeq \sum_{k = 0}^{n} a_k x^k \in R[x]$, then $\Phi(f(x)) = g(x)$. 
    So $\Phi$ is surjective.  
\end{description}

\section{} 
\subsection{} 
In any $\Z / m\Z$, $\overline{n} \in \Z / m\Z$ is a zerodivisor $\iff \exists na = 0 \mod m$ for some $1 < a < m$.
One learns in elementary number theory that such an $a$ exists $\iff (n, m) > 1$. 
So the only elements of $\Z / m\Z$ which are not zerodivisors are those that are not prime to $m$.

\begin{itemize}
  \item In $\Z / 4\Z$, the only zerodivisor is $2$. 
  \item $\Z / 5\Z$ has no zerodivisors because 5 is a prime number. 
  \item In $\Z / 6\Z$, the zerodivisors are 2, 3, and 4.  
\end{itemize}

\subsection{} 
Suppose $ab = 0$ for some $a, b \neq 0$. 
If $a$ is a unit, then $\exists u \in R \st au = 1$. 
By the first equation, $abu = 0u = 0$. 
By the second equation, $abu = aub = 1 b = b$, so $b = 0$. 
This contradicts the assumption that $b \neq 0$. 

Because every nonzero element in a field is a unit, no element in a field is a zerodivisor. 
Hence a field an integral domain. 

\subsection{}
Suppose the characteristic of $F$ is not 0; 
let $n$ be the smallest positive integer such that $\sum_{i = 1}^{n} 1 = 0$. 
Suppose $n = ab$ for integers $a, b > 1$.  
$$(\sum_{i = 1}^{a} 1)(\sum_{i = 1}^{b} 1) = \sum_{i = 1}^{n} 1 = 0$$
where the two factors on the left hand side are both nonzero becuase $a, b < n$ and $n$ is the chracteristic of $F$. 
This contradicts the fact that any field is an integral domain.

\section{} 
\subsection{} 
$x \mapsto 0$ and $y \mapsto 0$ means every polynomial of degree greater than 0 is in the kernal. 
Vice versa, a nonzero element in the kernal can't have degree 0 because $\forall p \neq 0 \in \R \subset \R[x, y]$, $p \mapsto p \neq 0 \in \R$. 
This shows that the nonzero elements of the kernal are exactly the polynomials with degree greater than 0

\subsection{} 
Denote $K = \ker \phi$
Let 
\begin{align*}
  f(x) & \defeq (x - (2 + i))(x - (2 - i))\\
  & = (x - 2)^2 - i^2 \\ 
  & = x^2 -4x + 5 
\end{align*}

Since $i + 2$ is a root of $f(x)$, $(f(x)) \subset K$. 
Conversely, take $g(x) \in K$. 
Divide $g(x)$ with remainder by $f(x)$: 
$$g(x) = f(x)q(x) + r(x),\quad r(x) = 0 \text{ or } \deg(r) < 2$$
Suppose $r(x) \neq 0$. 
$g(x), f(x) \in K \implies r(x) \in K$. 
$r(x)$ is clearly not a nonzero constant since the image of $r(x)$ under $\phi$ would then just be the constant itself, a real number, which is not zero. 
But $r(x)$ cannot be linear either;
if $r(x) = ax + b$ with $a, b \in \R$, then 
$$r(i + 2) = a(i + 2) + b = 0 \implies i = -b / a - 2$$ 
which contradicts the fact that $i$ is not a real number.
So $\deg(r) \geq 2$, which contradicts the definition of $r$ induced by the division algorithm. 
This shows $r(x) = 0$, i.e., $g(x) \in (f(x)) \implies K \subset (f(x))$. 

So $K = (x^2 - 4x + 5)$. 

\subsection{} 
Denote $K = \ker \phi$. 
Let 
\begin{align*}
  f(x) & \defeq (x - (1 + \sqrt{2}))(x - (1 - \sqrt{2})) \\
       & = (x - 1)^2 - \sqrt{2}^2 \\
       & = x^2 - 2x - 1
\end{align*}

Do the same thing as (b)!!! 
$q + \sqrt{2}$ is a root of $f(x)$, so $(f(x)) \subset K$. 
Conversely, $f(x)$ divides any $g(x) \in K$ by basically the same argument as (b) because $f(x)$ is a degree 2 polynomial; 
$f(x)$ is the irreducible (lowest degree) polynomial with $1 + \sqrt{2}$ as a root.
So $K = (x^2 - 2x - 1)$. 

\section{}
\subsection{}
Suppose $r^n = 0$. 
Then
$$(1 + r)(\sum_{k = 0}^{n} (-1)^{k} r^k) = \sum_{k = 0}^{n} (-1)^k r^k + \sum_{k = 0}^{n} (-1)^k r^{k + 1} = 1$$ 
I hope the last equality is clear to the grader hehe...

\subsection{}
If $n$ is any positive integer, let $n \defeq \sum_{i = 1}^{n} 1$ in the abstract ring $R$.  

Notice that $1 + r = 1 + r^{p^m}$ for any power $m$. 
This is because 
$$(1 + r)^p = \sum_{k = 0}^{p} \binom{p}{k} r^k = 1 + r^p$$
because $p$ is a zerodivisor in $R$ and, as one learns in elementary number theory, for any prime $p$, $p \mid \binom{p}{k}$ for any $0 < k < p$.
Now, one can induct on $m$. 
Assume $1 + r^{p^m} = (1 + r^{p^{m - 1}}$, where $(1 + r^{p^{m - 1}})$ is a power of $1 + r$;
this means $1 + r^{p^m}$ is a power of $1 + r$ as well. 
Next, the binomial theorem applies once again to yield $(1 + r^{p^{m}})^p = 1 + (r^{p^{m}})^p = 1 + r^{p^{m} + 1}$. 
Namely, this shows $1 + r^{p^{m} + 1}$ is a power of $1 + r$. 

Suppose $r$ is nilpotent with $r^n = 0$. 
Let $\alpha$ be the smallest power $\st p^\alpha > n$.
Then $1 + r^{p^{\alpha}} = 1$ is a power of of $1 + r$, so $1 + r$ is unipotent.  

\section{}
\subsection{}
Take $(a + b), (c + d) \in I + J$, where $a, c \in I$ and $b, d \in I$. 
Then $(a + b) + (c + d) = (a + c) + (b + d) \in I + J$ since $a + c \in I$ and $b + d \in J$. 
Next, take any $r \in R$. 
Then $r(a + b) = ra + rb \in I + J$ since $ra \in I$ and $rb \in J$. 
This shows $I + J$ is an ideal.

\subsection{}
Take $a, b \in I \cap J$ and $r \in R$. 
$a + b \in I$ and $a + b \in J$ since $I$ and $J$ are ideals $\implies a + b \in I \cap J$. 
Similarly, $ra \in I \cap J$, so $I \cap J$ is an ideal.  

\subsection{} 
Take any $x, y \in IJ$. 
$x + y$ is just another sum whose summands are a product of an element of $I$ and an element of $J$, hence $x + y \in IJ$. 
Take any $r \in R$. 
By the distributive property, $r(x + y)$ is a sum whose summands are of the form $rab$ with $a \in I$ and $b \in J$. 
But $ra \in I$ because $I$ is an ideal, so the summand is in $IJ$, and thus the sum $r(x + y) \in IJ$ as well.
This shows $IJ$ is an ideal. 

To demonstrate that $S \defeq \{ab \mid a \in I, b \in J\}$ need not be an ideal, consider 
$$R \defeq \Z[x, y, z, w], \quad (x, y) \lhd R, \, (z, w) \lhd R$$ 
Then $xz, yw \in S$, but $xz + yw \notin S$. 
Suppose not; suppose $\exists ax + by \in I$, $cz + dw \in J \st$
$$(ax + by)(cz + dw) = xz + yw$$
where $a, b, c, d \in R$. 
Expanding the left hand side, one gets the following system of equations: 
\begin{align}
  ad & = 0 \\ 
  bc & = 0 \\ 
  bd & = 1 \\
  ac & = 1
\end{align}
(3) and (4) show that $a, b, c, d$ are all units, which contradicts (1) and (2) showing that one of $a$ and $d$ and one of $b$ and $c$ must be 0 
(since $\Z$ is an integral domain, $\Z[x, y, z, w])$ is an integral domain)

\section{} 
\subsection{} 
Define the substitution homomorphism  
$$\varphi: \Q[x] \rightarrow \Q[\sqrt{2}]; \quad \left.\varphi\right|_{\Q} = \id, \quad x \mapsto \sqrt{2}$$

Denote $K \defeq \ker \varphi$. 
Then $x^2 - 2 \mapsto 0$, so $(x^2 - 2) \subset K$. 
Conversely, take any $f(x) \in K$. 
Divide $f(x)$ with remainder by $x^2 - 2$: 
$$f(x) = q(x)f(x) + r(x), \quad r(x) = 0 \text{ or } \deg(r) < 2$$
If $r(x) = 0$, then $f(x) \in (x^2 - 2)$. 
Suppose $r(x) \neq 0$. 
Since $f(x)$ and $x^2 - 2$ are both in the ideal $K$, $r(x) \in K$. 
Now, $r(x)$ cannot be a constant because $\forall q \in \Q$, $q \mapsto q \neq \sqrt{2}$ since $\sqrt{2}$ is irrational. 
But $r(x)$ cannot be a linear polynomial either; 
if $r(x) = ax + b$ with $a, b \in \Q$ and $r(\sqrt{2}) = 0$, then $\sqrt{2} = -b/a \in \Q$, which is a contradiction. 
So $r(x)$ has degree greater or equal to 2, which contradicts the assumption on $r(x)$ due to the division algorithm. 
This shows that $r(x) = 0$, which implies $f(x) \in (x^2 - 2)$ always. 
Thus $K = (x^2 - 2)$, and 
$$\Q[x] / (x^2 - 2) \equiv \Q[\sqrt{2}]$$ 
by the First Isomorphism Theorem. 

\subsection{} 
Denote $I \defeq (6, 2x - 1)$. 
Notice that $6x + 3(2x - 1) = 3 \in I$, so $I = (3, 6, 2x - 1)$.
Since $3 \mid 6$, 6 is a redundant generator for $I$. In other words, $I = (3, 2x - 1)$. 

First, let $c_3: \Z[x] \rightarrow (\Z / 3\Z)[x]$ be the reduction homomorphism mod 3. 
Clearly the elements of the kernal are exactly all the polynomials of $\Z[x]$ such that 3 divides all of the polynomial's coefficients,
This means $\ker c_3 = 3\Z[x]$ (the ideal generated by 3 in $\Z[x]$), and it follows $\Z[x] / (3\Z[x]) \equiv (\Z / 3\Z) [x]$ by the First Isomorphism Theorem. 

Consider the image of $2x - 1$ under $c_3$, which, using bar notation, is $\overline{2}x - \overline{1}$.
Notice that the ideal $(\overline{2}x - \overline{1}) \lhd (\Z / 3\Z)[x]$ corresponds to $(2x - 1, 3) \lhd \Z[x]$, and thus $(\Z / 3 \Z)[x] / (\overline{2}x - \overline{1}) \equiv \Z[x] / (3, 2x - 1)$ by the Correspondence Theorem.
Thus, to identify the latter quotient ring as $\Z / 3\Z$, consider the substitution homomorphism
$$\varphi: (\Z / 3\Z)[x] \rightarrow \Z / 3\Z; \quad \left.\varphi\right|_{\Z / 3\Z} = \id, \quad x \mapsto \overline{2}$$
  $\overline{2}x - \overline{1} \mapsto \overline{2}(\overline{2}) - \overline{1} = \overline{4} - \overline{1} = 0$, so $\overline{2}x - \overline{1} \subset \ker \varphi$.  
  Conversely, suppose $\overline{g}(x) \in \ker \varphi$.
  Dividing $\overline{g}(x)$ with remainder by $\overline{2}x - \overline{1}$ in $(\Z / 3 \Z)[x]$ yields 
  $$\overline{g}(x) = (\overline{2}x - \overline{1})\overline{q}(x) + \overline{r}(x), \quad \overline{r}(x) = \overline{0} \text{ or } \overline{r}(x) = c \text{ for } c \neq \overline{0} \in \Z / 3 \Z$$
Suppose $r(x) \neq 0$. 
Then $\overline{r}(x) \mapsto \overline{r}(\overline{2}) = c \neq 0$. 
But $\overline{g}(x), \overline{2}x - \overline{1} \in \ker \varphi \implies \overline{r}(x) = c \neq 0 \in \ker \varphi$, which is a contradiction.
So $\overline{r}(x) = 0$, which shows $\ker \varphi = (\overline{2}x - \overline{1})$. 
By the First Isomorphism Theorem, 
$$\Z[x] / (6, 2x - 1) = \Z[x] / (3, 2x - 1) \equiv (\Z / 3 \Z)[x] / (\overline{2}x - \overline{1}) \equiv \Z / 3\Z$$ 

\end{document}

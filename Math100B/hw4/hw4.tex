\documentclass[12pt]{article}

%% Basic document formatting
\usepackage{amsmath, amsthm, amssymb, amsfonts}
\usepackage{mathtools}
\usepackage{xspace}
\usepackage{thmtools}
\usepackage{graphicx}
\usepackage{setspace}
\usepackage{fancyhdr}
\usepackage{titling}
\usepackage[left=0.4in,right=0.4in,top=1in,bottom=1in]{geometry}
\usepackage{float}
\usepackage{tabularx}
\usepackage[utf8]{inputenc}
\usepackage[english]{babel}
\usepackage{framed}
\usepackage[dvipsnames]{xcolor}
\usepackage{environ}
\usepackage{tcolorbox}
\tcbuselibrary{theorems,skins,breakable}

\usepackage{enumitem}
\setlist[enumerate]{leftmargin=*}
\setlist[enumerate,1]{labelindent=\parindent}
\setlist[enumerate,2]{labelindent=0pt}

% Blackboard Bold
\newcommand{\N}{\mathbb{N}}         % natural numbers
\newcommand{\Z}{\mathbb{Z}}         % integers
\newcommand{\Zpl}{\mathbb{Z}_{+}}   % positive integers
\newcommand{\Q}{\mathbb{Q}}         % rationals
\newcommand{\Qpl}{\mathbb{Q}_{+}}   % positive Rationals
\newcommand{\R}{\mathbb{R}}         % reals
\newcommand{\Rpl}{\mathbb{R}_{+}}   % positive Reals
\newcommand{\C}{\mathbb{C}}         % complex numbers
\newcommand{\F}{\mathbb{F}}         % field

% Words
\newcommand{\st}{\text{ such that }}
\newcommand{\wrt}{\text{ with respect to }}
\newcommand{\with}{\text{ with }}
\newcommand{\ie}{\text{, i.e. }}

% Operators
\newcommand{\abs}[1]{\left|#1\right|}                   % absolute value
\newcommand{\floor}[1]{\left\lfloor #1 \right\rfloor}   % floor
\newcommand{\ceil}[1]{\left\lceil #1 \right\rceil}      % ceiling

% Algebra 
\newcommand{\Syl}{\text{Syl}}           % set of Sylow-p subgroups 
\newcommand{\<}{\langle}                % \<x\>, subgroup generated by x 
\renewcommand{\>}{\rangle}
\newcommand{\id}{\text{id}}             % identity element
\newcommand{\order}[1]{\text{o}(#1)}    % order of an element
\let\oldcong\cong
\let\oldequiv\equiv
\renewcommand{\cong}{\oldequiv}
\renewcommand{\equiv}{\oldcong}

\makeatletter % cycle
\newcommand{\cyc}[1]{(\mathbf{\cyc@process#1\relax})}
\def\cyc@process#1#2\relax{%
	#1%
	\ifx\relax#2\relax
	\else
		\,\cyc@process#2\relax
	\fi
}
\makeatother

% Linear Algebra
\newcommand{\GL}{\text{GL}}                             % general linear group
\newcommand{\SL}{\text{SL}}                             % special linear group
\newcommand{\bmat}[1]{\begin{bmatrix}#1\end{bmatrix}}   % bracketed matrix
\newcommand{\rank}{\operatorname{rank}}                 % rank
\newcommand{\nullity}{\operatorname{nullity}}           % nullity

% Topology/Analysis
\newcommand{\ball}[2]{\text{B}_{#1}(#2)}  % B_r(x): open r-balls around x
\newcommand{\diam}{\text{diam}}           % diamter of a set in metric space 

% Misc Notation
\newcommand{\defeq}{\vcentcolon=}   % :=
\newcommand{\eqdef}{=\vcentcolon}   % =: 
\renewcommand{\bf}[1]{\textbf{#1}}


\renewcommand{\thesection}{\arabic{section}.}
\renewcommand{\thesubsection}{(\alph{subsection})}

\newtheorem{claim}{Claim}
\newtheorem*{lemma}{Lemma}

%% Headers & title setup
\newcommand{\course}{Math100B}
\newcommand{\myname}{Jay Ser}
\setlength{\headheight}{14.5pt}
\pagestyle{fancy}
\fancyhf{}
\renewcommand{\headrulewidth}{0.4pt}
\lhead{\course}
\rhead{\myname}
\cfoot{\thepage}
\setlength{\droptitle}{-4em} 
\title{\course\ - HW \#4}
\author{\myname}
\date{2026.02.08}

\begin{document}
\maketitle
\thispagestyle{fancy}

%------------------------------------------------------------------------------%

%TODO: I can assume polynomials in R[x] can be factored into at worst irreducible quadratic polynomials? 
\section{} 
One can assume $f(x)$ is monic;
if $f(x)$ has leading coefficient $a_n \neq 1$, then $\<f(x)\> = \<\frac{1}{a_n} f(x)\>$ as ideals in $\R$ since $a_n$ is a unit in $\R$.  
Since $f(x)$ has no repeated roots, either 
\begin{enumerate} 
  \item $f(x) = (x - a_1)(x - a_2)(x - a_3)$ with $a_1, a_2, a_3 \in \R$ distinct. 
  \item $f(x) = (x - a)(x^2 + c)$ with $c \in \Rpl$.  
\end{enumerate}

First, consider case 1. 
Then the assumption for Question 3 holds, so its result applies: 
$$\R[x] / \<f(x)\> \equiv \R^3$$

Next, assume case 2. 
$\<x^2 + c\>$ and $\<x - a\>$ are comaximal: 
$(x - a)(x + a) = x^2 - a^2 \in \<x - a\>$, and $x^2 + c - (x^2 - a^2) = c + a^2 \neq 0$, so the ideal $\<x^2 + c\> + \<x - a\>$ contains a unit,
i.e., $\<x^2 + c\> + \<x - a\> = \R[x]$. 
By the Chinese Remainder Theorem, 
$$\R[x] / \<x^2 + c\> \<x - a\> \equiv \R[x] / \<x^2 + c\> \times \R[x] / \<x - a\>$$ 
By definition, the product of ideals $\<x^2 + c\>\<x - a\>$ is equal to $\<(x^2 + c)(x - a)\> = \<f(x)\>$.
Also, it was shown in midterm 1 that $\R[x] / \<x^2 + c\> \equiv \C$ and by homework something, $\R[x] / \<x - a\> \equiv \R$.  
Putting all these isomorphisms together,  
$$\R[x] / \<f(x)\> \equiv \C \times \R$$ 

\section{}
\subsection{} 
Suppose $(x, y) \in F \times F$ is nilpotent, say $(x, y)^n = 0$ with $n \in \Zpl$.  
By the definition of the product ring, 
$$(x, y)^n = (x^n, y^n) = (0, 0)$$
Through the projection homomorphisms, one sees $x^n = y^n = 0$ in $F$. 
But nonzero nilpotent elements are zerodivisors, of which there are none in fields. 
So $x = y = 0$ in $F$. 
This shows that $F \times F$ has no nonzero nilpotent elements. 

\subsection{} 
$x \notin \<x^2\>$ because $\Q[x]$ is a Euclidean domain and $x$ is of lower degree than $x^2$. 
So $\bar {x} \neq 0$ in $\Q[x] / \<x^2\>$ and of course $\bar{x}^2 = 0$. 
So $\bar{x}$ is nilpotent in $\Q[x] / \<x^2\>$, which means $\Q[x] / \<x^2\>$ cannot be isomorphic to $\Q \times \Q$. 

\section{}
By the Chinese Remainder Theorem, 
$$\varphi: F[x] \rightarrow \prod_{i = 1}^{n} (F[x] / \<x - a_i\>) \: f(x) \mapsto (f(x) + \<x - a_1\>, \ldots, f(x) + \<x - a_n\>) $$
is an isomorphism with kernal $\<x - a_1\> \cap \ldots \cap \<x - a_n\>$. 
But since the $x - a_i$'s are distinct linear polynomials, the ideals $\<x - a_i\>$ and $\<x - a_j\>$, $i \neq j$, are pairwise comaximal: $x - a_i - (x - a_j) = a_j - a_i \neq 0$, which is a unit because it is a nonzero coefficient over a field, so the ideal $\<x - a_i\> + \<x - a_j\> = F[x]$. 
So 
$$F[x] / (\prod_{i = 1}^{n} \<x - a_i\>) \equiv \prod_{i = 1}^{n} (F[x] / \<x - a_i\>)$$
Additionally, it follows immediately from definition that the product ideal $\<x - a_i\> \<x - a_j\>$ is equivalent to $\<(x - a_i) (x - a_j)\>$.  
Hence $\prod_{i = 1}^{n} \<x - a_i\> = \<\prod_{i = 1}^{k} (x - a_i)\> = \<p(x)\>$.  
Also, it was shown whenever ago that $F[x] / \<x - c\> \equiv F$ for any $c \in F \setminus 0$.  
So 
$$F[x] / \<p(x)\> \equiv F^n$$ 

\section{}
Denote $F$ as the field of fractions of $R$.
For any integer $a \in \Znn$, let $a = \sum_{i = 1}^{a} 1$ in $R$ and $F$.

Recall that, by the construction of the field of fractions, $R$ embeds itself into $F$ as a subring. 
Let $n \defeq \text{char} R$. 
If $n = 0$ as an integer, then $\forall m \in \Znn$, $m \neq 0$ in $R$. 
Thus $m \neq 0$ in $F$, which means $\text{char} F = 0$ as well. 
If $n > 0$ as an integer, then $n = 0$ in $R$, and thus in $F$. 
Furthermore, $\forall m \in \Znn$ where $m < n$, $m \neq 0$ in $R$. 
It follows that $m \neq 0$ in $F$ as well, hence $\text{char} F = n$. 
This shows $\text{char} F = \text{char} R$. 

Let $n = \text{char} R = \text{char} F$, and suppose $n$ is not zero and not prime;
say $n = ab$, $a, b > 1$. 
Then $ab \neq 0$ in $F$.
Furthermore, $a, b \neq 0$ in $F$ because $a, b < n$ and $n$ is the characteristic of $F$. 
This shows that $a$ is a zerodivisor in $F$, which contradicts the fact that fields don't have zerodivisors.
It follows that the characteristic of an integral domain is 0 or a prime number. 

\section{}
For any $a \in R \setminus 0$, define 
$$\varphi: R \rightarrow R; \: r \mapsto ar$$ 
Because $R$ is an integral domain $\varphi$ is injective: 
$$ar = as \iff r = s$$
Because $R$ is finite, injection implies surjection. 
Finally, by definition of $\varphi$, it is clear that $\varphi(R) = \<a\>$. 
So $\<a\> = R$, which means $a$ is a unit.
This shows that every nonzero element of $R$ is a unit; 
$R$ is a field. 

\section{}
Since $F$ is finite, $\text{char} F \neq 0$. 
Namely, then, $\text{char} F = p$ for some prime $p$. 
Note that $\text{char} F$ is just the order of $1$ in the abelian group $F^{+}$.
So $p \mid \abs{F}$. 
Take any $x \in R \setminus 0$. 
Since 
$$px = 0x = 0$$
in $F$, where the integer $p$ is viewed as $\sum_{i = 1}^{p} 1$ in $F$, it follows $\order{x} \mid p$.
$x \neq 0$, the identity in the abelian group $F^{+}$, so $\order{x} \neq 1$, and $p$ is a prime integer, so $\order{x} = p$. 
Thus, it is impossible for any prime $q$ other than $p$ to divide the order of $F$. 
If $q$ did divide $F$, then by the Cauchy theorem, $\exists y \in F^{+} \st \order{y} = q$, which is a contradiction. 

\section{}
Here's a helpful lemma:
\begin{lemma}
  Suppose $R$ is a principal ideal domain. 
  Then $p \in R$ is irreducible $\iff p$ is prime.  
\end{lemma} 
\begin{proof}[Proof of Lemma]
  In any integral domain, prime $\implies$ irreducible:
  if $p = ab$ and $p$ is prime, $p \mid a$ or $p \mid b$. 
  But $a \mid p$ and $b \mid p$, so $p$ is associate with $a$ or $b$.
  In other words, the only elements that divide $p$ are its associates and units, which means $p$ is irreducible. 

  Next, suppose $p \in R$ is irreducible and $p = ab$. 
  It must be shown that if $p \nmid a$, then $p \mid b$. 
  Consider the ideal $\<p, a\>$. 
  Because $R$ is a PID, $\exists d \in R \st \<d\> = \<p, a\>$. 
  But by assumption on $p$ and $a$, the only elements that divides both $p$ and $a$ are units.
  Hence we can take $d = 1$, then $\exists r, s \in R \st rp + sa = 1$. 
  Multiplying both sides by $b$, 
  $$rbp + sab = b$$
  $p$ divides the left side of this equation, so $p \mid b$ as well. 
  This shows $p$ is prime. 
\end{proof} 

It was already shown in class that $F[x]$ is a Euclidean Domain (we showed that a Euclidean algorithm exists for $F[x]$). 
We then proved in the midterm that $F[x]$ is in fact a principal ideal domain (namely, we showed that a greatest common denominator exists for any two elements $f(x), g(x) \in F[x]$). 
Thus the lemma above applies and can be used to solve the problem. 

Given any $\varphi: F[x] \rightarrow R$, where $R$ is an integral domain, 
$$F[x] / \ker \varphi \equiv \varphi(R)$$ 
by the first isomorphism theorem.
But because $R$ is an integral domain, its subrings, and thus ideals, are integral domains as well. 
So $F[x] / \ker \varphi$ is an integral domain, which means $\ker \varphi$ is a prime ideal. 
0 is a prime ideal, so $\ker \varphi$ could be the zero ideal or a nonzero prime ideal. 
Suppose $\ker \varphi$ is not the zero ideal. 
Since $F[x]$ is a PID, let $\ker \varphi = \<p(x)\>$. 
A principal ideal is prime $\iff$ its generator is a prime element. 
But by the lemma above, prime elements are irreducible elements. 
Irreducible elements generate maximal ideals in PIDs (namely, we showed that irreducible elements of $F[x]$, i.e., the irreducible polynomials, generate max ideals in $F[x]$), so $\ker \varphi$ must be a maximal ideal in $F[x]$.  

\section{} % TODO: was the binomial theorem proven in class?
\subsection{} 
Recall that the binomial theorem holds in any commutative ring: 
$$(a + b)^n = \sum_{k = 0}^{n} \binom{n}{k} a^k b^{n - k}$$
where $\binom{n}{k}$ as an element of $R$ is the sum of the identity $\binom{n}{k}$ times. 

Take $a, b \in \rad I$.
Suppose $a^n \in I$ and $b^m \in I$ for $n, m \in \Zpl$. 
For any $r \in R$, $(ra)^n = r^n a^n \in I$, so $\rad I$ is closed under scalar multiplication. 
To see it is closed under addition as well, consider 
$$(a + b)^{m + n} = \sum_{k = 0}^{m + n} a^k b^{m + n - k}$$
In each of the summand, either $k > n$ or $m + n - k > m$, so at least one of $a^k$ and $b^{m + n - k}$ is in $I$ and thus the entire term. 
Since each summand is in $I$, the sum is in $I$ as well. 
$(a + b)^{m + n} \in I$, so $a + b \in \rad I$. 
This shows that $\rad I$ is an ideal. 

\subsection{}
Let $I$ be a prime ideal. 
By definition, $I \subset \rad I$. 
Next, take any $a \in \rad I$ with $a^n \in I$. 
$a^n = a^{n - 1}a \in I$ and $I$ is prime, so either $a \in I$ or $a^{n - 1}$. 
If the former, there is nothing more to be shown. 
If the latter, $a^{n - 1} = a^{n - 2} a \in I$, so $a^{n - 2} \in I$ or $a \in I$.
If the latter, there is nothing more to be shown. 
If the former, then one can continue reducing the exponent of $a$ until either $a \in I$ or $a^2 \in I$. 
If $a^2 \in I$, then clearly $a \in I$. 

\subsection{}
The statement is equivalent to showing 
$$\forall P \in \spec{R}, \: \rad 0 \subset P$$
Take arbitrary $P \in \spec R$ and $a \in \rad 0$ with $a^n = 0$. 
$a^n = 0 \in P$, so by similar reasoning as (b), $a \in P$. 
This shows $\rad 0 \subset P$. 

\end{document}

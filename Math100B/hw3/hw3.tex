\documentclass[12pt]{article}

%% Basic document formatting
\usepackage{amsmath, amsthm, amssymb, amsfonts}
\usepackage{mathtools}
\usepackage{xspace}
\usepackage{thmtools}
\usepackage{graphicx}
\usepackage{setspace}
\usepackage{fancyhdr}
\usepackage{titling}
\usepackage[left=0.4in,right=0.4in,top=1in,bottom=1in]{geometry}
\usepackage{float}
\usepackage{tabularx}
\usepackage[utf8]{inputenc}
\usepackage[english]{babel}
\usepackage{framed}
\usepackage[dvipsnames]{xcolor}
\usepackage{environ}
\usepackage{tcolorbox}
\tcbuselibrary{theorems,skins,breakable}

\usepackage{enumitem}
\setlist[enumerate]{leftmargin=*}
\setlist[enumerate,1]{labelindent=\parindent}
\setlist[enumerate,2]{labelindent=0pt}

% Blackboard Bold
\newcommand{\N}{\mathbb{N}}         % natural numbers
\newcommand{\Z}{\mathbb{Z}}         % integers
\newcommand{\Zpl}{\mathbb{Z}_{+}}   % positive integers
\newcommand{\Q}{\mathbb{Q}}         % rationals
\newcommand{\Qpl}{\mathbb{Q}_{+}}   % positive Rationals
\newcommand{\R}{\mathbb{R}}         % reals
\newcommand{\Rpl}{\mathbb{R}_{+}}   % positive Reals
\newcommand{\C}{\mathbb{C}}         % complex numbers
\newcommand{\F}{\mathbb{F}}         % field

% Words
\newcommand{\st}{\text{ such that }}
\newcommand{\wrt}{\text{ with respect to }}
\newcommand{\with}{\text{ with }}
\newcommand{\ie}{\text{, i.e. }}

% Operators
\newcommand{\abs}[1]{\left|#1\right|}                   % absolute value
\newcommand{\floor}[1]{\left\lfloor #1 \right\rfloor}   % floor
\newcommand{\ceil}[1]{\left\lceil #1 \right\rceil}      % ceiling

% Algebra 
\newcommand{\Syl}{\text{Syl}}           % set of Sylow-p subgroups 
\newcommand{\<}{\langle}                % \<x\>, subgroup generated by x 
\renewcommand{\>}{\rangle}
\newcommand{\id}{\text{id}}             % identity element
\newcommand{\order}[1]{\text{o}(#1)}    % order of an element
\let\oldcong\cong
\let\oldequiv\equiv
\renewcommand{\cong}{\oldequiv}
\renewcommand{\equiv}{\oldcong}

\makeatletter % cycle
\newcommand{\cyc}[1]{(\mathbf{\cyc@process#1\relax})}
\def\cyc@process#1#2\relax{%
	#1%
	\ifx\relax#2\relax
	\else
		\,\cyc@process#2\relax
	\fi
}
\makeatother

% Linear Algebra
\newcommand{\GL}{\text{GL}}                             % general linear group
\newcommand{\SL}{\text{SL}}                             % special linear group
\newcommand{\bmat}[1]{\begin{bmatrix}#1\end{bmatrix}}   % bracketed matrix
\newcommand{\rank}{\operatorname{rank}}                 % rank
\newcommand{\nullity}{\operatorname{nullity}}           % nullity

% Topology/Analysis
\newcommand{\ball}[2]{\text{B}_{#1}(#2)}  % B_r(x): open r-balls around x
\newcommand{\diam}{\text{diam}}           % diamter of a set in metric space 

% Misc Notation
\newcommand{\defeq}{\vcentcolon=}   % :=
\newcommand{\eqdef}{=\vcentcolon}   % =: 
\renewcommand{\bf}[1]{\textbf{#1}}


\renewcommand{\thesection}{\arabic{section}.}
\renewcommand{\thesubsection}{(\alph{subsection})}

\newtheorem{claim}{Claim}
\newtheorem*{lemma}{Lemma}

%% Headers & title setup
\newcommand{\course}{Math100B}
\newcommand{\myname}{Jay Ser}
\setlength{\headheight}{14.5pt}
\pagestyle{fancy}
\fancyhf{}
\renewcommand{\headrulewidth}{0.4pt}
\lhead{\course}
\rhead{\myname}
\cfoot{\thepage}
\setlength{\droptitle}{-4em} 
\title{\course\ - HW \#3}
\author{\myname}
\date{2026.01.23}

\begin{document}
\maketitle
\thispagestyle{fancy}

%------------------------------------------------------------------------------%
Even though they are more or less the same things, I prefer thinking about ideals rather than quotient rings, so here are some propositions. 
\begin{claim}
  $R / I$ is an integral domain $\iff$ $I$ is a prime ideal 
\end{claim}
\begin{proof}[Proof of Claim 1] 
  \begin{align*}
    I \text{ is a prime ideal } & \iff \forall ab \in I, a \in I \text{ or } b \in I \\ 
                                & \iff \forall \bar{a} \bar{b} \in R / I, \, \bar{a} \bar{b} = 0 \implies \bar{a} = 0 \text{ or } \bar{b} = 0 \\ 
                                & \iff R / I \text{ is an integral domain}
  \end{align*}
\end{proof}

\begin{claim}
  $R / I$ is a field $\iff$ $I$ is a maximal ideal 
\end{claim}
\begin{proof}[Proof of Claim 2]
  By the definition of a field, namely, that every nonzero element has a multiplicative inverse, it is clear that a ring has only two ideals -- the zero ring and the entire ring -- if and only if the ring is a field. 
  Now, via the Correspondence Theorem, 
  \begin{align*}
    R / I \text{ is a field} & \iff 0 \text{ and } R / I \text{ are the only ideals in } R / I \\ 
                             & \iff \text{in } R, \text{ the only ideal properly containing } I \text{ is } R \\ 
                             & \iff I \text{ is a max ideal in } R
  \end{align*}
\end{proof}

\section{} 
Consider the substitution homomorphism 
$$\varphi: R[x] \rightarrow R; \: x \mapsto a, \, \left.\varphi\right|_{R} = \id$$
Clearly, $x - a \in \ker \varphi$, so $\<x - a\> \subset \ker \varphi$.
Take any $f(x) \in \ker \varphi$. 
Because $x - a$ is monic, it is possible to divide $f(x)$ with remainder by $x - a$ to get 
$$f(x) = q(x)(x - a) + r, \text{ where } r = c \text{ for some constant } c \in R$$
Because $f(x), x - a \in \ker \varphi$, $r \in \ker \varphi$. 
So $\varphi(f(x)) = \varphi(r) = r = c = 0$, which means $r = 0$. 
In other words, $x - a \mid f(x)$, which shows $\ker \varphi = \<x - a\>$.
Finally, $\varphi$ is clearly surjective since it is the identity function on constant polynomials.
Thus, by the First Isomorphism Theorem, $R[x] / \<x - a\> \equiv R$.

\section{} 
Suppose $n$ is a perfect square, say $n = a^2$ for some $a \in \Z$. 
Then $\sqrt{n} = a$, so $\Q[\sqrt{n}]$ is just $\Q[a] = \Q$. 
Meanwhile, $\Q[x] / \<x^2 - n\>$ is not even an integral domain: 
$x^2 - n = (x + a)(x - a)$;
namely, $x^2 - n$ divides the product $(x + a)(x - a)$. 
However, $x^2 - n$ clearly does not divide $x + a$ nor $x - a$ because $\Q[x]$ is an integral domain (see Q3) and $x^2 - n$ has a greater degree than both $x + a$ and $x - a$. 
This shows that the principal ideal $\<x^2 - n\>$ is not a prime ideal, and hence $\Q[x] / \<x^2 - n\>$ is not an integral domain. 
$\Q[\sqrt{n}] = \Q$, on the other hand, is an integral domain, so $\Q[a] \neq \Q[x] / \<x^2 - n\>$.

To show the reverse direction, suppose $n$ is not a perfect square. 
Then $x^2 - n$ is the least degree polynomial in $\Q[x]$ that has $\sqrt{n}$ as a root (clearly no linear polynomial in $\Q[x]$ has $\sqrt{n}$ as a root, since $\sqrt{n} \notin \Q$). 
Now consider the substitution homomorphism 
$$\varphi: \Q[x] \rightarrow \Q[\sqrt{n}]; \: x \mapsto \sqrt{n}, \, \left.\varphi\right|_{\Q} = \id$$
Clearly $x^2 - n \in \ker \varphi$, so $\<x^2 - n\> \subset \ker \varphi$. 
Next, take any $f(x) \in \ker \varphi$. 
Since $\Q$ is a field and therefore $\Q[x]$ is a Euclidean Domain, divide $f(x)$ with remainder by $x^2 - n$ to obtain 
$$f(x) = q(x)(x^2 - n) + r(x), \text{ where } r(x) = 0 \text{ or } \deg(r) < 2$$
Suppose $r(x) \neq 0$. 
$f(x), x^2 - n \in \ker \varphi$, so $r(x) \in \ker \varphi$ as well. 
So $\varphi(f(x)) = \varphi(r(x)) = 0$. 
But the image of $r(x)$ under the substitution map cannot be zero because $r(x)$ has degree less than $x^2 - n$, the latter being the least-degree polynomial that has $\sqrt{n}$ as a root, thus yielding a contradiction. 
As such, $r(x) = 0$; 
$x^2 - n \mid f(x)$. 
This shows $\<x^2 - n\> = \ker \varphi$. 
Finally, it is clear that the map is surjective because 1, $\sqrt{n}$ form a basis for the image of $\varphi(\Q[x])$, which generate the entirity of $\Q[\sqrt{n}]$. 
Thus, by the First Isomorphism Theorem, $\Q[x] / \<x^2 - n\> \equiv \Q[\sqrt{n}]$. 

\section{}
Take any $f(x), g(x) \in R[x] \setminus 0$ and let $f(x) = a_n x^n + \ldots + a_0$, $g(x) = b_m x^m + \ldots + b_0$, where $a_n \neq 0$ and $b_m \neq 0$.
Then $f(x) g(x) = a_n b_m x^{n + m} + $ lower degree terms. 
Because $R$ is an integral domain, $a_n b_m \neq 0$, so $f(x) g(x) \neq 0$ as well. 
This shows that $R[x]$ is an integral domain. 

Particularly, the proof above shows that $\deg(fg) = \deg(f) + \deg(g)$. 
Because every nonzero element of $\<f(x)\>$ is of the form $f(x) g(x)$ for $g(x) \neq 0$, every element in the ideal generated by the polynomial $f$ has a degree greater or equal to the degree of $f$.  

\section{}
\subsection{} 
Let $I \defeq \<2x - 6, x - 10\>$. 
Let's directly identify $\Z[x] / I$. 
First, following my solution to Q1, the substitution map 
$$\varphi: \Z[x] \rightarrow \Z; \: x \mapsto 10, \, \left.\varphi\right|_\Z = id$$ 
gives the isomorphism $\Z[x] / \<x - 10\> \equiv \Z$.
$\ker \varphi = \<x - 10\> \subset I$, so by the Correspondence Theorem, $I \mapsto \<2\overline{x} - 6\> = \<20 - 6\> = \<14\>$ in $\Z$.
14 is not a prime number, so $\Z / 14\Z$ is not an integral domain by HW\#2. 
The Correspondence Theorem says 
$\Z[x] / I \equiv \Z / \<14\>$, so $\Z[x] / I$ is not an integral domain.  

\subsection{} 
$\Z[x] / \<x^2 + x\>$ is not an integral domain because $\<x^2 + x\>$ is not a prime ideal in $\Z[x]$:
$x^2 + x = x(x + 1) \in \<x^2 + x\>$, but $x$ and $x + 1$ are both not in $\<x^2 + x\>$ because $x$ and $x + 1$ are both of lower degree than $x^2 + x$ and $\Z[x]$ is an integral domain; 
hence, $x^2 + x$ cannot divide $x$ or $x + 1$. 

\section{}
\subsection{} 
1 is not a zerodivisor: 
$\forall a \neq 0 \in R$, $1a = a \neq 0$. 
Now, the result follows by (b)

\subsection{} 
Suppose $\varphi$ is not injective:
$\exists a \neq b \in R \st a + \<f\> = b + \<f\>$ in $R[x] / (f)$.
The latter condition is equivalent to $a - b \in (f)$ in $R[x]$.
Now, since $a \neq b$ in $R$, $a - b \neq 0$ in $R[x]$.
Furthermore, $\forall g \in R[x]$, $\deg(fg) \geq \deg(f)$ since the leading coefficient of $f$ is not a zerodivisor.
So no nonzero constant is in $\<f\>$, which contradicts $a - b \in \<f\>$. 
This shows that $\varphi$ is injective.

\subsection{}
Consider
$$\varphi: \Z / 12 \Z \rightarrow (\Z / 12\Z)[x] / \<3x - 1\>$$
$4(3x - 1) = 12x - 4 \cong 8 \pmod{12}$, so $8 \in \<3x - 1\>$ in $(\Z / 12\Z)[x]$.
Denote $I = \<3x - 1\>$. 
Then $9 + I = 1 + I$ since $(9 - 1) + I = 8 + I = 0 + I$.
In other words, $9 \neq 1$ in $\Z / 12 \Z$, but they map to the same value in $(\Z / 12\Z)[x] / \<3x - 1\>$, so $\varphi$ is not injective. 

\section{} 
Suppose $x^2 - a$   has no root. 
This is equivalent to saying that $x^2 - a$ has no linear factor. 
But because $x^2 - a$ is a polynomial of degree 2 in the integral domain $F[x]$, $x^2 - a$ having no linear factor means $x^2 - a$ is irreducible in $F[x]$.
$F[x]$ is a Euclidean Domain. 
Namely, it is a Principal Ideal Domain, so $x^2 - a$ generates a maximal ideal in $F[x]$. 
This shows that $F[x] / (x^2 - a)$ is a field.

Conversely, if $x^2 - a$ does have a root, say $f(\alpha) = 0$, then 
$$x^2 - a = (x - \alpha)(x - \beta)$$ 
for some $\beta \in F$.
Because $F[x]$ is an integral domain, neither $x - \alpha$ nor $x - \beta$ are in $\<x^2 - a\>$ even though $(x - \alpha)(x - \beta) \in \<x^2 - a\>$. 
So $\<x^2 - a\>$ is not a prime ideal, which means $F[x] / \<x^2 - a\>$ is not even an integral domain;
then certainly, $F[x] / \<x^2 - a\>$ is not a field.  

\section{}
\subsection{} 
Consider the canonical homomorphism 
$$\pi: R[x] \rightarrow R[x] / \<ax - 1\>$$
By definition, $\ker \pi = \<ax - 1\>$, so $ax - 1 \mapsto 0$. 
Using bar notation to denote the image, 
$$\overline{a} \overline{x} - 1 = 0$$
So $\overline{a}$ indeed does have an inverse in the quotient; 
namely $\overline{a}^{-1} = \overline{x}$. 

\subsection{} 
Consider the map 
Take any $f(x) = b_n x^n + + b_{n - 1} x^{n-1} \ldots + b_1 x + b_0 \in R[x]$. 
Under the projection map to $R[x] / \<ax - 1\>$, 
\begin{align*}
  f(x) & \mapsto b_n a^{-n} + b_{n - 1}a^{-n + 1} + \ldots + b_1 a^{-1} + b_0 \\ 
       & = a^{-n} (b_n + b_{n - 1} a + \ldots + b_1 a^{n - 1} + b_0 a^n) \\ 
       & = a^{-n} b 
\end{align*}
where bars have been dropped for convenience and where $b = b_n + b_{n - 1} a + \ldots + b_1 a^{n - 1} + b_0 a^n \in R$.  

\subsection{} 
Suppose $a^n b = 0$. 
Then $\bar{a}^n \bar{b} = \bar{0}$. 
Multiplying both sides by $\bar{x}^n$, $\bar{b} = \bar{0}$. 
So $b \in \ker \varphi$.

Conversely, suppose $b \in \ker \varphi$. 
Now, by definition of $\phi$, $\ker \varphi = \ker \pi \cap R$, where $\pi$ is the canonical proejction map. 
So $b \in \ker \pi = \<2x = 1\>$, i.e., 
$$b = (ax - 1)(c_n x^n + c_{n - 1} x^{n - 1} + \ldots + c_0)$$ 
Then 
\begin{align*}
  c_0 & = -b \\ 
  a c_0 - c_1 & = 0 \\ 
  a c_1 - c_2 & = 0 \\ 
           & \vdots \\ 
  a c_{n - 1} - c_n & = 0 \\ 
  ac_n & = 0
\end{align*}
Inducting down the system of equations, $a^{n + 1} b = 0$

\subsection{}
The result follows from (c). 
By the First Isomorphism Theorem, $\varphi(S) \equiv R / \ker \varphi$. 
If $S = 0$, then $\varphi(S) = 0$, which forces $\ker \varphi = R$. 
Thus $\forall b \in R$, $a^n b = 0$ for some $n \in \Znn$.
Namely, $\exists m \in \Znn \st a^m a = 0$, or in other words, $a^{m + 1} = 0$. 
So $a$ is nilpotent. 

Conversely, if $a$ is nilpotent, say $a^k = 0$, then $\forall b \in R$, $ba^k = 0$. 
Hence $\ker \varphi = R$.
Extending the result to the substition map, which by definition of $\varphi$ is equivalent to the canonical projection map, there is a homomorphism
$$\varPhi: R[x] \rightarrow S; \: x \mapsto \bar{a}^{-1}, \, \left.\varPhi\right|_{R} = \varphi$$
But because $\varphi$ is just the zero map, $\varPhi$ is also just the zero map, i.e., $\ker \varPhi = \varPhi$.
Furthermore, $\varPhi$ is the canonical projection map, so it is surjection. 
Thus, by the First Isomorphism Theorem, 
$$R[x] / \ker \varPhi \equiv 0 \equiv S$$
and hence $S = 0$.
\\*** note that because $0 = 1$ in the zero ring, the homomorphisms $\varphi$ and $\varPhi$ still observe the $1 \mapsto 1$ rule for ring homomorphisms

\end{document}

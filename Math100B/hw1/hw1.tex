
\documentclass[12pt]{article}

%% Basic document formatting
\usepackage{amsmath, amsthm, amssymb, amsfonts}
\usepackage{mathtools}
\usepackage{xspace}
\usepackage{thmtools}
\usepackage{graphicx}
\usepackage{setspace}
\usepackage{fancyhdr}
\usepackage{titling}
\usepackage[left=0.4in,right=0.4in,top=1in,bottom=1in]{geometry}
\usepackage{float}
\usepackage{tabularx}
\usepackage[utf8]{inputenc}
\usepackage[english]{babel}
\usepackage{framed}
\usepackage[dvipsnames]{xcolor}
\usepackage{environ}
\usepackage{tcolorbox}
\tcbuselibrary{theorems,skins,breakable}

\usepackage{enumitem}
\setlist[enumerate]{leftmargin=*}
\setlist[enumerate,1]{labelindent=\parindent}
\setlist[enumerate,2]{labelindent=0pt}

% Blackboard Bold
\newcommand{\N}{\mathbb{N}}         % natural numbers
\newcommand{\Z}{\mathbb{Z}}         % integers
\newcommand{\Zpl}{\mathbb{Z}_{+}}   % positive integers
\newcommand{\Q}{\mathbb{Q}}         % rationals
\newcommand{\Qpl}{\mathbb{Q}_{+}}   % positive Rationals
\newcommand{\R}{\mathbb{R}}         % reals
\newcommand{\Rpl}{\mathbb{R}_{+}}   % positive Reals
\newcommand{\C}{\mathbb{C}}         % complex numbers
\newcommand{\F}{\mathbb{F}}         % field

% Words
\newcommand{\st}{\text{ such that }}
\newcommand{\wrt}{\text{ with respect to }}
\newcommand{\with}{\text{ with }}
\newcommand{\ie}{\text{, i.e. }}

% Operators
\newcommand{\abs}[1]{\left|#1\right|}                   % absolute value
\newcommand{\floor}[1]{\left\lfloor #1 \right\rfloor}   % floor
\newcommand{\ceil}[1]{\left\lceil #1 \right\rceil}      % ceiling

% Algebra 
\newcommand{\Syl}{\text{Syl}}           % set of Sylow-p subgroups 
\newcommand{\<}{\langle}                % \<x\>, subgroup generated by x 
\renewcommand{\>}{\rangle}
\newcommand{\id}{\text{id}}             % identity element
\newcommand{\order}[1]{\text{o}(#1)}    % order of an element
\let\oldcong\cong
\let\oldequiv\equiv
\renewcommand{\cong}{\oldequiv}
\renewcommand{\equiv}{\oldcong}

\makeatletter % cycle
\newcommand{\cyc}[1]{(\mathbf{\cyc@process#1\relax})}
\def\cyc@process#1#2\relax{%
	#1%
	\ifx\relax#2\relax
	\else
		\,\cyc@process#2\relax
	\fi
}
\makeatother

% Linear Algebra
\newcommand{\GL}{\text{GL}}                             % general linear group
\newcommand{\SL}{\text{SL}}                             % special linear group
\newcommand{\bmat}[1]{\begin{bmatrix}#1\end{bmatrix}}   % bracketed matrix
\newcommand{\rank}{\operatorname{rank}}                 % rank
\newcommand{\nullity}{\operatorname{nullity}}           % nullity

% Topology/Analysis
\newcommand{\ball}[2]{\text{B}_{#1}(#2)}  % B_r(x): open r-balls around x
\newcommand{\diam}{\text{diam}}           % diamter of a set in metric space 

% Misc Notation
\newcommand{\defeq}{\vcentcolon=}   % :=
\newcommand{\eqdef}{=\vcentcolon}   % =: 
\renewcommand{\bf}[1]{\textbf{#1}}


\renewcommand{\thesection}{\arabic{section}.}
\renewcommand{\thesubsection}{(\alph{subsection})}

\newtheorem{claim}{Claim}
\newtheorem*{lemma}{Lemma}

%% Headers & title setup
\newcommand{\course}{Math100B}
\newcommand{\myname}{Jay Ser}
\setlength{\headheight}{14.5pt}
\pagestyle{fancy}
\fancyhf{}
\renewcommand{\headrulewidth}{0.4pt}
\lhead{\course}
\rhead{\myname}
\cfoot{\thepage}
\setlength{\droptitle}{-4em} 
\title{\course\ - HW \#1}
\author{\myname}
\date{2026.01.11}

\begin{document}
\maketitle
\thispagestyle{fancy}

%------------------------------------------------------------------------------%

\section{} 
By the distributive law and cancellation law, $0r = (0 + 0)r = 0r + 0r \implies 0r = 0$

\section{} 
Take any $a \in \Q$. 
Then $a$ is the root of the linear polynomial $x - a$, hence $a$ is an algebraic number.  

\section{} 
$(x - (7 + \sqrt{2}))(x - (7 - \sqrt{2})) = (x - 7)^2 - (\sqrt{2})^2  = x^2 - 14x + 47 \in \Q[x]$. 
By the above computation, $7 + \sqrt{2}$ is a root of the polynomial $x^2 -14x + 47$, so it is an algebraic number over $\Q$. 

Similarly,
\begin{align*}
  & (x - (\sqrt{3} + \sqrt{-5})) (x - (\sqrt{3} - \sqrt{-5})) (x - (-\sqrt{3} + \sqrt{-5})) (x - (-\sqrt{3} - \sqrt{-5})) \\
  = & ((x - \sqrt{3})^2 - (\sqrt{-5})^2) ((x + \sqrt{3})^2 - (\sqrt{-5})^2) \\
  = & (x^2 - 2\sqrt{3}x + 8)(x^2 + 2\sqrt{3}x + 8) \\
  = & x^4 + 16x^2 + 64 - (2\sqrt{3}x)^2 \\
  = & x^4 + 4x^2 + 64 
\end{align*}
Clearly, $\sqrt{3} + \sqrt{-5}$ is a root of $x^4 + 4x^2 + 64 \in \Q[x]$, so $\sqrt{3} + \sqrt{-5}$ is an algebraic number over $\Q$. 

\section{} 
Take $a + b\sqrt{p}, c + d\sqrt{p} \in \Z[\sqrt{p}]$. 
\begin{itemize}
  \item $(a + b\sqrt{p}) + (c + d\sqrt{p}) = (a + c) + (b + d)\sqrt{p} \in \Z[p]$. 
  \item $(a + b\sqrt{p})(c + d\sqrt{p}) = (ac + bdp) + (bc + ad)\sqrt{p} \in \Z[p]$.  
\end{itemize}
Since multiplication and addition are commutative and associative in $\Z$, multiplication and addition in $\Z[\sqrt{p}]$ are commutative and associative too.
$0 + (a + b\sqrt{p}) = a + b\sqrt{p}$ so $0 \in \Z[\sqrt{p}]$ is the additive identity.
$\forall a + b\sqrt{p} \neq 0 \in \Z[\sqrt{p}]$, $(a + b\sqrt{p}) + (-a + (-b)\sqrt{p}) = 0$, so every nonzero element has a nonzero identity. 
$1 \cdot (a + b\sqrt{p}) = a + b\sqrt{p}$, so $1 \in \Z[\sqrt{p}]$ is the multiplicative identity. 
This verifies that $\Z[\sqrt{p}]$ is a commutative ring. 

Define the following norm function on $\Z[\sqrt{p}]$: 
$$\lambda: \Z[\sqrt{p}] \rightarrow \Z; \quad a + b\sqrt{p} \rightarrow a^2 - pb^2$$

The following calculation shows that $\lambda$ is a multiplicative function: 

\begin{align*}
  \lambda(a + b\sqrt{p}) \lambda(c + d\sqrt{p}) & = (a^2 - p b^2)(c^2 - p d^2) \\
                            & = a^2 c^2 + p^2 b^2 d^2 - p(b^2 c^2 + a^2 d^2) \\
  \lambda((a + b\sqrt{p})(c + d\sqrt{p})) & = \lambda(ac + pbd + (bc + ad)\sqrt{p}) \\
                            & = a^2 + 2pabcd + p b^2 d^2 - p(b^2 c^2 + 2abcd + a^2 d^2) \\
                            & = a^2 c^2 + p^2 b^2 d^2 - pb^2 c^2 - p a^2 d^2
\end{align*}
i.e., $\lambda(\alpha \beta) = \lambda(\alpha) \lambda(\beta) \, \forall \alpha, \beta \in \Z[\sqrt{p}]$.

Suppose $\alpha \in \Z[\sqrt{p}]$ is a unit. 
Then $\exists \beta \in \Z[\sqrt{p}] \st \alpha \beta = 1$.
Applying the norm on both sides of the equality yields 
$$\lambda(\alpha) \lambda(\beta) = 1$$
Since $\lambda$ maps to the integers, $\lambda(a) = \pm 1$.

Conversely, suppose $\lambda(\alpha) = \pm 1$. 
Write $\alpha = a + b\sqrt{p}$. 
Notice that $\lambda(a + b\sqrt{p}) = (a + b\sqrt{p})(a - b\sqrt{p})$.  
So if $\lambda(\alpha) = 1$, then $a - b\sqrt{p}$ is $\alpha$'s inverse;
if $\lambda(\alpha) = -1$, then $b\sqrt{p} - a$ is $\alpha$'s inverse.
This shows that $\alpha \in \Z[\sqrt{p}]$ is a unit $\iff \lambda(\alpha) = \pm 1$. 

One can define a similar norm function on the Gaussian integers, namely 
$$\lambda(a + bi) = a^2 + b^2$$
Notice that $\lambda$ now maps to only the nonnegative integers with $\lambda(\alpha) = 0 \iff \alpha = 0$. 
Furthermore, the exact same bidirectional proof for the units in $\Z[i]$ follow: 
$a + bi \in \Z[i]$ is a unit $\iff \lambda(a + bi) = 1$, where one need not consider the $\lambda(\alpha) = -1$ anymore. 

Specifically, $\lambda(a + bi) = a^2 + b^2 = 1 \iff a = \pm 1$ and $b = 0$ or $a = 0$ and $b = \pm 1$. 
One concludes that the only units of $\Z[i]$ are $\pm 1$ and $\pm i$. 

\section{} 
First, here are some trivial calculations.  
\begin{itemize}
  \item $\gamma^3 = 11\sqrt{2} + 9\sqrt{3}$, \quad, $\gamma^2 = 5 + 2\sqrt{6}$. 
  \item The irreducible (minimal) polynomial for $\gamma$ over $\Z$, and thus over $\Q$, is $x^4 - 10x^2 + 1$. 
  \item Since $\gamma$ is, well, the sum of $\sqrt{2}$ and $\sqrt{3}$, $\gamma \in \Q[\sqrt{2}, \sqrt{3}]$ and $\gamma \in \Z[\sqrt{2}, \sqrt{3}]$. 
    Thus $\Z[\gamma] \subset \Z[\sqrt{2}, \sqrt{3}]$ and $\Q[\gamma] \subset \Q[\sqrt{2}, \sqrt{3}]$. 
\end{itemize}

\subsection{$\Q[\gamma] = \Q[\sqrt{2}, \sqrt{3}]$}
It must be shown that $\sqrt{2}, \sqrt{3} \in \Q[\gamma]$. 
Well, $\gamma^3 = 11\sqrt{2} + 9\sqrt{3}$, so $\sqrt{2} = 2^{-1} (\gamma^3 - 9\gamma) \in \Q[\gamma]$ and $\sqrt{3} = -2^{-1} (\gamma^3 - 11\gamma) \in \Q[\gamma]$.

\subsection{$\Z[\gamma] \subsetneq \Z[\sqrt{2}, \sqrt{3}]$} 
It must be shown that either $\sqrt{2} \notin \Z[\gamma]$ or $\sqrt{3} \notin \Z[\gamma]$. 
I will cheat and use ring extensions. 
Since $x^4 - 10x^2 + 1$ is the irreducible polynomial for $\gamma$ over $\Z$, 
$$\Z[\gamma] \equiv \Z[x] / (x^4 - 10x^2 + 1)$$ 
This is by the First Isomorphism Theorem for rings:
consider the substition homomorphism 
$$\varphi: \Z[x] \rightarrow \Z[\gamma]; \quad x \rightarrow \gamma, \quad \varphi|_{\Z} = \id$$ 
Obviously $x^4 - 10x^2 + 1 \in \ker \varphi$, so $(x^4 - 10x^2 + 1) \subset \ker \varphi$. 
Conversely, suppose $f(x) \in \ker \varphi$. 
Dividing $f(x)$ with remainder by $x^4 - 10x^2 + 1$, 
$$f(x) = (x^4 - 10x^2 + 1)q(x) + r(x)$$
where $r(x) = 0$ or $\deg r < \deg x^4 - 10x^2 + 1$. 
Suppose $r(x) \neq 0$. 
Since $x^4 - 10x^2 + 1, f(x) \in \ker \varphi$, $r(x) \in \ker \varphi$. 
In other words, $r(\gamma) = 0$ and $r(x)$ has degree less than $x^4 -10x^2 + 1$. 
But $x^4 -10x^2 + 1$ is the minimal polynomial that has $\gamma$ as a root, which gives a contradiction. 
So $r(x) = 0$, and $f(x) \in (x^4 - 10x^2 + 1)$. 
This shows that $\ker \varphi = (x^4 - 10x^2 + 1)$, and thus $\Z[x] / (x^4 - 10x^2 + 1) \equiv \Z[\gamma]$ by the First Isomorphism Theorem. 

Since the irreducible polynomial for $\gamma$ over $\Z$ is monic and has degree 4, $\{1, \gamma, \gamma^2, \gamma^3\}$ form a basis for $\Z[\gamma]$ over $\Z$ 
(view $\Z[\gamma]$ as $\Z[x] / (x^4 - 10x^2 + 1)$ and perform division with remainder by $x^4 - 10x^2 + 1$ in the quotient ring).  

Suppose $\sqrt{2} \in \Z[\gamma]$. 
Then $\exists a_0, \ldots, a_3 \in \Z \st$
$$\sqrt{2} = a_3 \gamma^3 + a_2 \gamma^2 + a_1 \gamma + a_0$$
Substituting the calculated values for $\gamma^2$ and $\gamma^3$ yields a system of equations that can easily be solved: 
On the right hand side, $\gamma^2$ introduces $\sqrt{6}$ that is not present in any other term, so $a_2 = 0$. 
Since the left hand side doesn't have any integer part, the sum of integer terms in the left hand side must equal 0. 
Namely, $5a_2 + a_0 = 0 \implies a_0 = 0$. 
Finally, dealing with the coefficients of $\sqrt{2}$ and $\sqrt{3}$ yields the equations $11a_3 + a_1 = 1$ and $9a_3 + a_1 = 0$, which implies $2a_3 = 1$. 
This contradicts the fact that no integer $a_3$ solves $2a_2 = 1$, and thus $\sqrt{2} \neq \Z\gamma]$. 

\section{} 
$a \in \Z_n$ is a unit $\iff \exists a' \in \Z_n \st aa' = 1 \iff \exists a' \in \Z \st aa' = 1 \mod n$. 
An elementary result from number theory is that the last statement is true if and only if $(a, n) = 1$. 
This shows that the units in $\Z_n$ are equivalence classes of $\Z$ that are prime to $n$. 
Since prime numbers are the only integers that are coprime to every number less than it, it follows that $\Z_n$ is a field if and only if $n$ is prime. 

\section{} 
$f(x) \defeq x^2 + x + 1$ is monic, so one can divide $g(x) \defeq x^4 + 3x^3 + x^2 + 7$ with remainder by $f(x)$: 
$$x^4 + 3x^3 + x^2 + 7x + 5 = (x^2 + 2x - 2)(x^2 + x + 1) + 7x + 7$$ 
where $r(x) \defeq 7x + 7$ is the remainder. 

Reducing modulo $n$, one sees $f(x) \mid g(x)$ in $\Z / n\Z \iff r(x) = 0$ in $\Z / n\Z$.
Clearly, $7x + 7 = 0 \mod n \iff n$ is a multiple of 7, i.e. $f(x) \mid g(x) \iff 7 \mid n$. 

\section{}
Take $f(x) = \sum_{i = 0}^{\infty} a_i x^i, g(x) = \sum_{i = 0}^{\infty} b_i x^i \in F[[t]]$. 
\begin{itemize}
  \item $f(x) + g(x) = \sum_{i = 0}^{\infty} (a_i + b_i) x^i \in F[[t]]$,
  \item $f(x) g(x) = \sum_{i = 0}^{\infty} \sum_{j = 0}^{i} a_j x^j b_{i - j} x^{i - j} = \sum_{i = 0}^{\infty} \sum_{j = 0}^{i} a_j b_{i - j} x^i \in F[[t]]$ 
\end{itemize}
That $F$ is a field, thus equipped with commutative and associative $+$ and $\times$, allows for an easy verification that addition and multiplication are both commutative and associative in $F[[t]]$.
As with the polynomial ring, it is easily verified that 0 and 1 are the additive and multiplicative identities of $F[[t]]$, respectively. 
Every $f(x) \in F[[t]]$ has an additive inverse, namely the polynomial whose coefficient at each index is the additive inverse in $F$ of the corresponding coefficient in $f(x)$. 
This verifies that $F[[t]]$ is a commutative ring.

The following shows that $f(x) = \sum_{i = 0}^{\infty} a_i x^i \in F[[t]]$ is a unit $\iff a_0 \neq 0$.
Suppose $a_0 = 0$. 
Then clearly there is no $g(x) = \sum_{i = 0}^{\infty} b_i x^i \st f(x)g(x) = 1$ since the constant term is just 0.
This shows that if $f(x)$ is a unit, then $a_0 \neq 0$. 

Next, take any $f(x) = \sum_{i = 0}^{\infty} a_i x^i \in F[[t]]$ with $a_i \neq 0$. 
Inductively define $g(x) = \sum_{i = 0}^{\infty} b_i x^i \in F[[t]]$:
For $i = 0$, $b_0 \defeq a_{0}^{-1}$. 
For $i > 0$, suppose $b_0, \ldots, b_i$ have been defined. 
Then let 
$$b_{i + 1} \defeq a_{0}^{-1}(-\sum_{k = 0}^{i} a_{i - k + 1} b_k)$$

With this construction, one sees via the definition of multiplication above that $f(x)g(x) = 1$: 
If $f(x)g(x) = \sum_{i = 0}^{\infty} c_i x^i$, then $c_0 = a_0 b_0 = 1$, and for all $i > 0$, 
$$b_i = a_{0}^{-1} (-\sum_{k = 0}^{i - 1} a_{i - k}{b_k}) \implies  c_i = \sum_{k = 0}^{i} a_{k} b_{i - k} = 0 $$
** sorry for the messy indexing towards the end, got a rush of sleepiness...

\end{document}

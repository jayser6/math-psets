
\documentclass[12pt]{article}

%% Basic document formatting
\usepackage{amsmath, amsthm, amssymb, amsfonts}
\usepackage{mathtools}
\usepackage{xspace}
\usepackage{thmtools}
\usepackage{graphicx}
\usepackage{setspace}
\usepackage{fancyhdr}
\usepackage{titling}
\usepackage[left=0.4in,right=0.4in,top=1in,bottom=1in]{geometry}
\usepackage{float}
\usepackage{tabularx}
\usepackage[utf8]{inputenc}
\usepackage[english]{babel}
\usepackage{framed}
\usepackage[dvipsnames]{xcolor}
\usepackage{environ}
\usepackage{tcolorbox}
\tcbuselibrary{theorems,skins,breakable}

\usepackage{enumitem}
\setlist[enumerate]{leftmargin=*}
\setlist[enumerate,1]{labelindent=\parindent}
\setlist[enumerate,2]{labelindent=0pt}

% Blackboard Bold
\newcommand{\N}{\mathbb{N}}         % natural numbers
\newcommand{\Z}{\mathbb{Z}}         % integers
\newcommand{\Zpl}{\mathbb{Z}_{+}}   % positive integers
\newcommand{\Q}{\mathbb{Q}}         % rationals
\newcommand{\Qpl}{\mathbb{Q}_{+}}   % positive Rationals
\newcommand{\R}{\mathbb{R}}         % reals
\newcommand{\Rpl}{\mathbb{R}_{+}}   % positive Reals
\newcommand{\C}{\mathbb{C}}         % complex numbers
\newcommand{\F}{\mathbb{F}}         % field

% Words
\newcommand{\st}{\text{ such that }}
\newcommand{\wrt}{\text{ with respect to }}
\newcommand{\with}{\text{ with }}
\newcommand{\ie}{\text{, i.e. }}

% Operators
\newcommand{\abs}[1]{\left|#1\right|}                   % absolute value
\newcommand{\floor}[1]{\left\lfloor #1 \right\rfloor}   % floor
\newcommand{\ceil}[1]{\left\lceil #1 \right\rceil}      % ceiling

% Algebra 
\newcommand{\Syl}{\text{Syl}}           % set of Sylow-p subgroups 
\newcommand{\<}{\langle}                % \<x\>, subgroup generated by x 
\renewcommand{\>}{\rangle}
\newcommand{\id}{\text{id}}             % identity element
\newcommand{\order}[1]{\text{o}(#1)}    % order of an element
\let\oldcong\cong
\let\oldequiv\equiv
\renewcommand{\cong}{\oldequiv}
\renewcommand{\equiv}{\oldcong}

\makeatletter % cycle
\newcommand{\cyc}[1]{(\mathbf{\cyc@process#1\relax})}
\def\cyc@process#1#2\relax{%
	#1%
	\ifx\relax#2\relax
	\else
		\,\cyc@process#2\relax
	\fi
}
\makeatother

% Linear Algebra
\newcommand{\GL}{\text{GL}}                             % general linear group
\newcommand{\SL}{\text{SL}}                             % special linear group
\newcommand{\bmat}[1]{\begin{bmatrix}#1\end{bmatrix}}   % bracketed matrix
\newcommand{\rank}{\operatorname{rank}}                 % rank
\newcommand{\nullity}{\operatorname{nullity}}           % nullity

% Topology/Analysis
\newcommand{\ball}[2]{\text{B}_{#1}(#2)}  % B_r(x): open r-balls around x
\newcommand{\diam}{\text{diam}}           % diamter of a set in metric space 

% Misc Notation
\newcommand{\defeq}{\vcentcolon=}   % :=
\newcommand{\eqdef}{=\vcentcolon}   % =: 
\renewcommand{\bf}[1]{\textbf{#1}}


\renewcommand{\thesection}{\arabic{section}.}
\renewcommand{\thesubsection}{(\alph{subsection})}

\newtheorem{claim}{Claim}
\newtheorem*{lemma}{Lemma}

%% Headers & title setup
\newcommand{\course}{Math100B}
\newcommand{\myname}{Jay Ser}
\setlength{\headheight}{14.5pt}
\pagestyle{fancy}
\fancyhf{}
\renewcommand{\headrulewidth}{0.4pt}
\lhead{\course}
\rhead{\myname}
\cfoot{\thepage}
\setlength{\droptitle}{-4em} 
\title{\course\ - HW \#1}
\author{\myname}
\date{2026.01.11}

\begin{document}
\maketitle
\thispagestyle{fancy}

%------------------------------------------------------------------------------%

\section{} 
By the distributive law and cancellation law, $0r = (0 + 0)r = 0r + 0r \implies 0r = 0$

\section{} 
Take any $a \in \Q$. 
Then $a$ is the root of the linear polynomial $x - a$, hence $a$ is an algebraic number.  

\section{} 
$(x - (7 + \sqrt{2}))(x - (7 - \sqrt{2})) = (x - 7)^2 - (\sqrt{2})^2  = x^2 - 14x + 47 \in \Q[x]$. 
By the above computation, $7 + \sqrt{2}$ is a root of the polynomial $x^2 -14x + 47$, so it is an algebraic number over $\Q$. 

Similarly,
\begin{align*}
  & (x - (\sqrt{3} + \sqrt{-5})) (x - (\sqrt{3} - \sqrt{-5})) (x - (-\sqrt{3} + \sqrt{-5})) (x - (-\sqrt{3} - \sqrt{-5})) \\
  = & ((x - \sqrt{3})^2 - (\sqrt{-5})^2) ((x + \sqrt{3})^2 - (\sqrt{-5})^2) \\
  = & (x^2 - 2\sqrt{3}x + 8)(x^2 + 2\sqrt{3}x + 8) \\
  = & x^4 + 16x^2 + 64 - (2\sqrt{3}x)^2 \\
  = & x^4 + 4x^2 + 64 
\end{align*}
Clearly, $\sqrt{3} + \sqrt{-5}$ is a root of $x^4 + 4x^2 + 64 \in \Q[x]$, so $\sqrt{3} + \sqrt{-5}$ is an algebraic number over $\Q$. 

\section{} 
\section{} 
\section{} 
$a \in \Z_n$ is a unit $\iff \exists a' \in \Z_n \st aa' = 1 \iff \exists a' \in \Z \st aa' = 1 \mod n$. 
An elementary result from number theory is that the last statement is true if and only if $(a, n) = 1$. 
This shows that the units in $\Z_n$ are equivalence classes of $\Z$ that are prime to $n$. 

\section{} 
$f(x) \defeq x^2 + x + 1$ is monic, so one can divide $g(x) \defeq x^4 + 3x^3 + x^2 + 7$ with remainder by $f(x)$: 
$$x^4 + 3x^3 + x^2 + 7x + 5 = (x^2 + 2x - 2)(x^2 + x + 1) + 7x + 7$$ 
where $r(x) \defeq 7x + 7$ is the remainder. 

Reducing modulo $n$, one sees $f(x) \mid g(x)$ in $\Z / n\Z \iff r(x) = 0$ in $\Z / n\Z$.
Clearly, $7x + 7 = 0 \mod n \iff n$ is a multiple of 7, i.e. $f(x) \mid g(x) \iff 7 \mid n$. 

\section{}
\end{document}

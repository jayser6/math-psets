\documentclass[12pt]{article}

%% Basic document formatting
\usepackage{amsmath, amsthm, amssymb, amsfonts}
\usepackage{mathtools}
\usepackage{xspace}
\usepackage{thmtools}
\usepackage{graphicx}
\usepackage{setspace}
\usepackage{fancyhdr}
\usepackage{titling}
\usepackage[left=0.4in,right=0.4in,top=1in,bottom=1in]{geometry}
\usepackage{float}
\usepackage{tabularx}
\usepackage[utf8]{inputenc}
\usepackage[english]{babel}
\usepackage{framed}
\usepackage[dvipsnames]{xcolor}
\usepackage{environ}
\usepackage{tcolorbox}
\tcbuselibrary{theorems,skins,breakable}

\usepackage{enumitem}
\setlist[enumerate]{leftmargin=*}
\setlist[enumerate,1]{labelindent=\parindent}
\setlist[enumerate,2]{labelindent=0pt}

% Blackboard Bold
\newcommand{\N}{\mathbb{N}}         % natural numbers
\newcommand{\Z}{\mathbb{Z}}         % integers
\newcommand{\Zpl}{\mathbb{Z}_{+}}   % positive integers
\newcommand{\Q}{\mathbb{Q}}         % rationals
\newcommand{\Qpl}{\mathbb{Q}_{+}}   % positive Rationals
\newcommand{\R}{\mathbb{R}}         % reals
\newcommand{\Rpl}{\mathbb{R}_{+}}   % positive Reals
\newcommand{\C}{\mathbb{C}}         % complex numbers
\newcommand{\F}{\mathbb{F}}         % field

% Words
\newcommand{\st}{\text{ such that }}
\newcommand{\wrt}{\text{ with respect to }}
\newcommand{\with}{\text{ with }}
\newcommand{\ie}{\text{, i.e. }}

% Operators
\newcommand{\abs}[1]{\left|#1\right|}                   % absolute value
\newcommand{\floor}[1]{\left\lfloor #1 \right\rfloor}   % floor
\newcommand{\ceil}[1]{\left\lceil #1 \right\rceil}      % ceiling

% Algebra 
\newcommand{\Syl}{\text{Syl}}           % set of Sylow-p subgroups 
\newcommand{\<}{\langle}                % \<x\>, subgroup generated by x 
\renewcommand{\>}{\rangle}
\newcommand{\id}{\text{id}}             % identity element
\newcommand{\order}[1]{\text{o}(#1)}    % order of an element
\let\oldcong\cong
\let\oldequiv\equiv
\renewcommand{\cong}{\oldequiv}
\renewcommand{\equiv}{\oldcong}

\makeatletter % cycle
\newcommand{\cyc}[1]{(\mathbf{\cyc@process#1\relax})}
\def\cyc@process#1#2\relax{%
	#1%
	\ifx\relax#2\relax
	\else
		\,\cyc@process#2\relax
	\fi
}
\makeatother

% Linear Algebra
\newcommand{\GL}{\text{GL}}                             % general linear group
\newcommand{\SL}{\text{SL}}                             % special linear group
\newcommand{\bmat}[1]{\begin{bmatrix}#1\end{bmatrix}}   % bracketed matrix
\newcommand{\rank}{\operatorname{rank}}                 % rank
\newcommand{\nullity}{\operatorname{nullity}}           % nullity

% Topology/Analysis
\newcommand{\ball}[2]{\text{B}_{#1}(#2)}  % B_r(x): open r-balls around x
\newcommand{\diam}{\text{diam}}           % diamter of a set in metric space 

% Misc Notation
\newcommand{\defeq}{\vcentcolon=}   % :=
\newcommand{\eqdef}{=\vcentcolon}   % =: 
\renewcommand{\bf}[1]{\textbf{#1}}


\renewcommand{\thesection}{\arabic{section}.}
\renewcommand{\thesubsection}{(\alph{subsection})}

\newtheorem{claim}{Claim}
\newtheorem*{lemma}{Lemma}

%% Headers & title setup
\newcommand{\course}{Math200A}
\newcommand{\myname}{Jay Ser}
\setlength{\headheight}{14.5pt}
\pagestyle{fancy}
\fancyhf{}
\renewcommand{\headrulewidth}{0.4pt}
\lhead{\course}
\rhead{\myname}
\cfoot{\thepage}
\setlength{\droptitle}{-4em} 
\title{\course\ - HW \#9}
\author{\myname}
\date{2025.12.04}

\begin{document}
\maketitle
\thispagestyle{fancy}

%------------------------------------------------------------------------------%

\section{}
\subsection{}
Composing the homomorphisms $\pi: R \rightarrow R/I$ and $\iota: R/I \rightarrow (R/I) [x]$ yields a homomorphism $\varphi: R \rightarrow (R/I) [x]$.
By the substition principle, there is a unique homomorphism
$$\varPhi: R[x] \rightarrow (R/I) [x] \quad \text{ satisfying } \left.\varPhi\right|_{R} = \varphi, \quad x \mapsto x$$

$\varPhi$ is surjective because $\pi$ is surjective.
$f(x) \in \ker \varPhi \iff$ every coefficient of $f$ is in $I \iff f(x) \in I[x]$, so $\ker \varPhi = I[x]$.

\subsection{}
Suppose $P \in \spec R$, $f(x)g(x) \in P[x]$, and $f(x) \notin P[x]$.
Write
$$f(x) = \sum_{i = 0}^{n} a_i x^i, \quad g(x) = \sum_{i = 0}^{m} b_i x^i, \quad f(x)g(x) = \sum_{i = 0}^{n + m} c_i x^i = \sum_{i = 0}^{n + m}\sum_{j = 0}^{i} a_j b_{i - j} x^i$$
and let $k$ be the smallest integer such that $a_k \notin P$.
By assumption, $a_0, \ldots, a_{k - 1} \in P$.
Now, induct on $0 \leq l \leq m$ to show all $b_l \in P$.
Since $f(x)g(x) \in P[x]$, $c_k = \sum_{j = 0}^{k} a_j b_{k - j} \in P$. The terms $a_0 b_k, a_1 b_{k - 1}, \ldots, a_{k - 1}b_1$ of the sum are in $P$, so the only remaining term of the sum $a_k b_0 \in P$ as well. $a_k \notin P \implies b_0 \in P$, which verifies the base case.

Suppose $b_0, \ldots, b_{l - 1} \in P$ for $l - 1 < m$. Then $c_{k + l} = \sum_{j = 0}^{k + l} a_j b_{k + l - j} \in P$, and the terms of the sum
$$a_0 b_{k + l}, a_1 b_{k + l - 1}, \ldots, a_{k - 1} b_{l + 1}, a_{k + 1} b_{l - 1}, a_{k + 2} b_{l - 2}, \ldots, a_{k + l} b_0$$
are all either 0 (i.e. if $k + 1 > n$) or in $P$, so the only term of the sum not in this list $a_k b_l \in P$ as well. $a_k \neq P \implies b_l \in P$.

This shows every coefficient of $g(x) \in P \ie g(x) \in P[x]$, so $P[x] \in \spec R[x]$

\subsection{}

\end{document}

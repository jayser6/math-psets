\documentclass[12pt]{article}

%% Basic document formatting
\usepackage{amsmath, amsthm, amssymb, amsfonts}
\usepackage{mathtools}
\usepackage{xspace}
\usepackage{thmtools}
\usepackage{graphicx}
\usepackage{setspace}
\usepackage{fancyhdr}
\usepackage{titling}
\usepackage[left=0.4in,right=0.4in,top=1in,bottom=1in]{geometry}
\usepackage{float}
\usepackage{tabularx}
\usepackage[utf8]{inputenc}
\usepackage[english]{babel}
\usepackage{framed}
\usepackage[dvipsnames]{xcolor}
\usepackage{environ}
\usepackage{tcolorbox}
\tcbuselibrary{theorems,skins,breakable}

\usepackage{enumitem}
\setlist[enumerate]{leftmargin=*}
\setlist[enumerate,1]{labelindent=\parindent}
\setlist[enumerate,2]{labelindent=0pt}

% Blackboard Bold
\newcommand{\N}{\mathbb{N}}         % natural numbers
\newcommand{\Z}{\mathbb{Z}}         % integers
\newcommand{\Zpl}{\mathbb{Z}_{+}}   % positive integers
\newcommand{\Q}{\mathbb{Q}}         % rationals
\newcommand{\Qpl}{\mathbb{Q}_{+}}   % positive Rationals
\newcommand{\R}{\mathbb{R}}         % reals
\newcommand{\Rpl}{\mathbb{R}_{+}}   % positive Reals
\newcommand{\C}{\mathbb{C}}         % complex numbers
\newcommand{\F}{\mathbb{F}}         % field

% Words
\newcommand{\st}{\text{ such that }}
\newcommand{\wrt}{\text{ with respect to }}
\newcommand{\with}{\text{ with }}
\newcommand{\ie}{\text{, i.e. }}

% Operators
\newcommand{\abs}[1]{\left|#1\right|}                   % absolute value
\newcommand{\floor}[1]{\left\lfloor #1 \right\rfloor}   % floor
\newcommand{\ceil}[1]{\left\lceil #1 \right\rceil}      % ceiling

% Algebra 
\newcommand{\Syl}{\text{Syl}}           % set of Sylow-p subgroups 
\newcommand{\<}{\langle}                % \<x\>, subgroup generated by x 
\renewcommand{\>}{\rangle}
\newcommand{\id}{\text{id}}             % identity element
\newcommand{\order}[1]{\text{o}(#1)}    % order of an element
\let\oldcong\cong
\let\oldequiv\equiv
\renewcommand{\cong}{\oldequiv}
\renewcommand{\equiv}{\oldcong}

\makeatletter % cycle
\newcommand{\cyc}[1]{(\mathbf{\cyc@process#1\relax})}
\def\cyc@process#1#2\relax{%
	#1%
	\ifx\relax#2\relax
	\else
		\,\cyc@process#2\relax
	\fi
}
\makeatother

% Linear Algebra
\newcommand{\GL}{\text{GL}}                             % general linear group
\newcommand{\SL}{\text{SL}}                             % special linear group
\newcommand{\bmat}[1]{\begin{bmatrix}#1\end{bmatrix}}   % bracketed matrix
\newcommand{\rank}{\operatorname{rank}}                 % rank
\newcommand{\nullity}{\operatorname{nullity}}           % nullity

% Topology/Analysis
\newcommand{\ball}[2]{\text{B}_{#1}(#2)}  % B_r(x): open r-balls around x
\newcommand{\diam}{\text{diam}}           % diamter of a set in metric space 

% Misc Notation
\newcommand{\defeq}{\vcentcolon=}   % :=
\newcommand{\eqdef}{=\vcentcolon}   % =: 
\renewcommand{\bf}[1]{\textbf{#1}}


\renewcommand{\thesection}{\arabic{section}.}
\renewcommand{\thesubsection}{(\alph{subsection})}

\newtheorem{claim}{Claim}
\newtheorem*{lemma}{Lemma}

%% Headers & title setup
\newcommand{\course}{Math200B}
\newcommand{\myname}{Jay Ser}
\setlength{\headheight}{14.5pt}
\pagestyle{fancy}
\fancyhf{}
\renewcommand{\headrulewidth}{0.4pt}
\lhead{\course}
\rhead{\myname}
\cfoot{\thepage}
\setlength{\droptitle}{-4em} 
\title{\course\ - HW \#1}
\author{\myname}
\date{2026.01.16}

\begin{document}
\maketitle
\thispagestyle{fancy}

%------------------------------------------------------------------------------%

\section{} 
\subsection{} 
Suppose $D$ is a Bezout domain. 
Take $a, b \in D \setminus 0$. 
Then $\exists d \in D \st \<a, b\> = \<d\>$. 
$\<a\> \subset \<d\> \implies d \mid a$, and similarly, $d \mid b$. 
By definition of $d$, $d \in \<a, b\>$, which proves the forward direction. 

Next, suppose for arbitrary $a, b \in D \setminus 0$, $\exists d \in D \st d \mid a$, $d \mid b$, and $d \in \<a, b\>$.
The divisibility condition says $\<a, b\> \subset \<d\>$. 
Vice versa, $d \in \<a, b\> \implies \<d\> \subset \<a, b\>$. 
So for arbitrary $a, b \in D \setminus 0$, $\exists d \in D \st \<a, b\> = \<d\>$. 
Of course, if $a = 0$, then $\<a, b\> = \<b\>$. 
This shows $D$ is a Bezout Domain.

\subsection{} 
Induct on $n$, the number of elements generating $I \lhd D$.  
If $n = 1$, $I$ is a principal ideal by definition. 
Suppose ideals generated by at most $n - 1$ elements are principal and let 
$$I = \<a_1, a_2, \ldots, a_n\>$$ 
for some $a_1, \ldots, a_n \in D$. 
$\<a_1, \ldots, a_{n - 1}\> \subset I$, and by the inductive hypothesis, $\exists d \in D \st$ 
$$\<d\> = \<a_1, a_2, \ldots, a_{n - 1}\>$$
Namely, since $d \in \<a_1, \ldots, a_{n - 1}\>$, $d \in I$. 
Thus $I = \<d, a_n\>$. 
Since $D$ is a Bezout Domain, $\exists d' \in D \st$
$$\<d'\> = \<d, a_n\> = I$$ 
This shows $I$ is principal.

\subsection{} 
The forward direction is trivial. 
Suppose $D$ is a UFD and a Bezout Domain and take any $I \lhd D$. 
Since $D$ is a UFD, every element of $I$ has a unique prime factorization. 
Let $a \in I$ be an element that has the least number of (not necessarily distinct) prime factors. 
Note that such an element is unique up to associates: 
suppose 
$$a = u p_1 p_2 \cdots p_n, \quad a' = v q_1 q_2 \cdots q_n$$
are factorizations of two nonassociate elements of $I$ with the least number of prime factors. 
Then their greatest common divisor has strictly less prime factors than $a_1$ and $a_2$.
Because $D$ is a Bezout Domain, the greatest common divisor is in $I$, contradicting the assumption on $a$ and $a'$. 

Now, take any $b \in I$. 
Let $d \in D$ be the element satisfying $\<d\> = \<a, \>$. 
Because $d$ divides $a$, $d$ has at most the number of prime factors that $a$ has. 
But $d \in \<a, b\> \subset I$. 
Thus, by the assumption on $a$, $d$ has the exact same prime factors as $a$, i.e., 
$$\<d\> = \<a\> \subset \<a, b\> = \<d\>$$
which shows $\<a\> = \<a, b\>$. 
Since this holds for all $b \in I$, one concludes $I = \<a, b\>$. 

\section{} 
In the ring $A = \Q[x, xy, xy^2, \ldots]$, the chain 
$$\<x\> \subset \<x, xy\> \subset \<x, xy, xy^2\> \subset \ldots$$ 
is a strictly increasing chain because no power of $y$ is in $A$.  
Thus $A$ is not Noetherian.

\section{} 
\subsection{} 
Since $r / s$ is a root, the following equality holds: 
$$-a_0 = a_n (r / s)^n + a_{n - 1} (r / s)^{n - 1} + \ldots + a_1 (r / s)$$
Multiplying both sides by $s^n$, 
$$-s^n a_0 = a_n r^n + a_{n - 1} r^{n - 1}s + a_{n - 2} r^{n - 2}s^2 + \ldots + a_1 r s^{n - 1}$$ 
$s$ divides the left hand side, and $s$ divides all the terms on the right hand side except $a_0 r^n$, so $s \mid a_0 r^n$. 
$s$ and $r$ share no prime factors, so $s \mid a_0$. 
Similarly, $r$ divides the right hand side, and $r$ and $s$ share no common prime factor, so $r \mid a_0$. 

\subsection{}
Suppose $r / s \in D$.
Then $f(x) \defeq x - r/s \in D[x]$ is a monic polynomial with $r/s$ as a root. 

Conversely, suppose $f(x) \defeq a_n x^n + \ldots + a_0 \in D[x]$, where $a_n = 1$, has $r / s$ as a root. 
If $s \nmid r$, then define 
$$d \defeq \gcd(r, s), \quad r' \defeq r / d, \quad s' \defeq s / d$$ 
$d$ is nonassociate to $s$, so $s'$ is nonunit. 
$f(r' / s') = 0$, so $s' \mid a_0 = 1$ by the previous problem. 
This is a contradiction, so $s$ must divide $r$. 

\subsection{} 
Denote $R = \Z[2\sqrt{2}]$.
Consider $f(x) = x^2 - 2 \in R[x]$.  
$\sqrt{2} = \frac{2\sqrt{2}}{2}$ is a root of $f(x)$ and its numerator and denominator is in $R$, but $\sqrt{2}$ itself is not in $R$. 
By part (b), $R$ is not a UFD.

\section{} 

\section{}

\section{}

\section{} 
Suppose $p = (a + bi)(c + di)$, where $a + bi$ and $c + di$ are not units. 
Using the fact that the norm function on $\Z[i]$ is multiplicative, obtain
$$p^2 = (a^2 + b^2)(c^2 + d^2)$$
Since the two factors of $p$ are not units and the norm function maps to $\Znn$, it follows that $a^2 + b^2 = c^2 + d^2 = p$. 

Next, suppose $\exists a, b \in \Z \st a^2 + b^2 = p$. 
Then 
\begin{alignat*}{3}
             & b^2         &\;&\cong -a^2        &&\pmod{p} \\ 
  \implies{} & b^{-2}b^2   &  &\cong -(b^{-1}a)^2 &&\pmod{p} \\ 
  \implies{} & (b^{-1}a)^2 &  &\cong -1           &&\pmod{p}
\end{alignat*}
where $b^{-1}$ is guaranteed to exist because $\Z / p\Z$ is a multiplicative group. 
So the equation $x^2 \cong -1 \mod p$ does indeed have a solution. 

Finally, assume $b^2 \cong -1 \mod p$ for some $0 < b < p$. 
Then $p \mid b^2 + 1 = (b - i)(b + i)$. 
Suppose $p$ is irreducible in $\Z[i]$. 
Then $p$ is prime in $\Z[i]$, so $p \mid (b - i)$ or $p \mid (b + i)$. 
Without loss of generality, suppose $p \mid b - i$. 
Write $pq = b - i$ for some $q \in \Z[i]$ that is not a unit. 
Norming both sides, obtain 
$$a p^2 = b^2 + 1$$ 
where $a \defeq \lambda(q) > 1$. 
This contradicts the assumption that $b < p$. 
So $p$ is not irreducible in $\Z[i]$. 

\section{} 


\end{document}
